\documentclass[12pt]{article}
\usepackage[]{blindtext}
\usepackage[letterpaper, total{216mm, 279mm}]{}
\usepackage{amssymb,amsmath,amsfonts,verbatim}
\usepackage[breakable, skins]{tcolorbox}
\usepackage[parfill]{parskip}
\usepackage[english]{babel}
\usepackage{mathtools, amsthm}
\usepackage{amsfonts}
\usepackage{amssymb}
\usepackage{mathrsfs}
\usepackage{verbatim}

\newtheorem{notes}{Notes}[section]
\newtheorem{prob}[notes]{Problems}
\newtheorem{thm}{Theorem}[section]
\newtheorem{cor}[thm]{Corollary}
\newtheorem{lem}[thm]{Lemma}
\newtheorem{defn}[notes]{Definition}
\newtheorem{rem}[notes]{Remark}
\newtheorem{prop}[thm]{Proposition}

\newcommand{\rl}{\mathbb{R}}
\newcommand{\bz}{\mathbb{Z}}
\newcommand{\id}{\text{id}}
\newcommand{\dprime}{{\prime\prime}}
\newcommand{\xprime}{X^\prime}
\newtcolorbox{mybox}[2][]{
    arc=0mm, enhanced, frame hidden, breakable
}
\newcommand{\qedbox}{$\hfill\blacksquare$}
\newcommand\nindent{.5pt}
\newcommand\noverline[1]{%
  \kern\nindent\overline{\kern-\nindent#1\kern-\nindent}\kern\nindent}

\setlength{\topmargin}{-.65in}
\setlength{\textwidth}{170mm} 
\setlength{\textheight}{240mm}
\setlength{\oddsidemargin}{-3mm}

\title{Algebra I\\
\large Homework 3
}
\author{Nutan Nepal}
\newcommand{\mR}{\mathbb{R}}
\newcommand{\mz}{\mathbb{Z}}
\newcommand{\ds}{\displaystyle}
\newcommand{\al}{\alpha}

\begin{document}
\maketitle
\makebox[\linewidth]{\rule{190mm}{.5pt}}
\vspace{0mm}
\begin{enumerate}

\item[(5.4 - 8)] Assume that $x$, $y$ both commute with
    $[x, y]$. Show that $(xy)^n = x^ny^n[y, x]^{n\choose 2}$
    for any positive integer $n$.
\begin{mybox}
    
    We first note that $yx=xy[y,x]=x[y,x]y=[y,x]xy$.
    Assume that the statement
    $$P(k):\hspace*{5mm}(xy)^k = x^ky^k[y, x]^{k\choose 2}$$
    is true for some integer $k>1$. Then,
    
\end{mybox}

\item[(5.5 - 11)] Classify groups of order 28.

\begin{mybox}

\end{mybox}


\item[(6.1 - 6)] Show that if $G/Z(G)$ is nilpotent then
    $G$ is nilpotent.

\begin{mybox}
    
\end{mybox}

\item[(6.3 - 7)] Show that the quaternion group
    $\mathcal{Q}_8$ can be presented by $\{a, b|\ a^2 =
    b^2, a^{-1}ba = b^{-1}\}.$
\begin{mybox}
    
\end{mybox}

\item[(7.1 - 8)] Find the center of the real Hamiltonian
    Quaternions $\mathbb{H}$. Prove that
    $\{a + bi|a, b\in R\}$ is a subring of $\mathbb{H}$
    which is a field but is not contained in the
    center of $\mathbb{H}$.
\begin{mybox}

\end{mybox}

\item[(7.1 - 13)] An element $a\in R$ is called
    nilpotent if $x^m = 0$ for some $m \in\mathbb{Z}^+$.
    \begin{enumerate}
        \item Show that if $n = a^kb$ for some integers
        $a$, $b$, then $ab$ is nilpotent in $\bz/n\bz$.

        \item If $a\in\bz$, show that the element
        $\overline{a}\in\bz/n\bz$ iff every prime divisor of $n$ is
        also a divisor
        of $a$. In particular, find all nilpotent elements
        of $\bz/72\bz$.

        \item Let $R$ be the ring of functions from a
        nonempty set $X$ to a field $F$ . Prove that $R$
        contains no nilpotent elements.
    \end{enumerate}
\begin{mybox}

\end{mybox}

\item[(7.2 - 7)] Show that the center of the ring $M_n(R)$
    is $R\cdot I$, where $I = diag(1,\ldots,1)$.
\begin{mybox}
    
\end{mybox}

\item[(7.3 - 4)] Find all ring homomorphisms from
    $\bz$ to $\bz/30\bz$. In each case, describe
    the kernel and the image.
\begin{mybox}

\end{mybox}

\item[(7.3 - 18)] Prove that the intersection $I\cap J$
    of ideals $I$, $J$ of a ring $R$ is also an ideal of
    $R$. Let $\{I_\alpha\}_{\alpha\in S}$ be a collection
    of ideals of $R$. Show that $\bigcap_{\alpha\in S}
    {I_\alpha}$ is an ideal of $R$.
\begin{mybox}

\end{mybox}

\item[(7.3 - 29)] Let $R$ be a commutative ring. Prove that
    the set of nilpotent elements of
    $R$ form an ideal—called the nilradical $\mathfrak{N}(R)$.
\begin{mybox}
    
\end{mybox}

\end{enumerate}
\end{document}