\documentclass[12pt]{article}
\usepackage[]{blindtext}
\usepackage[letterpaper, total{216mm, 279mm}]{}
\usepackage{amssymb,amsmath,amsfonts,verbatim}
\usepackage[breakable, skins]{tcolorbox}
\usepackage[parfill]{parskip}
\usepackage[english]{babel}
\usepackage{mathtools, amsthm}
\usepackage{amsfonts}
\usepackage{amssymb}
\usepackage{mathrsfs}
\usepackage{verbatim}

\newtheorem{notes}{Notes}[section]
\newtheorem{prob}[notes]{Problems}
\newtheorem{thm}{Theorem}[section]
\newtheorem{cor}[thm]{Corollary}
\newtheorem{lem}[thm]{Lemma}
\newtheorem{defn}[notes]{Definition}
\newtheorem{rem}[notes]{Remark}
\newtheorem{prop}[thm]{Proposition}

\newcommand{\rl}{\mathbb{R}}
\newcommand{\id}{\text{id}}
\newcommand{\dprime}{{\prime\prime}}
\newcommand{\xprime}{X^\prime}
\newtcolorbox{mybox}[2][]{
    arc=0mm, enhanced, frame hidden, breakable
}
\newcommand{\qedbox}{$\hfill\blacksquare$}
\newcommand\nindent{.5pt}
\newcommand\noverline[1]{%
  \kern\nindent\overline{\kern-\nindent#1\kern-\nindent}\kern\nindent}

\setlength{\topmargin}{-.65in}
\setlength{\textwidth}{180mm} 
\setlength{\textheight}{240mm}
\setlength{\oddsidemargin}{-10mm}

\title{\textbf{Algebra I}\\
\large Homework 1
}
\author{Nutan Nepal}
\newcommand{\mR}{\mathbb{R}}
\newcommand{\ds}{\displaystyle}
\newcommand{\al}{\alpha}

\begin{document}
\maketitle
\makebox[\linewidth]{\rule{200mm}{1pt}}
\vspace{.1in}
\begin{enumerate}

\item[(1.3 - 13)] Show that an element has order 2 in 
    $S_n$ if and only if its cycle decomposition is a
    product of commuting 2-cycles.

\begin{mybox}

    $(\Longrightarrow)$
    Let $x\in S_n,\ n>1$ be an element that has order
    2. If $x(i)=j$ for some $i$, $j\in
    \{1,\ldots,n\}$. then since $x^2=1$ and we have $x(j)=i$.
    We see that $(i\ j)$ is a cycle in the cycle
    decomposition of the permutation $x$ and we can do
    the same for every other elements of $\{1,\ldots,n\}$.
    Then the cycle decomposition of $x$ is a product of
    disjoint 2-cycles. Since the disjoint cycles are also
    commuting, we have our proof.

    \vspace*{2mm}
    $(\Longleftarrow)$
    Let $x=\sigma_1\cdot\sigma_2\cdots\sigma_k \in S_n$
    where
    each $\sigma_i$ is a commuting 2-cycle. Then
    $$x^2=(\sigma_1\cdot\sigma_2\cdots\sigma_k)^2
    =\sigma_1^2\cdot\sigma_2^2\cdots\sigma_k^2
    =1\cdots 1=1$$
    Hence $x$ has order 2 in $S_n$.
\end{mybox}


\item[(1.4 - 11)] Let $H(F)=\left\{\left(\begin{array}{ccc}
    1 & a & b\\
    0 & 1 & c\\
    0 & 0 & 1 \end{array}
    \right):\ a,b,c\in F\right\}$ be called the Heisenberg
    group over $F$. Let $X=\left(\begin{array}{ccc}
        1 & a & b\\
        0 & 1 & c\\
        0 & 0 & 1 \end{array}
        \right)$
    and $Y=\left(\begin{array}{ccc}
        1 & d & e\\
        0 & 1 & f\\
        0 & 0 & 1 \end{array}
        \right)$ be elements of $H(F)$.
    \begin{enumerate}

        \item[(a)] Compute the matrix product $XY$ and
        deduce that $H(F)$ is closed under matrix
        multiplication. Exhibit explicit matrices such that
        $XY\neq YX$ (so that $H(F)$ is always non-abelian).
        \begin{mybox}
            
            $XY=\left(\begin{array}{ccc}
                1 & a & b\\
                0 & 1 & c\\
                0 & 0 & 1 \end{array}
                \right)\cdot
                \left(\begin{array}{ccc}
                    1 & d & e\\
                    0 & 1 & f\\
                    0 & 0 & 1 \end{array}
                \right)
                =\left(\begin{array}{ccc}
                    1 & a+d & e+af+b\\
                    0 & 1 & f+c\\
                    0 & 0 & 1 \end{array}
                \right)$. Hence $H(F)$ is closed
                under matrix multiplication.

                \vspace*{2mm}
                Let $X=\left(\begin{array}{ccc}
                    1 & 1 & 1\\
                    0 & 1 & 0\\
                    0 & 0 & 1 \end{array}
                    \right)$ and
                    $Y=\left(\begin{array}{ccc}
                        1 & 0 & 1\\
                        0 & 1 & 1\\
                        0 & 0 & 1 \end{array}
                    \right)$. Then
                    $XY=\left(\begin{array}{ccc}
                        1 & 1 & 3\\
                        0 & 1 & 2\\
                        0 & 0 & 1 \end{array}
                    \right)$ and
                    $YX=\left(\begin{array}{ccc}
                        1 & 1 & 2\\
                        0 & 1 & 1\\
                        0 & 0 & 1 \end{array}
                    \right)$. Hence we see that
                    $XY\neq YX$.
        \end{mybox}

        \item[(b)] Find an explicit formula for the matrix
        inverse $X^{-1}$ and deduce that $H(F)$ is closed
        under inverses.
        \begin{mybox}
            
            Let $Y$ be the inverse of $X$ with
            their respective entries from previous
            exercise. Then
            $$a+d=0,\ f+c=0, \ e+af+b=0$$
            Solving these equations gives us
            $$X^{-1}=\left(\begin{array}{ccc}
                1 & -a & ac-b\\
                0 & 1 & -c\\
                0 & 0 & 1 \end{array}
                \right)$$
            Since $X^{-1}$ is also an upper triangular
            matrix, $H(F)$ is closed under inverses.
        \end{mybox}

        \item[(c)] Prove the associative law for $H(F)$ and
        deduce that $H(F)$ is a group of order $|F|^3$.
        (Do not assume that matrix multiplication
        is associative.)
        \begin{mybox}
            
            Let $Z=\left(\begin{array}{ccc}
                1 & g & h\\
                0 & 1 & i\\
                0 & 0 & 1 \end{array}
                \right)$.
            Then
            \begin{align*}
                (XY)Z&=
                \left[\left(\begin{array}{ccc}
                    1 & a & b\\
                    0 & 1 & c\\
                    0 & 0 & 1 \end{array}
                \right)\cdot
                \left(\begin{array}{ccc}
                    1 & d & e\\
                    0 & 1 & f\\
                    0 & 0 & 1 \end{array}
                \right)\right]\cdot
                \left(\begin{array}{ccc}
                    1 & g & h\\
                    0 & 1 & i\\
                    0 & 0 & 1 \end{array}
                \right)\\
                &=\left(\begin{array}{ccc}
                    1 & a+d & e+af+b\\
                    0 & 1 & f+c\\
                    0 & 0 & 1 \end{array}
                \right)\cdot
                \left(\begin{array}{ccc}
                        1 & g & h\\
                        0 & 1 & i\\
                        0 & 0 & 1 \end{array}
                \right)\\
                &=\left(\begin{array}{ccc}
                    1 & a+d+g & h+ai+di+e+af+b\\
                    0 & 1 & c+f+i\\
                    0 & 0 & 1 \end{array}
                \right)
            \end{align*}
            and
            \begin{align*}
                X(YZ)&=
                \left(\begin{array}{ccc}
                    1 & a & b\\
                    0 & 1 & c\\
                    0 & 0 & 1 \end{array}
                \right)\cdot
                \left[\left(\begin{array}{ccc}
                    1 & d & e\\
                    0 & 1 & f\\
                    0 & 0 & 1 \end{array}
                \right)\cdot
                \left(\begin{array}{ccc}
                    1 & g & h\\
                    0 & 1 & i\\
                    0 & 0 & 1 \end{array}
                \right)\right]\\
                &=\left(\begin{array}{ccc}
                    1 & a & b\\
                    0 & 1 & c\\
                    0 & 0 & 1 \end{array}
                \right)\cdot
                \left(\begin{array}{ccc}
                        1 & d+g & h+di+e\\
                        0 & 1 & f+i\\
                        0 & 0 & 1 \end{array}
                \right)\\
                &=\left(\begin{array}{ccc}
                    1 & a+d+g & h+ai+di+e+af+b\\
                    0 & 1 & c+f+i\\
                    0 & 0 & 1 \end{array}
                \right).
            \end{align*}
            Hence, $H(F)$ is associative and is a
            subgroup of $GL_3(F)$. If the order of
            $F$ is finite. Then for $X\in H(F)$
            each $a,\ b,\ c$ has
            $|F|$ choices. So, $|H(F)|=|F|^3$.
        \end{mybox}

        \item[(d)] Find the order of each element of the
        finite group $H(\mathbb{Z}/2\mathbb{Z})$.
        \begin{mybox}
            
            There are $2^3=8$ elements in the group
            $H(\mathbb{Z}/2\mathbb{Z})$ which are given
            below with their orders:
            \begin{align*}
                e=&\left(\begin{array}{ccc}
                        1 & 0 & 0\\
                        0 & 1 & 0\\
                        0 & 0 & 1 \end{array}
                    \right),\ \ |e|=1\\
                x_1=&\left(\begin{array}{ccc}
                        1 & 0 & 0\\
                        0 & 1 & 1\\
                        0 & 0 & 1 \end{array}
                    \right),\ \ x_1^2=e\Longrightarrow |x_1|=2
            \end{align*}
            $$x_2=\left(\begin{array}{ccc}
                    1 & 1 & 0\\
                    0 & 1 & 0\\
                    0 & 0 & 1 \end{array}
                \right),\ \ x_2^2=e\Longrightarrow |x_2|=2$$
            
            $$x_3=\left(\begin{array}{ccc}
                    1 & 0 & 1\\
                    0 & 1 & 0\\
                    0 & 0 & 1 \end{array}
                \right),\ \ x_3^2=e\Longrightarrow |x_3|=2$$
            $$x_4=\left(\begin{array}{ccc}
                    1 & 1 & 1\\
                    0 & 1 & 0\\
                    0 & 0 & 1 \end{array}
                \right),\ \ x_4^2=e\Longrightarrow |x_4|=2$$
            
            $$x_5=\left(\begin{array}{ccc}
                    1 & 0 & 1\\
                    0 & 1 & 1\\
                    0 & 0 & 1 \end{array}
                \right),\ \ x_5^2=e\Longrightarrow |x_5|=2$$
            
            $$x_6=\left(\begin{array}{ccc}
                    1 & 1 & 0\\
                    0 & 1 & 1\\
                    0 & 0 & 1 \end{array}
                \right),\ \ x_6^4=e\Longrightarrow |x_6|=4$$
            $$x_7=\left(\begin{array}{ccc}
                    1 & 1 & 1\\
                    0 & 1 & 1\\
                    0 & 0 & 1 \end{array}
                \right),\ \ x_7^4=e\Longrightarrow |x_7|=4$$
        \end{mybox}

        \item[(e)] Prove that every nonidentity element of the
        group $H(\mathbb{R})$ has infinite order.
        \begin{mybox}
            
            First, we show that any n-$th$ power of an
            element in $H(\mathbb{R})$ is given by
            $$X^n=\left(\begin{array}{ccc}
                1 & a & b\\
                0 & 1 & c\\
                0 & 0 & 1 \end{array}
            \right)^n=\left(\begin{array}{ccc}
                1 & na & \frac{n(n-1)}{2}ac+nb\\
                0 & 1 & nc\\
                0 & 0 & 1 \end{array}
            \right)$$

            We can prove this by mathematical induction.
            The statement
            $$S(k):\ X^k=\left(\begin{array}{ccc}
                1 & ka & \frac{k(k-1)}{2}ac+kb\\
                0 & 1 & kc\\
                0 & 0 & 1 \end{array}
            \right)$$
            is trivial for the base case $k=1$. Let
            $S(k)$ be true for some positive integer
            $>1$. Then
                \begin{align*}
                S(k+1):\ X^{k+1}=&\left(\begin{array}{ccc}
                    1 & ka & \frac{k(k-1)}{2}ac+kb\\
                    0 & 1 & kc\\
                    0 & 0 & 1 \end{array}
                \right)\cdot
                \left(\begin{array}{ccc}
                    1 & a & b\\
                    0 & 1 & c\\
                    0 & 0 & 1 \end{array}
                \right)\\
                =&\left(\begin{array}{ccc}
                    1 & ka+a & b+kac+\frac{k(k-1)}{2}ac+kb\\
                    0 & 1 & c+kc\\
                    0 & 0 & 1 \end{array}
                \right)\\
                =&\left(\begin{array}{ccc}
                    1 & (k+1)a & \frac{k(k+1)}{2}ac+(k+1)b\\
                    0 & 1 & (k+1)c\\
                    0 & 0 & 1 \end{array}
                \right)
                \end{align*}
            Since $S(k)\implies S(k+1)$, the statement
            $S(k)$ is true for all positive integers.
            
            \vspace*{2mm}
            If $a,\ b,\ c \in \mathbb{R}$ are not all
            zero then $X^n$ cannot be the identity matrix
            for any $n$. Hence every nonidentity element
            of the group $H(\mathbb{R})$ has infinite order.
        \end{mybox}
    \end{enumerate}
  
\item[(2.1 - 12)] Let $A$ be an abelian group and fix some
    $n\in\mathbb{Z}$. Prove that the following sets are
    subgroups of $A$:
    \begin{enumerate}
        \item[(a)] $S_1=\left\{a^n : \ a\in A\right\}$
        \begin{mybox}
            
            We can prove that $S_1\neq \phi$ and if $x,y\in S_1$ then
            $xy^{-1}\in S_1$.

            \vspace*{2mm}
            Clearly, $S_1\neq \phi$ since $1^n=1\in S_1$.
            Let $x=a^n$ and $y=b^n$ are in $S_1$ for some
            $a$, $b \in A$. We have $y^{-1}=(b^n)^{-1}=(b^{-1})^n$.
            Then since $A$ is an abelian group,
            $xy^{-1}=a^n (b^{-1})^n=(ab^{-1})^n$. The last step here
            is justified by $A$ being an abelian group. Hence
            $xy^{-1}\in S_1$ and $S_1$ is a subgroup of $A$.
        \end{mybox}

        \item[(b)] $S_2=\left\{a\in A: \ a^n=1\right\}$
        \begin{mybox}
            
            Clearly $S_2\neq \phi$ since $1\in S_2$.
            If $x$, $y\in S_2$, then 
            \begin{align}
                (xy^{-1})^n=&x^n\cdot (y^{-1})^n\\
                =&x^n\cdot (y^n)^{-1}\\
                =& 1\cdot 1=1
            \end{align}

            Line 1 here is justified by the fact that $A$
            is an abelian group. Hence $xy^{-1}\in S_2$ and
            $S_2$ is subgroup.
        \end{mybox}
    \end{enumerate}

 
\item[(2.2 - 10)] Let $H$ be a subgroup of order 2 in $G$. Show that
    $N_G(H) = C_G(H)$. Deduce that if $N_G(H) = G$ then
    $H\leq Z(G)$.

\begin{mybox}

    Let $H=\{1,x\}$ is a subgroup of order 2 in $G$.
    Then for any $g\in N_G(H)$, $gH=Hg$ by definition.
    Since $gH=\{g,gx\}$ and $Hg=\{g,xg\}$, we must have
    $xg=gx$ for all $g\in N_G(H)$. So $N_G(H)\subset
    C_G(H)$. Now, suppose $g\in C_G(H)$, then $g1g^{-1}
    =1$ and $gx=xg\implies gxg^{-1}=x$. So $C_G(H)
    \subset N_G(H)$. Hence $N_G(H)=C_G(H)$.

    \vspace*{2mm}
    If $N_G(H)=G$ then $C_G(H)=G$ which means that
    that every elements of $G$ commutes with the
    elements of $H$. Hence $H\leq Z(G)$.
\end{mybox}


\item[(2.3 - 16)] Assume $|x| = n$ and $|y| = m$. Suppose
    that $x$ and $y$ commute: $xy = yx$. Prove that
    $|xy|$ divides the
    least common multiple of $m$ and $n$. Need this be true
    if $x$ and $y$ do not commute? Give an example of commuting
    elements $x$, $y$ such that the order of $xy$ is not
    equal to the least common multiple of $|x|$ and $|y|$.

\begin{mybox}
    
    Let $p$ be the least common multiple of $m$ and
    $n$. Then
    \begin{align*}
        (xy)^p  &=x^p\cdot y^p\ \ \text{(since $xy=yx$)}\\
                &=1\cdot 1\ \ \text{(since $m$, $n$ both divide $p$)}\\
                &=1
    \end{align*}
    So the order of $xy$ must divide the lease common
    multiple $p$.

    \vspace*{2mm}
    This need not be true if $x$ and $y$ do not commute.
    In $S_3$, we see that the order of $(1\ 2)$ and
    $(2\ 3)$ are 2. But $(1\ 2)(2\ 3)=(1\ 2\ 3)$ has order
    3 which is not the l.c.m of 2 and 2.

    \vspace*{2mm}
    In the abelian group
    $\mathbb{Z}_{12}$, the order of 2 is 6 and the order
    of 3 is 4. Here l.c.m of 4 and 6 is 12 but the order
    of the product $2\cdot 3=6$ is just 2.
\end{mybox}

    
\item[(2.3 - 23)] Show that $(\mathbb{Z}/2^n\mathbb{Z})^
{\times}$
    is not cyclic for any $n\geq 3$.
    [Find two distinct subgroups of order 2.]
  
\begin{mybox}
  
    If we show that there exists two distinct subgroups
    of $(\mathbb{Z}/2^n\mathbb{Z})^
    {\times}$ of order 2, then it is enough to prove
    that $(\mathbb{Z}/2^n\mathbb{Z})^
    {\times}$ is not cyclic.
    
    \vspace*{2mm}
    First, we note that in a group
    $(\mathbb{Z}/2^n\mathbb{Z})^{\times}$ where $n\geq 3$,
    $2^n-1$ and $2^{n-1}-1$ are distinct elements. Then
    $$\left(2^n-1\right)^2=2^{2n}-2\cdot2^n+1
    \equiv 1 \mod{2^n}$$
    and
    $$\left(2^{n-1}-1\right)^2=2^{2n-2}-2\cdot2^{n-1}+1
    \equiv 1 \mod{2^n}$$
    So we see that $2^n-1$ and $2^{n-1}-1$ both have
    order two in $(\mathbb{Z}/2^n\mathbb{Z})^{\times}$
    and $\{1,2^{n-1}-1\}$ and $\{1,2^n-1\}$
    are two distinct subgroups of
    $(\mathbb{Z}/2^n\mathbb{Z})^{\times}$. Hence the
    group $(\mathbb{Z}/2^n\mathbb{Z})^{\times}$ is
    not cyclic.
\end{mybox}


\item[(2.4 - 9)] Prove that $SL_2(\mathbb{F}_3)$ is the
subgroup of
$GL_2(\mathbb{F}_3)$ generated by $\left(\begin{array}{cc}
    1 & 1\\
    0 & 1
\end{array}\right)$ and $\left(\begin{array}{cc}
    1 & 0\\
    1 & 1
\end{array}\right)$. [Recall from Exercise 9 of Section 1
that $SL_2(\mathbb{F}_3)$ is the subgroup of matrices of
determinant 1. You may assume this subgroup has order 24
- this will be an exercise in Section 3.2.]

\begin{mybox}

    We need to show that the subgroup generated by
    $X=\left(\begin{array}{cc}
        1 & 1\\
        0 & 1
    \end{array}\right)$ and
    $Y=\left(\begin{array}{cc}
        1 & 0\\
        1 & 1
    \end{array}\right)$ is equal to the subgroup
    $SL_2(\mathbb{F}_3)$ of
    $GL_2(\mathbb{F}_3)$. Clearly,
    $X$, $Y\in SL_2(\mathbb{F}_3)$, so 
    $\langle X,Y\rangle \leq SL_2(\mathbb{F}_3)$.
    Since we can assume that the
    order of $SL_2(\mathbb{F}_3)$ is 24, we need to
    only show that $\langle X,Y \rangle$ has more than
    12 distinct elements as this would prove that the
    order of $\langle X,Y \rangle$ is 24 by Lagrange's
    theorem. We list 13 distinct elements of
    $\langle X,Y \rangle$ below:
    \begin{align*}
        &X=\left(\begin{array}{cc}
            1 & 1\\
            0 & 1
        \end{array}\right)
        &Y=\left(\begin{array}{cc}
            1 & 0\\
            1 & 1
        \end{array}\right)\hspace*{10mm}
        &X^2=\left(\begin{array}{cc}
            1 & 2\\
            0 & 1
        \end{array}\right)
        &Y^2=\left(\begin{array}{cc}
            1 & 0\\
            2 & 1
        \end{array}\right)\\
        &XY=\left(\begin{array}{cc}
            2 & 1\\
            1 & 1
        \end{array}\right)
        &YX=\left(\begin{array}{cc}
            1 & 1\\
            1 & 2
        \end{array}\right)\hspace*{10mm}
        &XYX=\left(\begin{array}{cc}
            2 & 0\\
            1 & 2
        \end{array}\right)
        &YXY=\left(\begin{array}{cc}
            2 & 1\\
            0 & 2
        \end{array}\right)\\
        &(XY)^2=\left(\begin{array}{cc}
            2 & 0\\
            0 & 2
        \end{array}\right)
        &(XY)^3=\left(\begin{array}{cc}
            1 & 2\\
            2 & 2
        \end{array}\right)\hspace*{10mm}
        &X^2Y^2=\left(\begin{array}{cc}
            2 & 2\\
            2 & 1
        \end{array}\right)
        &X^2Y=\left(\begin{array}{cc}
            0 & 2\\
            1 & 1
        \end{array}\right)\\
        &I=\left(\begin{array}{cc}
            1 & 0\\
            0 & 1
        \end{array}\right)
    \end{align*}
    Hence $\langle X,Y\rangle=SL_2(\mathbb{F}_3)$.
\end{mybox}
  

\item[(3.1 - 17)] Let $G$ be the dihedral group of order 16
    (whose lattice appears in Section 2.5):
    $$G=\langle r,s :\ r^8=s^2= 1, rs=sr^{-1}\rangle$$
    and let $\overline{G} = G\char`\\\langle r^4\rangle$ be
    the quotient
    of $G$ by the subgroup generated by $\langle r^4\rangle$
    (this subgroup is the center of $G$, hence is normal).
\begin{enumerate}
    
\item[(a)] Show that the order of $\overline{G}$ is 8.
\begin{mybox}
    
    Since $\langle r^4\rangle=\{1, r^4\}$,
    $|\langle r^4\rangle|=2$. Then by Lagrange's theorem,
    $$|\overline{G}|=\frac{|G|}{|\langle r^4\rangle|}=8.$$
\end{mybox}

\item[(b)] Exhibit each element of $\overline{G}$ in the form
$\overline{s}^a\overline{r}^b$, for some integers $a$ and $b$.
\begin{mybox}
    
    The elements of $\overline{G}$ are
    $\overline{1}$,  $\overline{r}$, $\overline{r}^2$,
    $\overline{r}^3$,  $\overline{s}$,
    $\noverline{s}\noverline{r}$,
    $\noverline{s}\noverline{r}^2$,
    $\noverline{s}\noverline{r}^3$.
\end{mybox}

\item[(c)] Find the order of each of the elements of
$\overline{G}$
exhibited in (b).
\begin{mybox}
    
    $|\overline{1}|=1$, $|\overline{r}|=4$,
    $|\overline{r}^2|=2$,
    $|\overline{r}^3|=4$,  $|\overline{s}|=2$,
    $|\noverline{s}\noverline{r}|=2$,
    $|\noverline{s}\noverline{r}^2|=2$,
    $|\noverline{s}\noverline{r}^3|=2$
\end{mybox}

\item[(d)] Write each of the following elements of
$\overline{G}$
in the form $\overline{s}^a\overline{r}^b$, for some
integers $a$ and
$b$ as in (b): $\overline{rs}, \overline{sr^{-2}s},
\overline{s^{-1}r^{-1}sr}$.
\begin{mybox}
    
    \begin{enumerate}
        \item[i.] $\overline{rs}=\overline{sr^{-1}}
            =\overline{sr^7}=\overline{sr^3}$
        \item[ii.] $\overline{sr^{-2}s} =
            \overline{ss(r^{-2})^{-1}} =
            \overline{s^2r^2} = \overline{r^2}$
        \item[iii.] $\overline{s^{-1}r^{-1}sr} =
            \overline{ssrr}=\overline{r^2}$.
    \end{enumerate}
\end{mybox}

\item[(e)] Prove that $\overline{H} = \langle\bar{s},
\overline{r}^2\rangle$
is a normal subgroup of $\overline{G}$ and $\overline{H}$ is
isomorphic to the Klein
4-group. Describe the isomorphism type of the complete preimage
of $\overline{H}$ in $G$.
\begin{mybox}

    Since $\overline{H}$ is generated by the elements of
    $\overline{G}$, it is a subgroup of $\overline{G}$.
    We show that, for any $g\in \overline{G}$,
    $g\overline{H}g^{-1}\subset H$ to prove that
    $\overline{H}$ is a normal subgroup of $\overline{G}$.
    It is enough to show that the conjugate
    of the generators $\{\overline{s},\overline{r^2}\}$
    belong to $\overline{H}$.
    If $g=\overline{r^k}\in \overline{G}$ for some integer
    $0\leq k\leq 3$ then

    
        $$g\overline{s}g^{-1}=\overline{r^ksr^{-k}}=
        \overline{r^{2k}s}=\overline{(r^2)^ks}\in
        \overline{H}$$
        and  $$g\overline{r^2}g^{-1}=\overline{r^kr^2r^{-k}}=
        \overline{r^{2}}\in
        \overline{H}.$$
        
        and if $g=\overline{sr^k}$ then

        $$g\overline{s}g^{-1}=\overline{sr^ks(sr^k)^{-1}}=
        \overline{s^2r^{-k}r^{-k}s^{-1}}=
        \overline{(r^2)^{-k}s}\in \overline{H}$$
        and
        $$g\overline{r^2}g^{-1}=
        \overline{sr^kr^2(sr^k)^{-1}}=
        \overline{sr^kr^2r^{-k}s^{-1}}=\overline{1}\in
        \overline{H}.$$
        Hence $\overline{H}$ is a normal subgroup.

        \vspace*{2mm}
    We see that all nonidentity elements of $\overline{H}
    =\{\overline{1}, \overline{r^2},
    \overline{s}, \overline{sr^2}\}$ have order 2.
    The Klein 4-group is given by
    $$V_4=\langle a,b:\ a^2=b^2=(ab)^2=1\rangle.$$
    We define a homeomorphism $\varphi:\overline{H}
    \to V_4$ by $\varphi(\overline{s})=a$ and
    $\varphi(\overline{r^2})=b$. Then we see that
    $\varphi$ is an isomorphism.

    \vspace*{3mm}
    The complete preimage of $\overline{H}$ are
    $$P= \{1, r^2, r^4, r^6, r^8, s, sr^2, sr^4, sr^8\}$$
    If we define a homeomorphism
    $\varphi:P\to D_8$ by
    $\varphi(r^2)=r$ and $\varphi(s)=s$, we see that it
    is bijective and hence an isomorphism.

\end{mybox}

\item[(f)] Find the center of $\overline{G}$ and describe
the isomorphism type
of $\overline{G}/Z(\overline{G})$.
\begin{mybox}
    
    The only element that commutes with $\overline{s}$
    is $\overline{r^2}$. We see that $\overline{r^2}$
    also commutes with all other elements of
    $\overline{G}$. So $Z(\overline{G})=\{ 1,
    \overline{r^2}\}$.

    \vspace*{3mm}
    The elements of $\overline{G}/Z(\overline{G})$ are
    $\{\overline{1}, \overline{\overline{r}},
    \overline{\overline{s}},\overline{\overline{sr}}\}$.
    We see that all the nonidentity elements have order
    2 and $|\overline{G}/Z(\overline{G})|=4$. Hence,
    $\overline{G}/Z(\overline{G})$ is isomorphic to
    the Klein 4-group.
\end{mybox}

\end{enumerate}


\item[(3.2 - 4)] Show that if $|G|=pq$ for some primes
$p$ and $q$ (not necessarily distinct) then either
$G$ is abelian or $Z(G) = 1$. [See Exercise 36 in Section 1.]
\begin{mybox}
    
    If $G$ is abelian, we are done. Suppose that $G$ is not
    abelian. Since $Z(G)$ is a subgroup and $G$ is not
    abelian, the order of $Z(G)$ must be either 1, $p$
    or $q$.
    
    \vspace*{2mm}
    We assume that $|Z(G|\neq 1$ and prove that this
    contradicts with our assumption. Suppose that the
    order of the center is $p$. Then since $Z(G)$ is
    normal, we take $G/Z(G)$ that has the order
    $|G|/|Z(G)|=pq/p=q$ which is prime. So, $G/Z(G)$ is
    cyclic.

    \vspace*{2mm}
    Let $gZ(G)$ be a generator of the group $G/Z(G)$.
    Then every coset of $Z(G)$ can be written in the form
    $g^kZ(G)$ for some integer $k$.
    We know that every element of $G$ belongs
    to some coset of $Z(G)$. Let $x$, $y\in G$ be written
    as $g^iz_1$ and $g^jz_2$ for
    $z_1$, $z_2\in Z(G)$. Then
    $xy=g^iz_1g^jz_2=g^jz_1g^iz_1=yx$. This shows that
    $G$ is an abelian group which contradicts our initial
    assumption.
    Hence $G$ is abelian or $Z(G)=1$.
\end{mybox}


\item[(3.2 - 16)] Use Lagrange's Theorem in the multiplicative group $(\mathbb{Z}
/p\mathbb{Z})^\times$ to prove Fermat's Little Theorem:
if $p$ is a prime then $a^p \equiv a\mod{p}$ for all $a\in\mathbb{Z}$.
\begin{mybox}
    
    We first note that
    $|(\mathbb{Z}/p\mathbb{Z})^\times|=p-1$ by
    Euler's totient function.
    Let $p\nmid a$ then
    $\overline{a}\in (\mathbb{Z}/p\mathbb{Z})^\times$ and
    by Lagrange's theorem, $|\overline{a}|^{p-1}=0$. So,
    $$a^{p-1}\equiv 0\mod{p}.$$
    Multiplying both sides by $a$ gives us the desired
    result.
    Now, if $p|a$ then $\overline{a}=0$ and
    $\overline{a}^p=0$. So again,
    $$a^p\equiv 0\equiv a \mod{p}.$$
\end{mybox}
\end{enumerate}
\end{document}