\documentclass[12pt]{article}
\usepackage[]{blindtext}
\usepackage[letterpaper, total{216mm, 279mm}]{}
\usepackage{amssymb,amsmath,amsfonts,verbatim}
\usepackage[breakable, skins]{tcolorbox}
\usepackage[parfill]{parskip}
\usepackage[english]{babel}
\usepackage{mathtools, amsthm}
\usepackage{amsfonts}
\usepackage{amssymb}
\usepackage{mathrsfs}
\usepackage{verbatim}

\newtheorem{notes}{Notes}[section]
\newtheorem{prob}[notes]{Problems}
\newtheorem{thm}{Theorem}[section]
\newtheorem{cor}[thm]{Corollary}
\newtheorem{lem}[thm]{Lemma}
\newtheorem{defn}[notes]{Definition}
\newtheorem{rem}[notes]{Remark}
\newtheorem{prop}[thm]{Proposition}

\newcommand{\rl}{\mathbb{R}}
\newcommand{\id}{\text{id}}
\newcommand{\dprime}{{\prime\prime}}
\newcommand{\xprime}{X^\prime}
\newtcolorbox{mybox}[2][]{
    arc=0mm, enhanced, frame hidden, breakable
}
\newcommand{\qedbox}{$\hfill\blacksquare$}

\setlength{\topmargin}{-.65in}
\setlength{\textwidth}{180mm} 
\setlength{\textheight}{240mm}
\setlength{\oddsidemargin}{-10mm}

\title{\textbf{Algebra I}\\
\large Homework 1
}
\author{Nutan Nepal}
\newcommand{\mR}{\mathbb{R}}
\newcommand{\ds}{\displaystyle}
\newcommand{\al}{\alpha}

\begin{document}
\maketitle
\makebox[\linewidth]{\rule{200mm}{1pt}}
\vspace{.1in}
\begin{enumerate}

\item[(1.3 - 13)] Show that an element has order 2 in 
    $S_n$ if and only if its cycle decomposition is a
    product of commuting 2-cycles.

\begin{mybox}

\end{mybox}


\item[(1.4 - 11)] Let $H(F)=\left\{\left(\begin{array}{ccc}
    1 & a & b\\
    0 & 1 & c\\
    0 & 0 & 1 \end{array}
    \right):\ a,b,c\in F\right\}$ be called the Heisenberg
    group over $F$. Let $X=\left(\begin{array}{ccc}
        1 & a & b\\
        0 & 1 & c\\
        0 & 0 & 1 \end{array}
        \right)$
    and $Y=\left(\begin{array}{ccc}
        1 & d & e\\
        0 & 1 & f\\
        0 & 0 & 1 \end{array}
        \right)$ be elements of $H(F)$.
    \begin{enumerate}

        \item[(a)] Compute the matrix product $XY$ and
        deduce that $H(F)$ is closed under matrix
        multiplication. Exhibit explicit matrices such that
        $XY\neq YX$ (so that $H(F)$ is always non-abelian).
        \begin{mybox}
            

        \end{mybox}

        \item[(b)] Find an explicit formula for the matrix
        inverse $X^{-1}$ and deduce that $H(F)$ is closed
        under inverses.
        \begin{mybox}
            
        \end{mybox}

        \item[(c)] Prove the associative law for $H(F)$ and
        deduce that $H(F)$ is a group of order $|F|^3$.
        (Do not assume that matrix multiplication
        is associative.)
        \begin{mybox}
            
        \end{mybox}

        \item[(d)] Find the order of each element of the
        finite group $H(\mathbb{Z}/2\mathbb{Z})$.
        \begin{mybox}
            
        \end{mybox}

        \item[(e)] Prove that every nonidentity element of the
        group $H(\mathbb{R})$ has infinite order.
        \begin{mybox}
            
        \end{mybox}
    \end{enumerate}
  
\item[(2.1 - 12)] Let $A$ be an abelian group and fix some
    $n\in\mathbb{Z}$. Prove that the following sets are
    subgroups of $A$:
    \begin{enumerate}
        \item[(a)] $\left\{a^n : \ a\in A\right\}$
        \begin{mybox}
            
        \end{mybox}

        \item[(b)] $\left\{a\in A: \ a^n=1\right\}$
        \begin{mybox}
            
        \end{mybox}
    \end{enumerate}

 
\item[(2.2 - 10)] Let $H$ be a subgroup of order 2 in $G$. Show that
    $N_G(H) = C_G(H)$. Deduce that if $N_G(H) = G$ then
    $H\leq Z(G)$.

\begin{mybox}

\end{mybox}


\item[(2.3 - 16)] Assume $|x| = n$ and $|y| = m$. Suppose
    that $x$ and $y$ commute: $xy = yx$. Prove that
    $|xy|$ divides the
    least common multiple of $m$ and $n$. Need this be true
    if $x$ and $y$ do not commute? Give an example of commuting
    elements $x$, $y$ such that the order of $xy$ is not
    equal to the least common multiple of $|x|$ and $|y|$.

\begin{mybox}

\end{mybox}

    
\item[(2.3 - 23)] Show that $(\mathbb{Z}/2^n\mathbb{Z})^
{\times}$
    is not cyclic for any $n\geq 3$.
    [Find two distinct subgroups of order 2.]
  
\begin{mybox}
  
\end{mybox}


\item[(2.4 - 9)] Prove that $SL_2(\mathbb{F}_3)$ is the
subgroup of
$GL_2(\mathbb{F}_3)$ generated by $\left(\begin{array}{cc}
    1 & 1\\
    0 & 1
\end{array}\right)$ and $\left(\begin{array}{cc}
    1 & 0\\
    1 & 1
\end{array}\right)$. [Recall from Exercise 9 of Section 1
that $SL_2(\mathbb{F}_3)$ is the subgroup of matrices of
determinant 1. You may assume this subgroup has order 24
- this will be an exercise in Section 3.2.]

\begin{mybox}
  
\end{mybox}
  

\item[(3.1 - 17)] Let $G$ be the dihedral group of order 16
    (whose lattice appears in Section 2.5):
    $$G=\langle r,s :\ r^8=s^2= 1, rs=sr^{-1}1\rangle$$
    and let $\overline{G} = G\char`\\\langle r^4\rangle$ be
    the quotient
    of $G$ by the subgroup generated by $\langle r^4\rangle$
    (this subgroup is the center of $G$, hence is normal).
\begin{enumerate}
    
\item[(a)] Show that the order of $\overline{G}$ is 8.
\begin{mybox}
    
\end{mybox}

\item[(b)] Exhibit each element of $\overline{G}$ in the form
$\overline{s}^a\overline{r}^b$, for some integers $a$ and $b$.
\begin{mybox}
    
\end{mybox}

\item[(c)] Find the order of each of the elements of
$\overline{G}$
exhibited in (b).
\begin{mybox}
    
\end{mybox}

\item[(d)] Write each of the following elements of
$\overline{G}$
in the form $\overline{s}^a\overline{r}^b$, for some
integers $a$ and
$b$ as in (b): $\overline{rs}, \overline{sr^{-2}s},
\overline{s^{-1}r^{-1}sr}$.
\begin{mybox}
    
\end{mybox}

\item[(e)] Prove that $\overline{H} = \langle\bar{s},
\overline{r}^2\rangle$
is a normal subgroup of $\overline{G}$ and $\overline{H}$ is
isomorphic to the Klein
4-group. Describe the isomorphism type of the complete preimage
of $\overline{H}$ in $G$.
\begin{mybox}
    
\end{mybox}

\item[(f)] Find the center of $\overline{G}$ and describe
the isomorphism type
of $\overline{G}\char`\\Z(\overline{G})$.
\begin{mybox}
    
\end{mybox}

\end{enumerate}


\item[(3.2 - 4)] Show that if $|G|=pq$ for some primes
$p$ and $q$ (not necessarily distinct) then either
$G$ is abelian or $Z(G) = 1$. [See Exercise 36 in Section 1.]
\begin{mybox}
    
\end{mybox}


\item[(3.2 - 16)] Use Lagrange's Theorem in the multiplicative group $(\mathbb{Z}
\char`\\p\mathbb{Z})^\times$ to prove Fermat's Little Theorem:
if $p$ is a prime then $a^p \equiv a\mod{p}$ for all $a\in\mathbb{Z}$.
\begin{mybox}
    
\end{mybox}
\end{enumerate}
\end{document}