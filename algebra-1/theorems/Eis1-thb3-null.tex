\begin{thm}[Nullstellensatz]
    \label{Nullstellensatz}
    Let $k$ be an algebraically closed field. If $I\subset k[x_1,\ldots,x_n]$ is an ideal, then
    $$I(Z(I))=\sqrt{I}.$$
    Thus the correspondences $I\mapsto Z(I)$ and $X\mapsto I(X)$ induces a bijection between the collection of algebraic
    subsets of $\mathbf{A}^n_k$ and radical ideals of $k[x_1,\ldots,x_n]$.
\end{thm}

\begin{cor}
    A system of polynomial equations $\{f_1=0,\ldots, f_m=0\}$ over an algebraically closed field $k$ has no solutions
    in $k^n$ iff $1$ can be expressed as a linear combination $1=\sum{p_if_i}$ with polynomial coefficients $p_i$.
\end{cor}

\begin{cor}
    If $k$ is an algebraically closed field and $A$ is a $k$-algebra, then $A=A(X)$ for some algebraic set $X$ iff $A$
    is reduced and finitely generated as $k$-algebra.
\end{cor}

\begin{cor}
    Let $k$ be an algebraically closed field and let $X\subset \mathbf{A}^n$ be an algebraic set. Every maximal ideal
    of $A(X)$ is of the form $\maxm_p:=(x_1-a_1,\ldots,x_n-a_n)/I(X)$ for some $p=(a_1,\ldots,a_n)\in X$.
\end{cor}

\begin{cor}
    The category of affine algebraic sets and morphisms (over an algebraically closed field $k$) is equivalent to the
    affine $k$-algebras with the arrows reversed.
\end{cor}