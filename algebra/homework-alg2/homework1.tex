\documentclass[12pt]{article}
\usepackage[]{blindtext}
\usepackage[letterpaper, total{216mm, 279mm}]{}
\usepackage{amssymb,amsmath,amsfonts,verbatim}
\usepackage[breakable, skins]{tcolorbox}
\usepackage[parfill]{parskip}
\usepackage[english]{babel}
\usepackage{mathtools, amsthm}
\usepackage{amsfonts}
\usepackage{amssymb}
\usepackage{mathrsfs}
\usepackage{verbatim}
\usepackage{tikz, graphicx}
\usepackage{verbatim}
\usepackage{tikz-cd}
\usepackage[]{geometry, titling}

\newtheorem{notes}{Notes}[section]
\newtheorem{prob}[notes]{Problems}
\newtheorem{thm}{Theorem}[section]
\newtheorem{cor}[thm]{Corollary}
\newtheorem{lem}[thm]{Lemma}
\newtheorem{defn}[notes]{Definition}
\newtheorem{rem}[notes]{Remark}
\newtheorem{prop}[thm]{Proposition}

\newcommand{\rl}{\mathbb{R}}
\newcommand{\bz}{\mathbb{Z}}
\newcommand{\id}{\text{id}}
\newtcolorbox{mybox}[1][]{
    arc=0mm, enhanced, frame hidden, breakable
}
\tikzcdset{
    bigcd/.style={
    arrows = {line width=0.8pt},
    cells = {sep=45pt,
    inner xsep=1ex, inner ysep=1ex},
    nodes = {font=\large},
    labels = {font=\normalsize}
    }
}
\NewEnviron{myequation}{%
    \begin{equation}
    \scalebox{3}{$\BODY$}
    \end{equation}
    }
\newcommand{\qedbox}{$\hfill\blacksquare$}
\newcommand\nindent{.5pt}
\newcommand\noverline[1]{%
  \kern\nindent\overline
  {\kern-\nindent#1\kern-\nindent}\kern\nindent}

%\setlength{\topmargin}{-.65in}
%\setlength{\textwidth}{190mm} 
%\setlength{\textheight}{240mm}
%\setlength{\oddsidemargin}{-13mm}
\geometry{top=20mm, left=15mm, right=15mm, bottom=20mm}
\pretitle{\begingroup\centering\huge}
\posttitle{\par\endgroup}
\title{Algebra II\\
\large Homework 1
}
\author{Nutan Nepal}
\newcommand{\mr}{\mathbb{R}}
\newcommand{\mz}{\mathbb{Z}}
\newcommand{\mq}{\mathbb{Q}}
\newcommand{\ds}{\displaystyle}
\newcommand{\al}{\alpha}
\newcommand{\nil}{\mathfrak{N}(R)}
\renewcommand{\hom}[1]{\text{Hom}_{#1}}
\newcommand{\tor}{\text{Tor}}
\newcommand{\im}{\text{im}}
%\newcommand{\ker}{\text{ker}}

\begin{document}
\maketitle
\makebox[\linewidth]{\rule{190mm}{.5pt}}
\vspace{0mm}
\begin{enumerate}

\item (10.1 - 5) For any left ideal $I$ of $R$ define
$$IM=\left\{\sum_{\text{finite}}
{a_im_i\ |\ a_i\in I,\ m_i\in M}\right\}$$
to be the collection of all finite sums of elements of
the form $am$ where $a\in I$ and $m\in M$. Prove that
$IM$ is a submodule of $M$.
\begin{mybox}
    We show that $IM$ is a submodule of $M$ and is closed
    under the action of the ring elements:

    \vspace*{2mm}
    Clearly, $0\in IM$.
    If $p=\alpha_1f_1+\cdots+\alpha_n f_n$ and
    $q=\beta_1 g_1+\cdots+\alpha_m g_m$ in $IM$ with
    $\alpha$, $\beta\in I$ and $f_i$, $g_i\in M$, then
    $p+q=\alpha_1f_1+\cdots+\alpha_n f_n+\cdots+
    \beta_1 g_1+\cdots+\alpha_m g_m$ is a finite sum with
    coefficients in the ideal and
    hence belongs to the set $IM$. For any finite sum
    $p=\alpha_1f_1+\cdots+\alpha_n f_n\in IM$, its
    inverse $-p$ is also a finite sum with coefficients
    in the ideal and hence belongs to $IM$.
    So, $IM$ is a subgroup. Now, for any ring element
    $r\in R$ and $p=\alpha_1f_1+\cdots+\alpha_n f_n \in
    IM$, we have $rp=r(\alpha_1f_1+\cdots+\alpha_n f_n)=
    r\alpha_1f_1+\cdots+r\alpha_n f_n$. Since $rI\subset
    I$, we see that $IM$ is closed under the ring action
    and hence is a submodule of $M$.
\end{mybox}

\item (10.1 - 6) Show that the intersection of any
nonempty collection of submodules of an $R$-module is
a submodule.

\begin{mybox}
Let $N=\bigcap_{\lambda\in \Lambda}{N_\lambda}$ be the
intersection of the nonempty collection of submodules
$N_\lambda$ of $M$ indexed by the set $\Lambda$. Clearly,
$0\in N$, so $N$ is nonempty. Now, for any $p$, $q\in N$,
$p+q\in N_\lambda$ for every $\lambda\in\Lambda$ since
each $N_\lambda$ is a submodule and thus $p+q\in N$. Also,
$p\in N\implies p\in N_\lambda\implies -p\in N_\lambda$
for all $\lambda$ which implies that $-p\in N$. Thus,
$N$ is a subgroup of $M$. Now, for $p\in N$ and $r\in R$,
we note that $rp\in N_\lambda$ for all $\lambda$. So
$rp\in N$ and $N$ is closed under the action of ring
elements. Thus $N$ is a submodule.
\end{mybox}


\item (10.2 - 6) Prove that $\hom{\mz}(\mz/n\mz, \mz/m\mz)
\simeq \mz/(n, m)\mz$.

\begin{mybox}
    We first note that if
    $\alpha\in\hom{\mz}(\mz/n\mz, \mz/m\mz)$ with $\alpha(1)
    =p$, then $0=\alpha(n)=pn$. Since $m\mid pn$ we have,
    $m/(m,n)\mid p$. Thus all elements $\alpha\in 
    \hom{\mz}(\mz/n\mz, \mz/m\mz)$ are given by:

    $$\alpha(1)=\frac{mx}{(m,n)}$$

    for $x\in \mz/n\mz$.
    
    Clearly, any map defined as above
    is a homomorphism in $\hom{\mz}(\mz/n\mz, \mz/m\mz)$.

    \vspace*{2mm}
    We define a map $\varphi:\mz/(n, m)\mz
    \longrightarrow \hom{\mz}(\mz/n\mz, \mz/m\mz)$
    defined by $\varphi(x)=\alpha_x$ where $\alpha_x(1)
    =mx/(m,n)$. Since these are all the unique
    homomorphisms in $\hom{\mz}(\mz/n\mz, \mz/m\mz)$,
    $\varphi$ is surjective. If $\varphi(x)=\varphi(y)$,
    then $mx/(m,n)=my/(m,n)\implies x=y$. Thus $\varphi$
    is injective. $\varphi$ is a homomorphism since
    $\varphi(x+y)(1)=m(x+y)/(m,n)=\varphi(x)(1)
    +\varphi(y)(1)$ and $\varphi(ry)(1)=mry/(m,n)
    =r\varphi(y)(1)$.
    Thus, this is an isomorphism of
    $\mz$-modules.
\end{mybox}

\item (10.2 - 10) Let $R$ be a commutative ring. Prove
that $\hom{R}(R, R)$ and $R$ are isomorphic as rings.
\begin{mybox}
    We first note that if $\varphi\in\hom{R}(R,R)$ then
    for any $x\in R$, $\varphi(x)=\varphi(x\cdot 1)=r\cdot
    \varphi(1)$. Since, $\varphi(1)$ can be any element of
    $R$ we have $\varphi(x)=x\cdot r$ for some $r\in R$.
    Hence $\varphi$ is completely determined
    by its value on the identity of $R$.

    \vspace*{2mm}
    Now, if $\alpha:R\to R$ is any map such that $\alpha
    (x)=x\cdot r$ for some $r\in R$ then we have
    $\alpha(p x+q y)=(px+qy)\cdot r=p\alpha(x)+q\alpha(y).$
    Thus, $\alpha\in\hom{R}(R,R)$ and we see that every
    $\varphi\in\hom{R}(R,R)$ is of this form. Hence we can
    define a map $\psi:\hom{R}(R,R)\to R$ by $\psi(\varphi)
    =\varphi(1)$ which is surjective as we saw above.
    Furthermore, if $\psi(\varphi)=\varphi(1)=0$ then
    $\varphi$ is a 0 map and hence $\psi$ is injective.
    We now show that $\psi$ is a ring homomorphism:

    \vspace*{2mm}
    \begin{enumerate}
        \item $\psi(\alpha+\beta)=(\alpha+\beta)(1)=
            \psi(\alpha)+\psi(\beta)$
        \item $\psi(\alpha\circ\beta)=(\alpha\circ\beta)(1)
            =\alpha(\beta(1))=\alpha(1\cdot\beta(1))=
            \alpha(1)\cdot\beta(1)=
            \psi(\alpha)\cdot\psi(\beta)$
        \item $\psi(e)=e(1)=1$
    \end{enumerate}
    where $e\in\hom{R}(R,R)$ is the identity map. Thus,
    $\psi$ is a ring isomorphism.
\end{mybox}

\item (10.3 - 4) An $R$-module $M$ is called a torsion
module if for each $m\in M$ there is a nonzero element
$r\in R$ such that $rm = 0$, where $r$ may depend on $m$
(i.e., $M = \tor(M)$ in the notation of Exercise 8 of
Section 1). Prove that every finite abelian group is a
torsion $\mz$-module. Give an example of an infinite
abelian group that is a torsion $\mz$-module.
\begin{mybox}

    If $p\neq 0$ is the order of the any given finite
    abelian group then $p\cdot m=0$ for every element
    $m\in M$. Thus every finite abelian group is a
    torsion $\mz$-module.
    
    \vspace*{2mm}
    We take $M=\mathbb{F}_2[x]$, the group of polynomials
    over the finite field $\mathbb{F}_2$ to be the our
    abelian group. Clearly, it's infinite as it has
    elements of every degree. Considered as a $\mz$-module,
    we see that $2\cdot f(x)=0$ for all $f(x)\in M$.
    Thus it is an infinite abelian torsion $\mz$-module.

\end{mybox}

\item (10.3 - 9) An $R$-module $M$ is called
\emph{irreducible} if $M\neq 0$ and if 0 and $M$ are the
only submodules of $M$. Show that $M$ is irreducible if
and only if $M\neq 0$ and $M$ is a cyclic module with any
nonzero element as generator. Determine all
the irreducible $\mz$-modules.
\begin{mybox}
    We first prove that if $M$ is irreducible then $M\neq
    0$ and $M$ is a cyclic module with any nonzero
    element as generator. Assume that $M\neq 0$ is
    irreducible and $m\in M$ is any element.
    Then since $Rm\subset M$ is a submodule of $M$ and $M$
    contains no non-trivial submodule, $Rm=M$ and $m$ is
    the nonzero generator. Now, assume that $M\neq 0$ is
    a cyclic module with any nonzero generator. Then if
    $m\neq 0$, $Rm\subset M$. But $M$ is generated by any
    nonzero element of $M$ so $M\subset Rm$. Thus $M=Rm$,
    meaning that $M$ and 0 are the only submodules of $M$.
    Hence $M$ is irreducible.

    \vspace*{2mm}
    From above, we see that the only irreducible
    $\mz$-modules are cyclic groups of prime order.
\end{mybox}

\item (10.4 - 2) Show that the element $``2\otimes 1"$ is
0 in $\mz\otimes_\mz \mz/2\mz$ but is nonzero in
$2\mz\otimes_\mz \mz/2\mz$.
\begin{mybox}

    We note that in $\mz\otimes_\mz \mz/2\mz$,
    $2\otimes 1=2\cdot1\otimes 1=1\otimes 2\cdot 1$.
    But since $2\equiv 0$ in $\mz/2\mz$, we have,
    $2\otimes 1=1\otimes 2=1\otimes 0=0$.

    To show that it is nonzero in $2\mz\otimes_\mz \mz/2\mz$,
    we consider the map $\alpha:2\mz\times \mz/2\mz
    \to \mz/2\mz$ defined by $\alpha(2m, n)=mn \mod 2$
    for $2m\in2\mz$ and $n\in \mz/2\mz$. Clearly, the
    following holds:

    \vspace*{2mm}
    \begin{enumerate}
        \item $\alpha(r_1\cdot 2m+r_2\cdot 2p,n)=
        r_1\cdot \alpha(2m,n)+r_2\cdot \alpha(2p,n)
        =r_1mn+r_2pn \mod 2.$
        \item $\alpha(2m, r_1\cdot n+r_2\cdot q)=
        r_1\cdot \alpha(2m,n)+r_2\cdot \alpha(2m,q)
        =r_1mn+r_2mq \mod 2.$
    \end{enumerate}
    for $m$, $p\in 2\mz$ and $n$, $q\in \mz/2\mz$. Then,
    $\alpha$ is a $\mz$-bilinear map and induces a
    $\mz$-linear map $\beta:2\mz\otimes_\mz \mz/2\mz
    \to \mz/2\mz$ such that $\beta(2m\otimes n)=
    mn\mod 2$.
    \[
    \begin{tikzcd}[bigcd]
        2\mz \times \mz/2\mz \ar[r,] \ar[dr, "\alpha"]
        & 2\mz\otimes_\mz \mz/2\mz \ar[d, "\beta"]\\
        & \mz/2\mz
        \end{tikzcd}
    \]
    Since $\beta(2\otimes 1)=1\cdot 1\mod 2=1\neq 0$ in
    $\mz/2\mz$, we see that $2\otimes 1\neq 0$ in
    $2\mz\otimes_\mz \mz/2\mz$ since $\beta$ is an $R$-linear
    map.
\end{mybox}

\item (10.4 - 4) Show that $\mq\otimes_\mz \mq$ and
$\mq\otimes_\mq \mq$ are isomorphic left $\mq$-modules.
[Show they are both 1-dimensional vector spaces over $\mq$.]
\begin{mybox}
    By Theorem 8, letting $\iota:\mq\to\mq\otimes_\mq \mq$
    be the ring homomorphism defined by $\iota(q)=1\otimes q$,
    we have the commutative diagram:
    \[
    \begin{tikzcd}[bigcd]
        \mq \ar[r,"\iota"] \ar[dr, "\varphi"]
        & \mq\otimes_\mq \mq \ar[d, "\varPhi"]\\
        & \mq
        \end{tikzcd}
    \]
    If we let $\varphi$ to be the identity map then we get
    $\varPhi\circ\iota$ to be the identity map. Then $\iota$ is
    an isomorphism of $\mq$-modules. To show that
    $\mq\otimes_\mz \mq\simeq\mq$ we define a map
    $\beta:\mq\to\mq\otimes_\mz \mq$ by
    $\beta(q)=1\otimes q$. We can prove that this is
    a $\mq$-linear map by
    $$\beta\left(\frac{a}{b}q\right)=1\otimes\frac{a}{b}q
    =a\otimes \frac{q}{b}=\frac{ab}{b}\otimes\frac{q}{b}
    =\frac{a}{b}\otimes q=\frac{a}{b}(1\otimes q)
    =\frac{a}{b}\beta(q).$$
    Now, we can define a map
    $\alpha:\mq\otimes_\mz \mq\to\mq$ by $\alpha(p\otimes
    q)=pq$. It is a well-defined map and we have
    $\alpha\circ\beta(q)=q$ and
    $$\beta\circ\alpha(p\otimes q)=\beta(pq)=1\otimes pq
    =p\otimes q.$$
    Thus, since the maps are inverse of each other, we see
    that $\beta$ is an isomorphism of $\mq$-modules. Since,
    both  $\mq\otimes_\mz \mq$ and
    $\mq\otimes_\mq \mq$ are isomorphic to $\mq$ as a
    bimodule, we see that they are isomorphic to each other
    as a left $\mq$-module.
\end{mybox}

\item (10.5 - 1) Suppose that
    \[
        \begin{tikzcd}[bigcd]
              A \ar[r, "\psi"] \ar[d, "\alpha"]
              & B \ar[d, "\beta"] \ar[r,"\varphi"]
              & C \ar[d, "\gamma"]\\
              A' \ar[r, "\psi'"]
              & B' \arrow[r,"\varphi'"]
              & C'
        \end{tikzcd}
    \]

is a commutative diagram of groups and that the rows are
exact. Prove that
\begin{enumerate}
    \item if $\varphi$ and $\alpha$ are surjective, and
    $\beta$ is injective then $\gamma$ is injective. [If
    $c\in \ker(\gamma)$, show there is a $b\in B$ with
    $\varphi(b) = c$. Show that $\varphi'(\beta(b)) = 0$
    and deduce that $\beta(b) = \psi'(a')$ for some $a'
    \in A'$. Show there is an $a\in A$ with $\alpha(a)=a'$
    and that $\beta(\psi(a))=\beta(b)$. Conclude that
    $b = \psi(a)$ and hence $c = \varphi(b) = 0.$]

    \begin{mybox}
        Let $c\in\ker(\gamma)$. Since $\varphi$ is surjective,
        there exists $b\in B$ with $\varphi(b)=c$. The
        given diagram is commutative and hence we have
        $\varphi'\circ \beta(b)=\gamma\circ\varphi(b)=0$.
        Hence $\beta(b)\in\ker(\varphi')=\im(\psi')$ and we
        have $\beta(b)=\psi'(a')$ for some $a'\in A'$. Since
        $\alpha$ is surjective, there exists $a\in A$ such that
        $\alpha(a)=a'$. Thus $\psi'(\alpha(a))=\beta(\psi(a))
        =\beta(b) \implies b=\psi(a).$ Hence $c=\varphi(b)
        =\varphi\circ\psi(a)=0$. So, $\gamma$ is injective.
    \end{mybox}

    \item if $\psi'$, $\al$, and $\gamma$ are
    injective, then $\beta$ is injective,

    \begin{mybox}
        Since $\psi'\circ\alpha$
        is injective, we see that $\beta\circ\psi$ is
        injective and so $\psi$ is injective.
        Let $b\in\ker(\beta)$. Then $\varphi'\circ\beta
        (b)=0=\gamma\circ\varphi(b)$. Since, $\gamma$ is
        injective and $\varphi(b)\in\ker(\gamma)$,
        $\varphi(b)=0\implies b\in \im(\psi)$. Thus there
        exists $a\in A$ with $b=\psi(a)$. But $\psi$ is
        injective and $\beta(b)=\beta\circ\psi(a)=0$.
        So $a=0$ and $b=\psi(a)=0$. $\beta$ is injective.

    \end{mybox}
    
    \item if $\varphi$, $\al$, and $\gamma$ are
    surjective, then $\beta$ is surjective,

    \begin{mybox}
        Let $b'\in B'$ be any element. We want to show that
        there exists a $p\in B$ with $\beta(p)=b'$.

        \vspace*{2mm}
        Since $\gamma\circ\varphi$
        is surjective, there exists a $b\in B$ such that
        $\gamma\circ\varphi(b)=\varphi'(b')=\varphi'\circ
        \beta(b)$. Then $\varphi'(\beta(b)-b')=0\implies
        b'-\beta(b)\in \im(\psi')$. So there exists $a'\in A'$
        with $\psi'(a')=\beta(b)-b'$. Since $\alpha$ is
        surjective, we have $a\in A$ such that $\psi'\circ
        \alpha(a)=\beta(b)-b'$. So, $\beta\circ\psi(a)
        =\psi'\circ\alpha(a)=\beta(b)-b'\implies\varphi'(\beta
        \circ\psi(a))=\varphi'(\beta(b)-b')$. If we let
        $p=\psi(a)+b$, 
        $$\beta(\psi(a)+b)=\beta\circ\psi(a)+\beta(b)
        =b'=\beta(b)+\beta(b)=b'.$$
        So, $\beta$ is surjective.
    \end{mybox}

    \item if $\beta$ is injective, $\al$ and $\varphi$ are
    surjective, then $\gamma$ is injective,

    \begin{mybox}
        Let $c\in\ker(\gamma)$. Then since $\varphi$ is
        surjective, there exists $b\in B$ with
        $c=\varphi(b)$. Then $\varphi'\circ\beta(b)
        =\gamma\circ\varphi(b)=0$. Then $b'=\beta(b)
        \in\ker(\varphi')=\im(\psi')$. So we have,
        $a'\in A$ with $\psi'(a)=b'$. Since $\alpha$ is
        surjective, we have $a\in A$ with $\alpha(a)=a'$
        and $\psi'\circ\alpha(a)=\beta\circ\psi(a)=b'
        =\beta(b)$. Since $\beta$ is injective, we have
        $\psi(a)=b\implies b\in\im(\psi)\implies
        c=\varphi(b)=0$. Thus $\gamma$ is injective.

    \end{mybox}

    \item if $\beta$ is surjective, $\gamma$ and $\psi'$
    are injective, then $\al$ is surjective.
    \begin{mybox}
        Let $a'\in A'$ be any element. We want to show that
        there exists a $a\in A$ with $\alpha(a)=a'$.

        \vspace*{2mm}
        Since $\beta$ is surjective, there exists $b\in B$
        with $\beta(b)=\psi'(a')=b'$ where $b'\in B'$ is
        some element of $B'$. Then $\gamma\circ\varphi(b)=
        \varphi'\circ\beta(b)=\varphi'\circ\psi'(a')=0$
        follows from the exactness of the lower row.
        Since $\gamma$ is injective, $\gamma\circ\varphi
        (b)=0\implies\varphi(b)=0\implies b\in\im(\psi)$.
        So there exists $a\in A$ such that $b=\psi(a)$.
        Then $\beta\circ\psi(a)=\psi'\circ\alpha(a)=
        \psi'(a')$. But since $\psi'$ is injective, we have
        $\alpha(a)=a'$ as required.
    \end{mybox}

\end{enumerate}

\item (10.5 - 4) Let $Q_1$ and $Q_2$ be $R$-modules.
Prove that $Q_1\oplus Q_2$ is an injective $R$-module
if and only if both $Q_1$ and $Q_2$ are injective.
\begin{mybox}
    If $Q_1\oplus Q_2$ is injective, then for any
    $R$-modules $L$, $M$, if $0\to L\xlongrightarrow{\psi}
    M$ is exact, then every $R$-module homomorphism $f$ from
    $L$ to $Q_1\oplus Q_2$ lifts to an $R$-module homomorphism
    $F$ of $M$ into $Q_1\oplus Q_2$. Now, if $g:L\to Q_1$
    is any $R$-module homomorphism, then $g$ factors through
    $Q_1\oplus Q_2$ with $g=\pi_1\circ f$ for some
    $f\in\hom{R}(L,Q_1\oplus Q_2)$. If $F$ is the lift of
    $f$, then $\pi_1\circ F\in \hom{R}(M,Q_1)$
    is the lift of $g$. Hence, $Q_1$ is injective. We see that
    $Q_2$ is also injective similarly.
    \[
    \begin{tikzcd}[bigcd]
        0 \ar[r]
        & L \ar[d, "f"] \ar[r, "\psi"]
        & M \ar[dl, dashed, "F"]
        \ar[ddll, dashed, out=260, in=10, "G"]
        \\
        & Q_1 \oplus Q_2 \ar[dl, "\pi_1"]
        & \\
        Q_1 \ar[from=uur, out=220, in=90 , "g"]
    \end{tikzcd}
    \]

    Now, if $Q_1$ and $Q_2$ are injective modules and $f:L\to
    Q_1\oplus Q_2$ is any module homomorphism, then
    $\pi_1\circ f$ and $\pi_2\circ f$ are homomorphisms from $L$
    to $Q_1$ and $Q_2$ respectively. If $F_1$ and $F_2$ are the lift
    of these homomorphisms to $\hom{R}(M,Q_1)$ and $\hom{R}(M,Q_2)$
    respectively, then we define $F:M\to Q_1\oplus Q_2$ by
    $F(m)=(F_1(m),F_2(m))$. Then this is a lift of the map $f$ and
    hence the direct sum is injective.
\end{mybox}
\end{enumerate}
\end{document}