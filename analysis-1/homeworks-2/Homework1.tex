\documentclass[12pt]{article}
\usepackage[]{blindtext}
\usepackage[letterpaper, total{216mm, 279mm}]{}
\usepackage{amssymb,amsmath,amsfonts,verbatim}
\usepackage[breakable, skins]{tcolorbox}
\usepackage[parfill]{parskip}
\usepackage[english]{babel}
\usepackage{mathtools, amsthm}
\usepackage{amsfonts}
\usepackage{amssymb}
\usepackage{mathrsfs}
\usepackage{verbatim}

\newtheorem{notes}{Notes}[section]
\newtheorem{prob}[notes]{Problems}
\newtheorem{thm}{Theorem}
\newtheorem{cor}[thm]{Corollary}
\newtheorem{lem}[thm]{Lemma}
\newtheorem{defn}[notes]{Definition}
\newtheorem{rem}[notes]{Remark}
\newtheorem{prop}[thm]{Proposition}

\newcommand{\rl}{\mathbb{R}}
\newcommand{\id}{\text{id}}
\newcommand{\dprime}{{\prime\prime}}
\newcommand{\xprime}{X^\prime}
\newtcolorbox{mybox}[2][]{
    arc=0mm, enhanced, frame hidden, breakable
}
\newcommand{\qedbox}{$\hfill\blacksquare$}
\setcounter{MaxMatrixCols}{10}

\setlength{\topmargin}{-.65in}
\setlength{\textwidth}{190mm} 
\setlength{\textheight}{240mm}
\setlength{\oddsidemargin}{-15mm} 
\setlength{\evensidemargin}{-15mm}
\parindent=0pt

\title{Analysis II \\
\large Homework 1
}
\author{Nutan Nepal}
\newcommand{\packpledge}{
    $\text{{\bf Pack Pledge:} I have neither given nor
    received unauthorized aid on this
    test or assignment.}$}

\begin{document}
\maketitle
\packpledge\\
\makebox[\linewidth]{\rule{200mm}{1pt}}
\vspace{1mm}


\newcommand{\mR}{\mathbb{R}}
\newcommand{\mN}{\mathbb{N}}
\newcommand{\mC}{\mathbb{C}}
\newcommand{\mQ}{\mathbb{Q}}
\newcommand{\cP}{\mathcal{P}}
\newcommand{\cB}{\mathcal{B}}
\newcommand{\cM}{\mathcal{M}}
\newcommand{\ds}{\displaystyle}
\newcommand{\al}{\alpha}
\newcommand{\li}{l^{\infty}}
\newcommand{\ep}{\varepsilon}
\newcommand{\de}{\delta}
\newcommand{\T}{\mathcal{T}}
\newcommand{\linf}{l^{\infty}}
\newcommand{\cD}{\mathcal{D}}
\newcommand{\cR}{\mathcal{R}}
\newcommand{\cN}{\mathcal{N}}
\newcommand{\lsn}{\limsup_{n \to \infty}}
\newcommand{\lin}{\liminf_{n \to \infty}}
\newcommand{\eq}{\Leftrightarrow}

\begin{enumerate}

\item \begin{enumerate}
\item Show that if $P$ is any partitions of $[a,b]$, and
$Q$ is a refinement of $P$ (i.e. $P \subseteq Q$), then 
$$L(f,P) \leq L(f,Q) \leq U(f,Q) \leq U(f,P).$$
\item Let $P$ and $Q$ be any partitions of $[a,b]$.
Show that $L(f, P) \leq U(f, Q)$.
\end{enumerate}

\begin{mybox}

\begin{enumerate}
  \item Let $P=\{x_0,x_1,\ldots,x_n\}$ and $Q=\{y_0,y_1,\ldots,
  y_m\}$ be the given partitions. Then, each interval $I_i=
  [x_i,x_{i+1}]$ of $P$ contains subintervals $J_1
  =[x_i=y_\alpha,y_{\alpha+1}],\ldots,
  J_k=[y_{\alpha+k-1},y_{\alpha+k}=x_{i+1}]$ of $Q$ and the
  length of the intervals $J_k$ add up to the length of $I$.
  Furthermore,
  $m_I=\inf_{x\in I}
  {f(x)}\leq m_{J_\beta}$ and $M_I=\sup_{x\in I}
  {f(x)}\geq M_{J_\beta}$ for each $\beta=1,\ldots,
  k$. Then for each $i=0,\ldots,n-1$,
  $$m_I\cdot(x_{i+1}-x_i)\leq \sum_{p=1}^k{m_{J_p}\cdot(x_{i+1}
  -x_i)}=\sum_{p=1}^k{m_{J_p}\cdot(y_{\alpha+p-1}
  -y_{\alpha+p})}.$$
  $$L(f,P)=\sum_{i=0}^n{m_I\cdot(x_{i+1}-x_i)}\leq L(f,Q).$$
  For upper sums, we obtain similar result with
  $U(f,Q) \leq U(f,P)$ and we have $L(f,P) \leq U(f,P)$ because
  $m_I\leq M_I$ for all intervals $I$. Combining the inequalities,
  we obtain the required inequality
  $$L(f,P) \leq L(f,Q) \leq U(f,Q) \leq U(f,P).$$

  \item Either $P$ is a refinement of $Q$ or $Q$ is a refinement
  of $P$. For both cases we see that $L(f, P) \leq U(f, Q)$
  from the result in (a).
\end{enumerate}
\end{mybox}

 \item Let $f: [a,b] \to \mR$ be continuous.
 Show that $f$ is Riemann integrable on $[a,b]$.

\begin{mybox}

Since $f$ is continuous on the closed interval, it is uniformly
continuous on $[a,b]$.
For any given $\varepsilon>0$ we have $\delta>0$ such that
$|x-y|<\delta$ implies $|f(x)-f(y)|<\varepsilon/(b-a)$
at all points $x,y\in [a,b]$. Then, let $P=\{x_0,x_1,
\ldots,x_n\}$ be a partition on $[a,b]$ such that $x_{i+1}
-x_i<\delta$. Then, if $m_i$ and $M_i$ denote the infinimum
and supremum of $f(x)$ on the partition $[x_i,x_i+1]$, we have
$M_i-m_i<\varepsilon/(b-a)$. Multiplying by $(x_{i+1}-x_i)$ and
summing over all $i$ we get
$$U(f,P)-L(f,P)<(\varepsilon/(b-a))\sum_{i=0}^{n-1}
{(x_{i+1}-x_i)}=\varepsilon.$$
Since, $\varepsilon$ was arbitrary, we see that $f$ is
Riemann integrable.

\end{mybox}
  
\item Show that if $f: [a,b] \to \mR$ is monotone
(without loss of generality, assume that f is increasing),
then $f$ is Riemann integrable on $[a,b]$.
\begin{mybox}

Let $P_n=\{x_0,x_1,\ldots,x_{n}\}$ be the partition of $[a,b]$
into $n$ equal intervals of length $(b-a)/n$. Let $m_i$ and
$M_i$ denote the infinimum and supremum of the function
$f$ on the interval $[x_i,x_{i+1}]$. Then since $f$ is monotonic,
$m_i=f(x_i)$ and $M_i=f(x_{i+1})$. So,
$$L(f,P_n)=\sum_{i=0}^{n-1}{f(x_i)(x_{i+1}-x_i)},
\hspace*{5mm}\text{and}\hspace*{5mm}
U(f,P_n)=\sum_{i=0}^{n-1}{f(x_{i+1})(x_{i+1}-x_i)}.$$
Since $x_{i+1}-x_i=(b-a)/n$, we have
$$U(f,P_n)-L(f,P_n)=\frac{b-a}{n}\sum_{i=0}^{n-1}{f(x_{i+1})
-f(x_i)}=\frac{b-a}{n}(f(b)-f(a)).$$
We see that the difference goes to 0 as $n\longrightarrow 0$.
Hence $f$ is Riemann integrable.

\end{mybox}
 
 
\item If $f$ and $g$ are Riemann-integrable on $[a,b]$ and
 $f(x) \leq g(x)$ for all $x \in [a,b]$, show that
 $$\int_a^b f(x) dx \leq \int_a^b g(x) dx.$$

\begin{mybox}

  Let $P=\{x_0,x_1,\ldots,x_n\}$ be a partition for $[a,b]$.
  We define a function $h(x)=g(x)-f(x)$ and see that
  $h(x)\geq 0$ for all $x$ in the interval $[a,b]$.
  Let $m_i$ and $M_i$ denote
  the infinimum and supremum of $h(x)$ on the interval
  $(x_i,x_{i+1})$. Then $0\leq m_i\leq M_i$ for all $i$ and
  since $g$ and $f$ are Riemann integrable, so is $h$. Hence
  $$\int_a^b h(x) dx = \int_a^b g(x) dx-
  \int_a^b f(x) dx \geq 0$$
  which gives the required result.
\end{mybox}

\item Let $f:[0,1]\to\mR$, \ $f(x) = \begin{cases}
  1, & x \in \mQ\\
  0, & x \notin \mQ
  \end{cases}
  $ (the Dirichlet function). 
  \begin{enumerate}
  \item Show that $f$ is discontinuous everywhere on $[0,1]$
  \item Show that $f$ is not Riemann integrable. 
  \end{enumerate}
\begin{mybox}

  \begin{enumerate}
    \item Let $x_0\in\mQ$. Then, for $\varepsilon=1$, we see
    that there exists no $\delta>0$ such that $f(x)\in (0,2)$
    for all $x\in (x_0-\delta,x_0+\delta)$ since every interval
    $(x_0-\delta,x_0+\delta)$ contains another irrational
    point. Hence, $f$ is not continuous on rational points.
    Similarly, $f$ is not continuous on irrational points in
    $[0,1]$ either.
    \item For any partition $P=\{x_0,x_1,\ldots,x_n\}$ of
    $[0,1]$, since every intervals contain rational as well as
    irrational points, $\inf{f(x)}=0$ and $\sup{f(x)}=1$ for
    each interval. Then $L(f,P)=0$ and $U(f,P)=1$ for every
    partition. For $\varepsilon=1$ we see that no such partition
    satifying $U(f,P)-L(f,P)$ exists. Hence, $f$ is not
    Riemann-integrable.
    \end{enumerate}
\end{mybox}

    
\item Let $f: [a,b] \to \mR$ be bounded and Riemann integrable.
Suppose that $F: [a,b] \to \mR$ is continuous and 
$$F'(x) = f(x),   \ \ \text{for all} \ x \in (a,b)$$
Show that 
$$F(b) - F(a) = \int_a^b f(x) dx $$

\begin{mybox}
    
Let $P=\{x_0,x_1,\ldots,x_n\}$ be a partition for $[a,b]$.
Then for any interval $I=[x_i,x_{i+1}]$, by mean value theorem,
we have $$F(x_{i+1})-F(x_i)=F'(x_{i_k})(x_{i+1}-x_i)=
f(x_{i_k})(x_{i+1}-x_i)$$
for some $x_{i_k}\in (x_i,x_{i+1})$. Let $m_i$ and $M_i$ denote
the infinimum and supremum of $f(x)$ on the interval
$(x_i,x_{i+1})$. Then $m_i\leq f(x_{i_k})\leq M_i$ and taking
sum over all $i$ from 0 to $n-1$, we obtain
$$L(f,P)\leq \sum_{i=0}^{n-1}{F(x_{i+1})-F(x_i)}\leq U(f,P).$$
However, since $P$ was an arbitrary partition and $f$ is
integrable, we have
$$\int_a^b f(x) dx =\sum_{i=0}^{n-1}{F(x_{i+1})-F(x_i)}
=F(b) - F(a).$$
\end{mybox}

\item Recall the Fundamental Theorem of Calculus:
If $f$ is continuous on $[a,b]$ and $F(x) = \int_a^x
f(t) dt$, then $F \in C^1[a,b]$ and $F'(x) = f(x)$.

\vspace*{1mm}
Now assume that $f$ is Riemann integrable on $[a,b]$ (and
implicitly on each subinterval of $[a,b]$) and $F(x) =
\int_a^x f(t) dt$. True or false: $F$ is differentiable.
Justify your answer. 
  
\begin{mybox}
    
  We define $f$ piecewise as follows:
  $f:[a,b]\to\mR$, \ $f(x) = \begin{cases}
    1, & x \in [a, a+(b-a)/2]\\
    0, & x \in (a+(b-a)/2,b]\mQ
  \end{cases}$
  $f$ is Riemann integrable and the function $F$ is increasing
  for the first half interval. However, it stays constant for
  the second half and thus it is not differentiable at
  $x=a+(b-a)/2$.
\end{mybox}
  
\item Show that a countable set of real numbers has
measure 0. 
\begin{mybox}
    
  Given a countable set $\{x_1,x_2,\ldots\}$
  and an arbitrary $\varepsilon>0$,
  we choose open covers $(x_i-\varepsilon/2^{i+2},
  x_i+\varepsilon/2^{i+2})$ for $x_i$. The open covers
  are each of length $\varepsilon/2^{i+1}$.
  We have, $\sum_{i=1}^\infty{\varepsilon/2^{i+1}}=\varepsilon/2
  <\varepsilon$. Since, the epsilon was arbitrary, we see that
  the countable set has measure 0.
\end{mybox}

\item Suppose that $f: X \to Y$ is a function and suppose
that $A_{\al}$, $A$, and $B$ are subsets of $Y$. Show that
\begin{enumerate}

\item $ f^{-1}\left(\bigcup_{\al} A_{\al}\right)
= \bigcup_{\al}
f^{-1}(A_{\al})$
\item $ f^{-1}\left(\bigcap_{\al} A_{\al}\right)
= \bigcap_{\al}
f^{-1}(A_{\al})$
\item $ f^{-1}(A^c) = (f^{-1}(A))^c$
\item $ f^{-1}(A \setminus B) = f^{-1}(A)
\setminus f^{-1}(B)$
\end{enumerate}

Which of these remain true when $f^{-1}$ is replaced by $f$
(and the sets are now subsets of $X$)?

\begin{mybox}

\begin{enumerate}
  \item Let $x\in f^{-1}\left(\bigcup_{\al} A_{\al}\right)$.
  Then, there is $y\in \bigcup_{\al} A_{\al}$ such that
  $f(x)=y$. So $y$ is in some $A_{\al}$. Thus $x\in
  \bigcup_{\al}f^{-1}(A_{\al})$. Now, let $x\in \bigcup_{\al}
  f^{-1}(A_{\al})$, then $x$ is in some $f^{-1}(A_{\al})$.
  So there exists some $y\in A_{\al}$ such that $f(x)=y$.
  Since $y\in \bigcup_{\al} A_{\al}$, we have
  $x\in f^{-1}\left(\bigcup_{\al} A_{\al}\right)$.

  \vspace*{2mm}
  \item Let $x\in f^{-1}\left(\bigcap_{\al} A_{\al}\right)$.
  Then, there is $y\in \bigcap_{\al} A_{\al}$ such that
  $f(x)=y$. So $y$ is in all $A_{\al}$ and $x$ belongs to all
  $f^{-1}(A_{\al})$. Thus $x\in
  \bigcap_{\al}f^{-1}(A_{\al})$. Now, let $x\in \bigcap_{\al}
  f^{-1}(A_{\al})$, then $x$ is in all $f^{-1}(A_{\al})$
  and there exists $y$ in every $A_{\al}$ such that $f(x)=y$.
  Since $y\in \bigcap_{\al} A_{\al}$, we have
  $x\in f^{-1}\left(\bigcap_{\al} A_{\al}\right)$.

  \vspace*{2mm}
  \item Let $x\in f^{-1}(A^c)$. Then there is some $y\notin A$
  such that $f(x)=y$. So $x\notin f^{-1}(A)$ and hence
  $x\in (f^{-1}(A))^c$. Now, let $x\in (f^{-1}(A))^c$. Then
  $f(x)\notin A$ and hence $f(x)\in A^c\implies x\in f^{-1}
  (A^c)$.

  \vspace*{2mm}
  \item Let $x\in f^{-1}(A\setminus B)$.
  Then there is some $y\in A$ and $y\notin B$
  such that $f(x)=y$. So $x\notin f^{-1}(B)$ and hence
  $x\in f^{-1}(A)\setminus f^{-1}(B)$.
  Now, let $x\in f^{-1}(A)\setminus f^{-1}(B)$. Then
  $f(x)\notin B$ and hence $f(x)\in A\setminus B
  \implies x\in f^{-1}
  (A\setminus B)$.
\end{enumerate}
\end{mybox}


\item
\begin{enumerate}
\item If $I$ and $J$ are open intervals in $\mR$ and
if $f: X \to \mR^2$ is any function with $f(x) = (u(x), v(x))$,
show that $$f^{-1}(I \times J) = u^{-1}(I) \cap v^{-1}(J).$$
\begin{mybox}

If $x\in f^{-1}(I\times J)$, then there exists $(y,z)
\in I\times J$ such that $y=u(x)$ and $z=v(x)$. Hence
$x\in u^{-1}(I)$ and $v^{-1}(J)$. Now, let $x\in
u^{-1}(I) \cap v^{-1}(J)$. Then $x\in u^{-1}(I)$ and there exists
$y\in I$ such that $y=f(x)$. Also $x\in v^{-1}(J)$ and so there
exists $z\in J$ such that $z=f(x)$. Thus $x$ is in the preimage
of $(y,z)$ and so in $f^{-1}(I\times J)$. Hence we have the
required result:
$$f^{-1}(I \times J) = u^{-1}(I) \cap v^{-1}(J).$$ 
\end{mybox}

\item Show that if $u: X \to \mR$ and  $v: X \to \mR$
are measurable functions, then so is $f = (u,v): X \to \mR^2$.
[You can assume that any open set in $\mR^2$ can be written
as countable union of open rectangles $R_k = I_k \times J_k$,
where $I_k$ and $J_k$ are open intervals in $\mR$.]
\end{enumerate} 
\begin{mybox}
  
Since any open set in $\rl^2$ can be witten as countable union
of open rectangles $R_k = I_k \times J_k$, we have, for
open set $U\subset \rl^2$,

$$U=\bigcup_{k=1}^\infty{I_k\times J_k}.$$
From (a), we see that
$f^{-1}(I \times J) = u^{-1}(I) \cap v^{-1}(J)$. Since $u$
and $v$ are measurable, $u^{-1}(I)$ and $v^{-1}(J)$ are
measurable sets and so is their intersection. Hence $f$ is
a measurable function.

\end{mybox}

\end{enumerate}

\end{document}
