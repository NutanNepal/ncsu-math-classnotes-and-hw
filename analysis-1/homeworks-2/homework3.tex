\documentclass[12pt]{article}
\usepackage[]{blindtext}
\usepackage[letterpaper, total{216mm, 279mm}]{}
\usepackage{amssymb,amsmath,amsfonts,verbatim}
\usepackage[breakable, skins]{tcolorbox}
\usepackage[parfill]{parskip}
\usepackage[english]{babel}
\usepackage{mathtools, amsthm}
\usepackage{amsfonts, xcolor}
\usepackage{amssymb}
\usepackage{mathrsfs}
\usepackage{verbatim}

\newtheorem{notes}{Notes}[section]
\newtheorem{prob}[notes]{Problems}
\newtheorem{thm}{Theorem}
\newtheorem{cor}[thm]{Corollary}
\newtheorem{lem}[thm]{Lemma}
\newtheorem{defn}[notes]{Definition}
\newtheorem{rem}[notes]{Remark}
\newtheorem{prop}[thm]{Proposition}

\newcommand{\rl}{\mathbb{R}}
\newcommand{\id}{\text{id}}
\newcommand{\dprime}{{\prime\prime}}
\newcommand{\xprime}{X^\prime}
\newtcolorbox{mybox}[0]{
    arc=0mm, enhanced, frame hidden, breakable,
}
\newcommand{\qedbox}{$\hfill\blacksquare$}
\setcounter{MaxMatrixCols}{10}

\setlength{\topmargin}{-.65in}
\setlength{\textwidth}{190mm} 
\setlength{\textheight}{240mm}
\setlength{\oddsidemargin}{-15mm} 
\setlength{\evensidemargin}{-15mm}
\parindent=0pt

\title{Analysis II \\
\large Homework 3
}
\author{Nutan Nepal}
\newcommand{\packpledge}{
    $\text{{\bf Pack Pledge:} I have neither given nor
    received unauthorized aid on this
    test or assignment.}$}

\begin{document}
\maketitle
\packpledge\\
\makebox[\linewidth]{\rule{200mm}{1pt}}
\vspace{1mm}


\newcommand{\mR}{\mathbb{R}}
\newcommand{\mN}{\mathbb{N}}
\newcommand{\mC}{\mathbb{C}}
\newcommand{\mQ}{\mathbb{Q}}
\newcommand{\mq}{\mathbb{Q}}
\newcommand{\cP}{\mathcal{P}}
\newcommand{\cB}{\mathcal{B}}
\newcommand{\cM}{\mathcal{M}}
\newcommand{\ds}{\displaystyle}
\newcommand{\al}{\alpha}
\newcommand{\li}{l^{\infty}}
\newcommand{\ep}{\varepsilon}
\newcommand{\de}{\delta}
\newcommand{\T}{\mathcal{T}}
\newcommand{\linf}{l^{\infty}}
\newcommand{\cD}{\mathcal{D}}
\newcommand{\cR}{\mathcal{R}}
\newcommand{\cN}{\mathcal{N}}
\newcommand{\lsn}{\limsup_{n \to \infty}}
\newcommand{\lin}{\liminf_{n \to \infty}}
\newcommand{\eq}{\Leftrightarrow}
\newcommand{\dmu}{\ d\mu}
\newcommand{\ix}{\int_X}
\newcommand{\cL}{\mathcal{L}(\mathbb{R})}
\newcommand{\soi}{\sum_{i=1}^{\infty}}
\newcommand{\son}{\sum_{i=1}^{n}}

\begin{enumerate}

    \item Let $(X, \cM)$ be a measurable space.
    \begin{enumerate}
    \item Let $f: X \to \mR$, measurable and bounded.
    Show that for each $\ep > 0$, there are simple functions
    $\varphi_{\ep}$ and $\psi_{\ep}$ on $X$ such that
    $$\varphi_{\ep} \leq f \leq \psi_{\ep} \ \ \text{and} \ \ 0 \leq \psi_{\ep} -  \varphi_{\ep}  < \ep \ \text{on} \ X.$$
    
    \item Show that $f:X\to [-\infty, \infty]$ is
    measurable if and only if there exists a sequence
    $\{s_n\}$ of simple, measurable functions
    on $X$ such that $s_n \to f$ pointwise as
    $n \to \infty$, and $|s_n| \leq |f|$ on $X$, for
    all $n$.
    
    \end{enumerate}
    
    \begin{mybox}
        \begin{enumerate}
            \item Since $f$ is bounded, we note that there exists an open
            interval $[c,d]$ such that $f(X)\subset [c,d]$. For every
            $\varepsilon>0$, we can take the partition of the interval
            $[c,d]$ such that $y_{k}-y_{k-1}<\varepsilon$ and
            $c=y_0<y_1<\cdots<y_{n-1}<y_n=d$ for some integer $n$. For
            each $I_k=[y_{k-1},y_k)$, we define $E_k$ to be
            $f^{-1}(I_k)$ which is measurable since $f$ is measurable.

            \vspace*{2mm}
            Now we define the functions $\varphi_\varepsilon$ and
            $\psi_\varepsilon$ by
            $$\varphi_\varepsilon(x)=\sum_{k=1}^{n}{y_{k-1}\cdot
            \chi_{E_k}}\hspace*{5mm}\text{and}\hspace*{5mm}
            \psi_\varepsilon(x)=\sum_{k=1}^{n}{y_k\cdot\chi_{E_k}}.$$
            For $x\in E$, there is a unique $k\in \overline{1,\ldots,n}$
            such that $x\in E_k$ and we have $y_{k-1}\leq f(x)<y_k$.
            But $\varphi_\varepsilon(x)=y_{k-1}$ and $\psi_\varepsilon(x)
            =y_k$ and hence $\varphi_\varepsilon(x)\leq f(x)<\psi
            _\varepsilon(x)$ with $\psi_\varepsilon-\varphi_\varepsilon
            <\varepsilon$ on $X$.

            \vspace*{3mm}
            \item For a measurable function $f$, we take the sequence of
            simple functions defined on \textbf{Homework 2 - Problem 11}
            which we proved to be monotone increasing and pointwise
            convergent to $f$.

            \vspace*{2mm}
            Now assume that we have a sequence $\{s_n\}$ of simple,
            measurable functions that converge to $f$ pointwise
            and $|s_n|\leq|f|$ on $X$ for all $n$. We first note that
            for any number $c\in[-\infty,\infty]$,
            since $\lim_{n\to\infty}{s_n(x)}=f(x)$ for each $x$, we have
            $f(x)<c$ if and only if there exist $n$, $k\in\mN$ with
            $f_j(x)<c-1/n$ for all $j\geq k$. Since each set $E_{j,n}
            =\{x\in E:\ f_j(x)<c-1/n\}$ is measurable (because $f_j$ is
            measurable), for each $k$, we have $\bigcap_{j=k}^\infty
            {E_{j,n}}$ is measurable. Hence
            $$\{x\in E:\ f(x)<c\}=\bigcup_{1\leq k,n<\infty}\left(
            \bigcap_{j=k}^\infty{E_{j,n}}\right)$$
            which, in turn, implies that $f$ is a measurable
            function.
        \end{enumerate}
    \end{mybox}
    \item Let $X = \mN$, $\cM = \cP(\mN)$, and $\mu$ is the counting measure. Show that for every function $f: \mN \to [0, \infty]$, 
    $$ \int_X f \ d\mu = \sum_{n=1}^{\infty} f(n).$$
    
    \begin{mybox}
        We take the sets $E_k=\{1,2,\ldots,k\}$ and define the sequence
        $\{f_k\}$ by $f_k=f\cdot\chi_{E_k}$. Then $\{f_k\}$ is a monotone
        increasing sequence of simple functions
        that converges pointwise to $f$ and
        $$\int_X{f_k\dmu}=\int_{E_k}{f_k\dmu}=\sum_{n=1}^{k}{f(n)}.$$
        By monotone convergence theorem we have the required result as
        $k\longrightarrow \infty$.
    \end{mybox}

    \item Let $(X, \cM, \mu)$ be a measure space. Let $f: X  \to [0, \infty]$ be measurable. Show that $\varphi(A) = \int_A f \ d\mu$ is a positive measure on $\cM$.
    
    \begin{mybox}
        \begin{enumerate}
            \item For any set $A\in\cM$, we see that $\varphi(A)=\int_A
            {f\ \dmu}\geq 0$ since $f$ is a non-negative function
            ($\varphi(\emptyset)=0$).

            \item For any countable collection of measurable pairwise
            disjoint sets $\{A_n\}$, we have
            $$\varphi\left(\bigcup_{n=1}^\infty{A_n}\right)=
            \int_{\bigcup_{n=1}^\infty{A_n}}{f\ \dmu}=\sum_{n=1}^\infty
            {\left(\int_{A_n}{f\ \dmu}\right)}=\sum_{n=1}^\infty{\varphi(A_n)}$$
            where the second equality follows from the additivity property
            of integrals over the domain of integration.
        \end{enumerate}
        Thus $\varphi$ is a positive measure on $\cM$.
    \end{mybox}
    
    \item Let $f_n: X \to [0, \infty]$ be a monotone decreasing sequence of  functions with $f_n \searrow f$ pointwise. 
    \begin{enumerate}
    \item Show by counterexample that $\ds \lim_{n \to \infty}\int_X f_n\ d \mu$ is not necessarily $\int_X f \ d\mu$. 
    \item Find an additional assumption that would make the statement true. 
    \end{enumerate}
    
    \begin{mybox}
        \begin{enumerate}
            \item We define a sequence $\{f_n\}$ of functions
            $f_n:X=[0,\infty]\to[0,\infty]$ as follows:
            $$f_n(x)=
            \begin{dcases}
                n &x\in [n,\infty]\\
                0 &x\in[0,n).
            \end{dcases}$$
            We see that $\{f_n\}$ is clearly a decreasing sequence of
            functions and $\int_{X}{f_n\ \dmu}=\infty$ for all $n$, but
            $\int_X{\lim_{n\to\infty}{f_n}\ \dmu}=\int_X{0\ \dmu}=0$.
            
            \vspace*{2mm}
            \item If $f_1\in L^1(\mu)$ then the statement is true. We first
            note that if $f_1\geq f_2\geq\cdots \geq f\geq 0$, then
            $-f_1\leq-f_2\leq\cdots \leq -f\leq 0$ is a monotone increasing
            sequence. We define $g_n=f_1-f_n$ and take the sequence
            $\{g_n\}$ which is monotone increasing and converges to $f_1-f$. $f$ is
            measurable since it is a pointwise limit of measurable
            functions and by monotone convergence theorem for increasing
            sequence we have
            $$\lim_{n\to\infty}{\int_X{f_1-f_n\ \dmu}}=\int_X{f_1-f\ \dmu}
            \implies
            \lim_{n\to\infty}{\int_X{f_n\ \dmu}}=\int_X{f\ \dmu}.$$
        \end{enumerate}
    \end{mybox}
    
    \item Let $(X, \cM, \mu)$ be a measure space with $\mu(X) =1$. Suppose $E_1, E_2, ..., E_n$ are a finite number of measurable sets in $X$, such that each point in $X$ belongs to at least $M$ of these sets (where $M$ is a positive integer with $M \leq n$). Show that there exists $k$ such that $\mu(E_k) \geq \frac{M}{n}$.
    
    \begin{mybox}
        Define a function $g:X\to\mN$ by $g(x)=\sum_{k=1}^{n}{\chi_{E_k}}$.
        Then $g(x)\geq M$ for all $x\in X$. We have
        $$M=M\cdot\mu(X)\leq \int_{X}{g(x)\dmu}\leq \sum_{k=1}^{n}
        {\int_{E_k}{g(x)\dmu}}.$$
        For each $E_k$, $\int_{E_k}{g(x)\dmu}$ is at most $\mu(E_k)$. Thus,
        $M\leq n\cdot\mu(E_k)$ for some $k$ and hence we have the required
        result.
    \end{mybox}

    \item Prove an analogous result to Fatou's Lemma for $\ds \lsn$. 
    
    \begin{mybox}
        \begin{thm}[Fatou's Lemma, reverse]
            Let $\{f_n\}$ be a sequence of non-negative 
            bounded measurable functions
            on $X$ and $f_n \to f$ pointwise, then
            $$\lsn{\int_X{f_n\ \dmu}}\leq\int_X{f\ \dmu}.$$
        \end{thm}
        \begin{proof}
            Since $\{f_n\}$ is a bounded
            sequence and hence there exists a function $g$ such that 
            $f_n(x)\leq g(x)$ for all $n$. Then $\{g-f_n\}$ is a sequence of
            non-negative functions that converge to $g-f$. By Fatou's Lemma,
            we have
            $$\int_X{g-f\ \dmu}\leq \lin\int_X{g-f_n\ \dmu}.$$
            Using linearity and multiplying by -1, we have
            $$\int_X{f\ \dmu}\geq -\lin\int_X{-f_n\ \dmu}=\lsn\int_X{f_n\ \dmu}$$
            as required.
        \end{proof}
    \end{mybox}

    \item Give an example where we have strict inequality in Fatou's Lemma. Then illustrate by example that the assumption ``$f_n$ are non-negative" is necessary in Fatou's Lemma. 
    
    \begin{mybox}
        We define a sequence of functions $\{f_n\}$ by
        $$f_n(x)=
        \begin{dcases}
            1/n &x\in [0,n]\\
            0 &\text{otherwise.}
        \end{dcases}$$
        The sequence $\{f_n\}$ converges to the 0 function which has integral
        0. However, each function $f_n$ has integral 1. Thus we have the
        strict inequality.

        \vspace*{2mm}
        To see the importance of the non-negativity, we define a similar
        functions as above by
        $$f_n(x)=
        \begin{dcases}
            -1/n &x\in [0,n]\\
            0 &\text{otherwise.}
        \end{dcases}$$
        Each functions $f_n$ has integral -1 but the limit is the 0 function
        which has integral 0 which is more than the lim inf of the integral.
    \end{mybox}
    
    
    \item Suppose that $\mu(X)  < \infty$, and $\{f_n\}$ is a sequence of bounded complex measurable functions on $X$, and $f_n \to f$ uniformly on $X$. Prove that 
    $$\lim_{n \to \infty} \int_X f_n \ d\mu = \int_X f \ d\mu.$$
    
    \begin{mybox}
        We know that each $f_n$ is bounded since $f_n$ converges uniformly to
        $f$ and thus $f$ is also bounded. Since each $f_n$ are measurable, $f$
        is also measurable. Now, for each $\varepsilon>0$, we know that
        there exists $N\in\mN$ such that $|f-f_n|<\varepsilon/\mu(X)$
        for all $n>N$. Then
        $$\left|\int_{E}{f}-\int_E{f_n}\right|=\left|\int_E{f-f_n}\right|
        \leq\int_E{|f-f_n|}\leq \frac{\varepsilon}{\mu(X)}\cdot\mu(X)=
        \varepsilon.$$
        Thus, $\lim_{n\to\infty}\int_E{f_n}=\int_E{f}$.
    \end{mybox}
    
    
    \item Assume that $f \in L^1(\mu)$ and 
    $\ds \left| \ix f \dmu \right| = \ix |f| \dmu$. 
    Then there exists $\alpha \in \mC$, with $|\alpha| = 1$ such that $\alpha f = |f|$ a.e. 
    on X. 
    \begin{mybox}
        We define $\ds\beta=\left.\left(\int_X{f\ \dmu}\right)\middle/
        \left|\int_X{f\ \dmu}\right|\right.$. Clearly, $|\beta|=1$ and we
        have $\ds\int_X{f\ \dmu}=\beta\left|\int_X{f\ \dmu}\right|=
        \beta\int_X{|f|\ \dmu}$. Since the integrals are equal, we conclude
        that $f=\beta|f|$ a.e. on $X$. Taking $\alpha=1/\beta$ gives the
        required equality.
    \end{mybox}

    \item Let $(X, \cM, \mu)$ be a measure space and suppose $f$ is a non-negative measurable function on $X$. If $\int_X f\ d\mu = 0$, show that $\mu(\{x \in X \ | \ f(x) \neq 0 \}) = 0$ (i.e. $f=0$ a.e. on $X$).
    
    \begin{mybox}
        Assume that $\mu(\{x \in X \ | \ f(x) \neq 0 \}) > 0$ instead.
        We take the sets $A_n = \{x \ | \ f(x) > 1/n\}$ and note that
        $\mu(A_k)>0$ for some $k$. Then,
        $$\int_X{f\ \dmu}\geq \int_{A_k}{f\ \dmu}>\int_{A_k}{\frac{1}{k}\ \dmu}
        =\frac{1}{k}\cdot\mu(A_k)>0$$
        which is a contradiction. Thus $\mu(\{x \in X \ | \ f(x) \neq 0 \}) = 0$.
    \end{mybox}
    
    \item (A small extension of the LDCT) 
    
    Let $\{f_n\}$ be a sequence of either complex-valued or extended real-valued functions such that $f_n(x) \to f(x)$ a.e. on $X$ and suppose there is $g \in L^1(\mu)$ such that $|f_n(x)| \leq g(x)$ for a.e. $X$. Show that
    $$\lim_{n \to \infty} \int_X f_n(x)\ \dmu = \int_X f(x)\ \dmu$$ 
    
    \begin{mybox}
        Let $N$ be the set where $|f_n(x)|>|g(x)|$ or where $f_n$ doesn't converge
        to $f$. This set is a union of two sets that have measure 0 and hence 
        itself has measure 0. Then by Lebesgue Dominated Convergence theorem,
        $$\lim_{n \to \infty} \int_{X-N} f_n(x) \dmu = \int_{X-N} f(x)\ \dmu.$$
        
        But, for each $n$, we have $\ds\int_{X}{f_n(x)\ \dmu}=\int_{X-N}
        {f_n(x)\ \dmu}+\int_{N}{f_n(x)\ \dmu}$. We note that
        $\ds\int_{N}{f_n(x)\ \dmu}\leq\int_{N}{g(x)\ \dmu}\leq\sup_{x\in N}
        {g(x)}\cdot\mu(N)=0$. Using this result for each $n$ and also for
        $f$ itself, we get the required result.
    \end{mybox}
    
    \item (Absolute Continuity of the Integral) Let $f$ be a  non-negative measurable function in $L^1(\mu)$. Show that for each $\ep > 0$ there exists a $\delta>0$ such that for every measurable set $A$ with $\mu(A) < \delta$, we have $\int_A f\ \dmu  < \ep$. 
    
    {\bf Hint:} Argue by contradiction: If not, then there is some $\ep_0$ and a sequence of measurable sets $A_n$ with $\mu(A_n) < 2^{-n}$ and 
    $$ \int_{A_n} f \dmu > \ep_0.$$ 
    Consider the sequence $g_n = f \cdot \chi_{A_n}$. Show that $g_n$ converges to 0 except at points $x$ in infinitely many of the sets $A_n$. What is the measure of this ``exceptional" set? Now apply a convergence result to the sequence $f_n = f - g_n$ to get a contradiction. 
    
    \begin{mybox}
        We prove the given statement by contradiction. If the statement is
        false then there is some
        $\varepsilon_0>0$ and a sequence of measurable sets $A_n$ with
        $\mu(A_n)<2^{-n}$ and $\int_{A_n}{f \dmu}>\varepsilon_0$. We define a
        sequence $\{g_n\}$ of functions by $g_n=f\cdot\chi_{A_n}$. For
        $x\notin A_n$, we have $g_n(x)=0$ and for $x\in A_n$,
        $g_n(x)=f$ with $\mu(A_n)=2^{-n}$. Then $f_n=f-g_n$ is 0 in the
        set $A_n$ and is $f$ otherwise.
        $$
        0=\int_{A_n}{f-g_n \dmu}=\int_{A_n}{f\dmu}-
        \int_{A_n}{g_n\dmu}.$$
        
        \textbf{Not able to continue. I cannot see how I can make progress on this
        question.}
    \end{mybox}
\end{enumerate}

\end{document}
