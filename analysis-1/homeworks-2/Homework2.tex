\documentclass[12pt]{article}
\usepackage[]{blindtext}
\usepackage[letterpaper, total{216mm, 279mm}]{}
\usepackage{amssymb,amsmath,amsfonts,verbatim}
\usepackage[breakable, skins]{tcolorbox}
\usepackage[parfill]{parskip}
\usepackage[english]{babel}
\usepackage{mathtools, amsthm}
\usepackage{amsfonts, xcolor}
\usepackage{amssymb}
\usepackage{mathrsfs}
\usepackage{verbatim}

\newtheorem{notes}{Notes}[section]
\newtheorem{prob}[notes]{Problems}
\newtheorem{thm}{Theorem}
\newtheorem{cor}[thm]{Corollary}
\newtheorem{lem}[thm]{Lemma}
\newtheorem{defn}[notes]{Definition}
\newtheorem{rem}[notes]{Remark}
\newtheorem{prop}[thm]{Proposition}

\newcommand{\rl}{\mathbb{R}}
\newcommand{\id}{\text{id}}
\newcommand{\dprime}{{\prime\prime}}
\newcommand{\xprime}{X^\prime}
\newtcolorbox{mybox}[0]{
    arc=0mm, enhanced, frame hidden, breakable,
}
\newcommand{\qedbox}{$\hfill\blacksquare$}
\setcounter{MaxMatrixCols}{10}

\setlength{\topmargin}{-.65in}
\setlength{\textwidth}{190mm} 
\setlength{\textheight}{240mm}
\setlength{\oddsidemargin}{-15mm} 
\setlength{\evensidemargin}{-15mm}
\parindent=0pt

\title{Analysis II \\
\large Homework 1
}
\author{Nutan Nepal}
\newcommand{\packpledge}{
    $\text{{\bf Pack Pledge:} I have neither given nor
    received unauthorized aid on this
    test or assignment.}$}

\begin{document}
\maketitle
\packpledge\\
\makebox[\linewidth]{\rule{200mm}{1pt}}
\vspace{1mm}


\newcommand{\mR}{\mathbb{R}}
\newcommand{\mN}{\mathbb{N}}
\newcommand{\mC}{\mathbb{C}}
\newcommand{\mQ}{\mathbb{Q}}
\newcommand{\mq}{\mathbb{Q}}
\newcommand{\cP}{\mathcal{P}}
\newcommand{\cB}{\mathcal{B}}
\newcommand{\cM}{\mathcal{M}}
\newcommand{\ds}{\displaystyle}
\newcommand{\al}{\alpha}
\newcommand{\li}{l^{\infty}}
\newcommand{\ep}{\varepsilon}
\newcommand{\de}{\delta}
\newcommand{\T}{\mathcal{T}}
\newcommand{\linf}{l^{\infty}}
\newcommand{\cD}{\mathcal{D}}
\newcommand{\cR}{\mathcal{R}}
\newcommand{\cN}{\mathcal{N}}
\newcommand{\lsn}{\limsup_{n \to \infty}}
\newcommand{\lin}{\liminf_{n \to \infty}}
\newcommand{\eq}{\Leftrightarrow}

\begin{enumerate}


\item Let $f: X \to \mR$, where $X$ is a measurable space.
Show that if $V_r:=\{ x \ | \ f(x) \geq r \}$ is measurable for every rational number $r$, then $f$ is measurable.

\begin{mybox}

  We note that for the open set $U_r:=(-\infty,r)$, $V_r=f^{-1}
  (U_r^c)=(f^{-1}(U_r))^c$. Thus, $(f^{-1}(U_r))$ is measurable
  in $X$. Now, for any open set $W\subset \rl$, we know that
  $W$ is an arbitrary union or finite intersections
  of the sets of the form $(a,b)$, so it suffices to show
  that $f^{-1}(a,b)$ is measurable.

  \vspace*{2mm}
  If $(a_n)\to a$ is a sequence in $\mq$ that decreases to
  $a$ then
  $$(a,b)=(-\infty,b)\cap \bigcup_{n=1}^\infty{(-\infty,
  a_n)^c}\implies
  f^{-1}(a,b)=f^{-1}(-\infty,b)\cap \bigcup_{n=1}^\infty{
  (f^{-1}(-\infty,a_n))^c}.
  $$
  Since each of the sets on the right are measurable,
  $f$ is measurable.
\end{mybox}

\item Let $f:\mR \to \mR$ such that $f^{-1}(c)$ is
measurable for each number $c$. Is $f$ necessarily measurable?

\begin{mybox}
  Let $E$ be a non-measurable set if $\rl$ and define a
  function $f:\rl\to\rl$ by $f(x)=2^x$ if $x\in E$ and
  $f(x)=-2^x$ if $x\notin E$. Then $f^{-1}(c)$ is empty if
  $c=0$ and a singleton set if $c\neq 0$. Thus $f^{-1}(c)$
  is measurable for each $c$. However, $f$ is not measurable
  since $f^{-1}(0,\infty)=E$ is not a measurable set.
\end{mybox}

\item Let $f :\mR \to \mR$ be measurable and $g :\mR \to \mR$ be continuous. Is the composition $f \circ g$ 
necessarily measurable?

\begin{mybox}
  For each measurable set $W\subset\rl$,
  $f^{-1}(W)$ is measurable. However the preimage of the
  measurable set $f^{-1}(W)$ under the continuous 
  function $g$ may not be measurable. Thus $f\circ g$
  is not necessarily measurable.
\end{mybox}

\item Give an alternate proof that if $f, g: X \to \mR$ are measurable, then so is $f+g$, by showing directly that 
$$ (f+g)^{-1}(a,\infty) = \{x \ | \ (f+g)(x) > a \}$$ 
is measurable for every $a \in \mR$. \\
Hint: Show that $\ds \{x \ | \ (f+g)(x) > a \} = \cup_{b \in \mQ} \big( \{x \ | \ f(x) > b \} \cap \{ x \ | \ g(x) > a-b\} \big)$. 

Similarly show directly that $cf$ (for $c$ constant) is measurable, as is $f^2$. Using these results, show that $fg$ is measurable.

\begin{mybox}
  Let $A_a=(f+g)^{-1}(a,\infty)=\{x \ | \ (f+g)(x) > a \}$
  for some $a\in \rl$.
  Since $(f+g)(x)=f(x)+g(x)$, if $f(x)>b$ for some
  $b\in \mq$, then
  $(f+g)(x)>a\implies g(x)>a-b$. Then $A_a=
  \{x \ | \ f(x) > b\ \text{and}\ g(x)>a-b\}$.
  So, $A_a =\{x \ | \ (f+g)(x) > a \} =\bigcup_{b \in \mQ}
  \left(\{x \ | \ f(x) > b \} \cap \{ x \ | \ g(x) > a-b\}
  \right)$ is a countable union of intersections of
  measurable sets. Thus, $A_a$ is measurable for every
  $a\in \rl$ and $f+g$ is measurable.

  \vspace*{2mm}
  Now, let $A_a=(cf)^{-1}(a,\infty)=\{x \ | \ (cf)(x)>a\}
  =\{x\ |\ f(x)>a/c\}$. But this is a measurable set since
  $f^{-1}(a/c,\infty)$ is measurable in $X$ for all
  $a\in \rl$. Thus $cf$ is measurable.

  \vspace*{2mm}
  Now, let $A_a=(f^2)^{-1}(a,\infty)=\{x \ | \ (f^2)(x)>a\}
  =\{x\ |\ f(x)>\sqrt{a}\}\cup \{x\ |\ f(x)>\sqrt{a}\}$.
  But this is a union measurable sets since
  $f^{-1}(\sqrt{a},\infty)$  and $f^{-1}(\sqrt{a},\infty)$
  are measurable in $X$ for all
  $a>0\in \rl$. For $a<0$, we have $(f^2)^{-1}(a,\infty)
  =X$. Thus $f^2$ is measurable.
\end{mybox}


\item Let $X$ be an uncountable set. Let 
$$ M = \{ E \subset X  \ \text{such that either} \ E \ \text{or}\ E^c \ \text{is countable}\}.$$
Set $\mu(E) = 0$ if $E$ is countable and $\mu(E) = 1$ if $E^c$ is countable. Show that $M$ is a $\sigma$-algebra and that $\mu$ is a measure on $M$. 

\begin{mybox}
  \begin{enumerate}
    \item Since the empty set $\emptyset$ is countable,
      and $X^c=\emptyset$, $\emptyset\in M$ and $X\in M$.

      \vspace*{1mm}
    \item If $\{E_k\}_{k=1}^\infty$ is a countable collection
      of sets in $M$ where either $E_k$ is countable or $E_k^c$
      is countable, then if all $E_k$ are countable, then
      $E=\bigcup_{k=1}^\infty{E_k}$ is a countable union of
      countable sets and hence is countable itself. Thus
      $E\in M$. If there exists one $E_k$ such that
      $E_k^c$ is countable, then $E^c$ is countable and
      thus $E\in M$.

      \vspace*{1mm}
    \item If $\{E_k\}_{k=1}^\infty$ is a countable collection
      of sets in $M$ where each $E_k^c$ is countable, then
      $E^c=\left(\bigcap_{k=1}^\infty{E_k}\right)^c$ is a countable intersection
      of countable sets and hence is countable itself and
      $E\in M$. If one of the $E_k$ is countable, then $E$ is
      countable and so $E\in M$.
  \end{enumerate}
  All sets $E\in X$ are either countable or uncountable,
  so image of $\mu=\{0,1\}$. If $\{E_k\}_{k=0}^\infty$ is a
  countable collection of measurable pairwise disjoint
  sets then, either

  \vspace*{2mm}
  \begin{enumerate}
    \item all $E_k$ are countable and hence
    $\bigcup_{k=1}^\infty{E_k}$ is countable, or
    \item one $E_k^c$ is countable and hence
    $\left(\bigcup_{k=1}^\infty{E_k}\right)^c$ is
    countable.
  \end{enumerate}

  We note that two distinct $E_i$ and $E_j$ cannot both
  have countable complements since $E_i\subset E_j^c$.
  Then,
  $$\mu\left(\bigcup_{k=1}^\infty{E_k}\right)=
  \begin{dcases}
    0=\sum_{i=1}^\infty{E_k} &\text{all $E_k$ are countable}\\
    1=0+1=\sum_{i=1}^\infty{E_k}&\text{one $E_k^c$ is countable.}
  \end{dcases}
  $$
  Thus $\mu$ is a measure on $M$.
\end{mybox}

\item Let $A$ and $B$ be any sets. Show that
$$\chi_{A\cap B} = \chi_A\cdot\chi_B,\ \ \
\chi_{A\cup B} = \chi_A + \chi_B - \chi_A\cdot\chi_B,\ \ \
\chi_{A^c}=1-\chi_A.$$
\begin{mybox}
  \begin{enumerate}
    \item $\chi_{A\cap B}(x)=1\iff x\in A\wedge x\in B\iff
    (\chi_A(x)=1)\wedge (\chi_B(x)=1).$
  
    And, $\chi_A\cdot\chi_B(x)=1\iff
    (\chi_A(x)=1)\wedge (\chi_B(x)=1).$ Thus
    $\chi_{A\cap B} = \chi_A\cdot\chi_B$.

    \vspace*{3mm}
    \item $\chi_{A\cup B}(x)=0\iff (x\notin A)\wedge
    (x\notin B)$.

    $\chi_A + \chi_B - \chi_A\cdot\chi_B=0\iff
    (x\notin A)\wedge (x\notin B)$. Thus
    $\chi_{A\cup B} = \chi_A + \chi_B - \chi_A\cdot\chi_B$.

    \vspace*{3mm}
    \item Taking $B=A^c$ in the above formula, we have,
    $\chi_{A\cup A^c}=\chi_A+\chi_{A^c}-\chi_A\cdot\chi_{A^c}.$
    But $\chi_{A\cup A^c}=1$ and $\chi_A\cdot\chi_{A^c}=0$ in
    all cases, and hence, $\chi_{A^c}=1-\chi_A$.
  \end{enumerate}
\end{mybox}

\item ``Continuity Property of Decreasing Intersections":
Let $A_n \in \cM$ s.t. $A_1 \supseteq A_2 \supseteq A_3 ... $, and $\mu(A_1) < \infty$.  Show that $\ds \mu(\cap_1^{\infty} A_n ) = \lim_{n \to \infty}\mu(A_n)$.  

\begin{mybox}
  We define $\{B_k\}_{k=1}^\infty$ collection of measurable
  sets by $B_k=A_1-A_k$. Then $\{B_k\}_{k=1}^\infty$ is an
  ascending collection of measurable sets with
  $$\mu\left(\bigcup_{k=1}^\infty{B_k}\right)=
  \lim_{k\to\infty}{\mu(B_k)}.$$
  We have $\bigcup_{k=1}^\infty{B_k}=\bigcup_{k=1}^\infty{
  (A_1-A_k)}=A_1-\bigcap_{k=1}^\infty{A_k}.$ Since
  $\mu(A_k)\leq\mu(A_1)<\infty$, we write $\mu(B_k)=
  \mu(A_1)-\mu(A_k)$. So,
  $$\mu\left(\bigcup_{k=1}^\infty{B_k}\right)=
  \mu\left(A_1-\bigcap_{k=1}^\infty{A_k}\right)
  =\mu(A_1)-\mu\left(\bigcap_{k=1}^\infty{A_k}\right)
  =\mu(A_1)-\lim_{k\to\infty}{\mu(A_k)}.$$
  Hence we have, $\ds\mu\left(\bigcap_{k=1}^\infty{A_k}\right)
  =\lim_{k\to\infty}{\mu(A_k)}$ as required.
\end{mybox}

\item  Suppose that $(X, \cM, \mu)$ is a measure space. Show that if $A_1, A_2, ... \in \cM$, but not necessarily pairwise disjoint, with $\mu(A_i) = 0$ for each $i$, then $\ds \mu(\cup_j A_j) = 0$.

\begin{mybox}
  Let $B_i=\bigcup_{j=1}^i{A_j}$, then $\{B_i\}_{i=1}^\infty$
  is an ascending collection of measurable sets and for
  each $i\in\mN$,
  $$\mu(B_i)= \mu\left(\bigcup_{j=1}^i{A_j}\right)
  \leq\sum_{j=1}^i{\mu(A_j)}=0.$$
  Then, by the continuity of measure,
  $$\mu\left(\bigcup_{j=1}^\infty{A_j}\right)
  =\mu\left(\bigcup_{j=1}^\infty{B_j}\right)
  =\lim_{j\to\infty}{\mu(B_j)}=0.$$
\end{mybox}

\item Let $(X, \cM, \mu)$ be a measure space. If $A$ and $B$ are disjoint measurable sets, and $\ds \mu(A \cup B) = \mu(A)$, must $\mu(B) = 0$?

\begin{mybox}
  Since $A$ and $B$ are disjoint,
  $\mu(A)=\mu(A\cup B)=\mu(A)+\mu(B)\implies \mu(B)=0$.
\end{mybox}

\item Let $(X, \cM, \mu)$ be a measure space, and let $B \in \cM$. Define $\nu(A) = \mu(A \cap B)$ for $A \in \cM$. Show that $\nu$ is a measure.

\begin{mybox}
  Since $\nu(A)=\mu(A\cap B)\leq \mu(A)\leq\infty$ and
  $\nu(A)\geq 0$.

  \vspace*{2mm}
  Now if $\{A_k\}_{k=0}^\infty$ is a countable pairwise
  disjoint collection of sets then $\{B\cap A_k\}_{k=0}
  ^\infty$ is a countable collection of pairwise disjoint
  sets. Then
  $$\nu\left(\bigcup_{n=0}^\infty{A_k}\right)
  =\mu\left(B\cap\bigcup_{n=0}^\infty{A_k}\right)
  =\sum_{n=0}^{\infty}{B\cap A_k}=\sum_{n=0}^{\infty}
  {\nu(A_k)}.$$
  Thus, $\nu$ is countably additive and is a measure.
\end{mybox}

\item Let $f: X \to [0, \infty]$ be a measurable function. Let
  $$s_n(x) =
  \begin{cases}
			n, & f(x) \geq n\\
			\frac{i-1}{2^n}, & \frac{i-1}{2^n} \leq f(x) < \frac{i}{2^n}, i = \overline{1, n\cdot 2^n}
	\end{cases}$$
Show that $\{s_n\}$ is a monotone increasing sequence and $s_n \to f$ pointwise as $n \to \infty$. 

\begin{mybox}
  We see that $s_n(x)\leq f(x)$ for all $x$. Clearly,
  $s_{n+1}(x)\geq s_n(x)$ for $f(x)\geq n$. Now, if
  $f(x)<n$, then $s_n=p/2^n$ where $p$ is the greatest
  number that is less than $2^nf(x)$. Then if $q=s_{n+1}(x)$
  is the greatest integer less than $2^{n+1}f(x)=2\cdot
  2^nf(x)$ then $q>p$. Thus $\{s_n\}$ is a monotone
  increasing sequence.

  \vspace*{2mm}
  From above, we have $s_n(x)=p/2^n> (2^nf(x)-1)/2^n
  =f(x)-1/2^n$
  since $p>2^nf(x)$ for all $x$. Then $f(x)-s_n(x)<1/2^n$
  and as $n\to \infty$, $s_n\to f$ pointwise.
\end{mybox}

\item Let $(X, \cM, \mu)$ be a measure space. Let $A \in \cM$.
Let $s$ and $v$ be non-negative, simple, measurable functions.
Let $\al, \beta \geq 0$. Show that 
$$\int_A (\al s + \beta v ) \ d\mu = \al  \int_A s \ d\mu + \beta \int_A v  \ d\mu$$
Moreover, if $s \leq v$ on $A$, then show that 
$$ \int_A s \ d\mu \leq \int_A v \ d\mu.$$

\begin{mybox}
  Now, we note that if $f=\alpha s+\beta v$,
  we choose a finite disjoint collection of measurable
  subsets $\{E_i\}_{i=1}^n$ of $A$ such that their union
  is $A$ and $s$ and $v$ are constant on each
  $E_i$. Let $p_i$ and $q_i$ be the values taken by $s$
  and $v$ for each $i$. Then
  $$\int_A{s\ d\mu}=\sum_{i=1}^n{p_i\cdot\mu(E_i)}
  \ \ \ \ \text{and}\ \ \ \
  \int_A{v\ d\mu}=\sum_{i=1}^n{q_i\cdot\mu(E_i)}.$$
  Then clearly,
  $$\int_A{\alpha s+\beta v\ d\mu}=
  \sum_{i=1}^n{(\alpha p_i+\beta q_i)\cdot\mu(E_i)}
  =\alpha\int_A{s\ d\mu} +\beta \int_A{v\ d\mu}.$$
  Now, if $s\leq v$ on $A$, then we take $r=v-s$ to be the
  simple non-negative function and by linearity, we have,
  $$\int_A{v\ d\mu}-\int_A{s\ d\mu}
  =\int_A{r\ d\mu}\leq 0$$
  as required.
\end{mybox}

\end{enumerate}

\end{document}
