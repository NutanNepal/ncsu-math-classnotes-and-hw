\documentclass[12pt]{article}
\usepackage[]{blindtext}
\usepackage[letterpaper, total{216mm, 279mm}]{}
\usepackage{amssymb,amsmath,amsfonts,verbatim}
\usepackage[breakable, skins]{tcolorbox}
\usepackage[parfill]{parskip}
\usepackage[english]{babel}
\usepackage{mathtools, amsthm}
\usepackage{amsfonts}
\usepackage{amssymb}
\usepackage{mathrsfs}
\usepackage{verbatim}

\newtheorem{notes}{Notes}[section]
\newtheorem{prob}[notes]{Problems}
\newtheorem{thm}{Theorem}
\newtheorem{cor}[thm]{Corollary}
\newtheorem{lem}[thm]{Lemma}
\newtheorem{defn}[notes]{Definition}
\newtheorem{rem}[notes]{Remark}
\newtheorem{prop}[thm]{Proposition}

\newcommand{\rl}{\mathbb{R}}
\newcommand{\id}{\text{id}}
\newcommand{\dprime}{{\prime\prime}}
\newcommand{\xprime}{X^\prime}
\newtcolorbox{mybox}[2][]{
    arc=0mm, enhanced, frame hidden, breakable
}
\newcommand{\qedbox}{$\hfill\blacksquare$}
\newcommand{\mR}{\mathbb{R}}
\newcommand{\mQ}{\mathbb{Q}}
\newcommand{\ds}{\displaystyle}
\newcommand{\al}{\alpha}

\setcounter{MaxMatrixCols}{10}

\setlength{\topmargin}{-.65in}
\setlength{\textwidth}{195mm} 
\setlength{\textheight}{240mm}
\setlength{\oddsidemargin}{-15mm} 
\setlength{\evensidemargin}{-15mm}
\parindent=0pt

\title{Analysis I \\
\large Homework 3
}
\author{Nutan Nepal}
\newcommand{\packpledge}{
    $\text{{\bf Pack Pledge:} I have neither given nor
    received unauthorized aid on this
    test or assignment.}$}

\begin{document}
\maketitle
\packpledge\\
\makebox[\linewidth]{\rule{200mm}{1pt}}
\vspace{1mm}

\begin{enumerate}

\item \textbf{Kreyszig p.303 / Problem 4.}
    It is important that in Banach's theorem 5.1-2
    the condition (1) cannot be replaced by $d(Tx,Ty)
    < d(x,y)$ when $x \neq y$. To see this, consider
    $X = \{x \mid 1 \leq x < \infty \}$, taken with
    the usual metric of the real line, and $T: X \to X$
    defined by $x \to x + x^{-1}$. Show that $|Tx-Ty| <
    |x-y|$ when $x \neq y$ but the mapping has
    no fixed points.

\begin{mybox}

    For the given map, we see that $$|Tx-Ty|=
    |x+x^{-1}-y-y^{-1}|=\left|x-y+\frac{y-x}{xy}\right|.$$
    Since $(x-y)$ and $(y-x)/xy$ have different signs,
    $\left|x-y+(y-x)/xy\right|<|x-y|$ and we
    see that $|Tx-Ty|<|x-y|$.
    
    \vspace*{3mm}
    Now, taking $T(x)=x$, we have $1/x=0$. But no point
    $x\in X$ satisfies this. Thus, $T$ has no fixed point
    in $X$.
\end{mybox}


\item {\bf Kreyszig p.303 / Problem 6.}
    If $T$ is a contraction, show that $T^n$,
    $(n \in \mathbb{N})$ is a contraction. If $T^n$ is a
    contraction for an $n > 1$, show that $T$
    need not be a contraction. 
\begin{mybox}

    If $T$ is a contraction on a metric space $X$,
    then there is a positive
    real number $\alpha<1$ such that
    $$d(Tx,Ty)\leq \alpha d(x,y)$$
    for all $x$, $y\in X$. Then, for $n\in \mathbb{N}$,
    $$d(T^nx,T^ny)=d(T\cdot T^{n-1}x, T\cdot T^{n-1}y)
    \leq \alpha d(T^{n-1}x, T^{n-1}y)$$
    Continuing this process, we get
    $$d(T^nx,T^ny)<\alpha^n d(x,y).$$
    Since, $0<\alpha^n<1$, $T^n$ is a contraction.
    To show that $T^n$ being a contraction does
    not imply that $T$ is a contraction, we define $T:\rl
    \to \rl$ and by
    $$T(x)=
    \begin{dcases}
        -2x,   & x<0\\
        x/4,  & x\geq 0
    \end{dcases}$$
    We see that $T$ increases the distance when $x<0$, so
    it is not a contraction. But $T(X)=[0,\infty)$, so
    $T^2$ decreases the distance by $1/4$. Hence $T^2$
    is a contraction.
\end{mybox}
 
 
\item {\bf Kreyszig p.32 / Problem 2.}
    If $(x_n)$ is Cauchy and has a convergent
    subsequence, say, $x_{n_k} \to x$, show that
    $(x_n)$ is convergent with the limit $x$. 
\begin{mybox}

    If $\{x_{n_k}\}_{n_k=1}^\infty$ is the convergent
    subsequence of the sequence $\{x_n\}_1^\infty$
    in the metric space $X$, then as $x_{n_k}\to x$,
    for every $\varepsilon>0$ there exists $N_1\in\mathbb{Z}$
    such that for all $n_k>N_1$, we have 
    $$d(x_{n_k},x)<\varepsilon/2.$$
    Also, as $\{x_n\}_1^\infty$ is Cauchy in $X$, for
    $\varepsilon>0$ (same as above), there exists $N$ such
    that for all $m$, $n>N$ we have, 
    $$d(x_n,x_m)<\varepsilon/2.$$
    Since there are infinitely many terms in the
    subsequence, we have $M$ such that $n_k>N$ for all $k>M$.
    Then, for all $k>M$ and $n>N$ we have,
    $$d(x_n,x)\leq d(x_n,x_{n_k})+d(x_{n_k},x)=\varepsilon
    /2+\varepsilon/2=\varepsilon.$$
    Since, this is true for any arbitrary $\varepsilon>0$,
    we have $x_n\to x$.
\end{mybox}

\item \textbf{Kreyszig p.56 / Problem 5.}
    Show that $\{x_1, \cdots, x_n \}$, where
    $x_j(t) = t^j$, is a linearly independent set
    in the space $C[a,b]$.
\begin{mybox}

    We note that if the linear combinations of
    $x_j(t) = t^j$ is identically 0, that is
    $$\alpha_1 t+\alpha_2 t^2+\cdots +\alpha_n
    t^n=0$$
    then equating the corresponding coefficients, we
    must have each $\alpha_i=0$. Hence the set
    $\{x_1, \cdots, x_n \}$ must be linearly independent.
\end{mybox}
 
\item \textbf{Kreyszig p.56 / Problem 10.}
    If $Y$ and $Z$ are subspaces of a vector space
    $X$, show that $Y \cap Z$ is a subspace of $X$.
\begin{mybox}
    
    For $x$, $y\in Y\cap Z$, we see that $x+y\in Y$
    and $x+y\in Z$ since $Y$ and $Z$ are both subspaces
    of $X$. So, $x+y\in Y\cap Z$ and similarly, for
    $\alpha\in \rl$, $\alpha \cdot x\in Y$ and
    $\alpha\cdot x\in Z$. So, $\alpha\cdot x\in Y\cap Z$.
    Since, $Y\cap Z$ is closed under addition and scalar
    multiplication, it is a subspace of $X$. (Since
    $Y$ and $Z$ are subset of $X$, they satisfy the
    linearity property.)
\end{mybox}
 
\item \textbf{Kreyszig p.66 / Problem 11.}
    Show that the closed unit ball
    $$B(0,1)=\{ x\in X :\|x\|=1\}$$
    in a normed space $X$ is convex.
\begin{mybox}

    We need to show that if $x$, $y\in B(0,1)$
    then any point $z$ given by $z=\alpha x+(1-\alpha)y$
    for some $\alpha\in [0,1]$ is also
    in $B(0,1)$. Since $x$, $y\in B(0,1)$,
    we see that
    $$\|z\|=\|\alpha x+(1-\alpha)y\|\leq
    |\alpha|\|x\|+|1-\alpha|\|y\|\leq \alpha+1-\alpha=1.$$
    So, $z\in B(0,1)$ and hence $B(0,1)$ is convex.
\end{mybox}
 
\item \textbf{Kreyszig p.70 / Problem 2.}
    Show that $c_0$ in Prob. 1 (the space of all sequences
    of scalars converging to zero) is a closed
    subspace of $l^{\infty}$, so that $c_0$ is
    complete by 1.5-2 and 1.4-7.
\begin{mybox}

    To show that $c_0$ is closed, we show that
    every sequence $\{x_i^k\}$ in $c_0$ has a
    limit point in $c_0$. Here, each $x_i$ is a
    sequence of scalars that converge to 0. Since
    $l^\infty$ is complete, let $y\in l^\infty$ be
    the limit point of the sequence
    $\{x_i^k\}$ in $c_0$. Then for all $\varepsilon>0$,
    we have $N$ such that
    $$\sup_{1\leq i<\infty} |x_i^k-y_i|<\varepsilon/2$$
    for all $k>N$. Since for each $k$, $x_i^k\to 0$,
    we have for each $k$, there exists $N_1\in \mathbb{N}$
    such that 
    $$|x_i^k|=|x_i^k-0|<\varepsilon/2$$
    for all $i>N_1$. Then
    $$|y_i-0|=|y_i|\leq|x_i^k-y_i|+|x_i^k|<\varepsilon$$
    for all $i>N_1$ and $k>N$. This means that $y_i\to 0$
    and thus $\{y_i\}_1^\infty\in c_0$. Thus, $c_0$ is
    closed.
\end{mybox}
 
\item \textbf{Kreyszig p.70 / Problem 4.}
    \textbf{(Continuity of vector space operations)}
    Show that in a normed space $X$, vector addition and
    multiplication by scalars are continuous operations with
    respect to the norm; that is, the mappings defined by
    $(x,y) \to x+y$ and $(\alpha, x) \to \alpha x$ are
    continuous.
\begin{mybox}

    We define the norm in $X\times X$ to be
    $$\|(x,y)\|_{X\times X}=\max\ \{\|x\|_X,\|y\|_X\}$$
    and we define the norm in $\rl\times X$ to be
    $$\|(\alpha,y)\|_{\rl\times X}=
    \max\ \{|\alpha|,\|y\|_X\}.$$
    Then, for an arbitrary point $(x_0,y_0)\in X\times X$,
    for every $\varepsilon>0$, we have $\delta=\varepsilon
    /2$ such that
    $$\|(x,y)-(x_0,y_0)\|_{X\times X}=
    \|(x-x_0,y-y_0)\|_{X\times X}=\max\ \{\|x-x_0\|_X,
    \|y-y_0\|_X\}<\delta$$
    implies
    $$\|T(x,y)-T(x_0,y_0)\|_X=
    \|x-x_0+y-y_0\|_X\leq \|x-x_0\|+
    \|y-y_0\|_X<\delta+\delta= \varepsilon.$$
    Hence, the map $T(x,y)=x+y$ is continuous.

    \vspace*{3mm}
    Now, for the map $T:\rl\times X\to X$ given by
    $T(\alpha, x)=\alpha x$, for every $\varepsilon>0$,
    we take $\delta=\min\{\varepsilon/2(|\alpha_0|+1),
    \varepsilon/2(\|x_0\|_X+1), 1\}$, we get
    $$\|(\alpha,x)-(\alpha_0,x_0)\|=
    \|(\alpha-\alpha_0,x-x_0)\|=
    \max\{|\alpha-\alpha_0|, \|x-x_0\|_X\}<\delta$$
    implies
    \begin{align*}
        \|T(\alpha,x)-T(\alpha_0,x_0)\|_X=
    \|\alpha x-\alpha_0x_0\|_X\leq &|\alpha_0|\|x-x_0\|+
    |\alpha-\alpha_0|\|x\|_X\\
    \leq&|\alpha_0|\|x-x_0\|+|\alpha-\alpha_0|(\|x-x_0\|_X
    +\|x_0\|_X)\\
    <&\delta|\alpha_0|+\delta(\delta + \|x_0\|_X)\\
    \leq&\frac{\varepsilon|\alpha_0|}{2(|\alpha_0|+1)}+
    \frac{\varepsilon}{2(\|x_0\|_X+1)}(\|x_0\|_X+1)
    <\varepsilon.
    \end{align*}
    Hence, $T$ is continuous.
\end{mybox}


\item \textbf{Kreyszig p.70 / Problem 5.}
    Show that $x_n \to x$ and $y_n \to y$ implies
    $x_n + y_n \to x + y$. Show that $\alpha_n \to \alpha$
    and $x_n \to x$ implies $\alpha_n x_n \to \alpha x$.
\begin{mybox}

    If $x_n \to x$, $y_n \to y$ and $\alpha_n\to\alpha$,
    then for every
    $\varepsilon>0$ we have $N\in \mathbb{Z}$
    such that for all $n>N$,
    $$\|x_n-x\|<\varepsilon/2, \ \
    \|y_n-y\|<\varepsilon/2\ \ \text{ and }
    \ \ \|\alpha_n-\alpha\|<\varepsilon/2.$$
    Then for the same $N$, $$\|(x_n+y_n)-(x+y)\|=
    \|(x_n-x)-(y_n-y)\|\leq\|x_n-x\|+\|y_n-y\|<\varepsilon.
    $$
    Hence, $x_n+y_n \to x+y$ and similarly,
    $$\|\alpha_n x_n-\alpha x\|=
    \|\alpha_n x_n-\alpha_n x +\alpha_n x-\alpha x\|=
    \|\alpha_n(x_n-x) +x(\alpha_n-\alpha)\|
    \leq|\alpha_n|\|x_n-x\|+\|y\||\alpha_n-\alpha|<\varepsilon.
    $$
    Hence $\alpha_n x_n\to\alpha x$.
\end{mybox}

 
\item \textbf{Kreyszig p.71 / Problem 9.}
    Show that in a Banach space, an absolutely
    convergent series is convergent.
\begin{mybox}

    If a series $\sum_1^\infty{x_i}$ in $X$
    absolutely converges, then
    the sequence of partial sums $\{y_i\}_1^\infty$
    given by
    $$y_k=\sum_{i=1}^k{\|x_i\|}$$ converges to some
    point $y\in \rl$. Hence, $\{y_i\}_1^\infty$ is Cauchy
    and for every $\varepsilon>0$, we have $N\in \mathbb{Z}
    $ such that for all $m>n>N$,
    $$|y_m-y_n|=\sum_{i=n+1}^m{\|x_i\|}<\varepsilon.$$
    Now, if $\{s_i\}_1^\infty$ is the sequence of
    partial sums of the sequence $\{x_i\}_1^\infty$
    given by $s_k=\sum_{i=1}^k{x_i}$, then
    for the same $\varepsilon>0$ and $m>n>N$ as above,
    we have
    $$\left\|s_m-s_n\right\|=\left\|\sum_{i=n+1}^m{x_i}
    \right\|\leq \sum_{i=n+1}^m{\|x_i\|}<\varepsilon.$$
    Hence, the sequence $\{s_i\}_1^\infty$ is Cauchy in
    $X$ and since $X$ is complete, the sequence converges
    in $X$. Hence, the absolutely convergent series is
    convergent.
\end{mybox}
 
\item \textbf{Kreyszig p.71 / Problem 12.}
    A seminorm on a vector space $X$ is a mapping
    $p : X \to \mathbb{R}$ satisfying (N1),
    (N3), (N4) in Sec. 2.2 (Some authors call this
    a pseudonorm.) Show that $$p(0) = 0,$$ $$|p(y)-p(x)|
    \leq p(y-x).$$ (Hence if $p(x) = 0$ implies $x = 0$
    then $p$ is a norm.)
\begin{mybox}

    From (N3), we have $p({\bf 0})=p(0\cdot{\bf 0})
    =0\cdot p({\bf 0})=0$.
    
    \vspace*{3mm}
    Again, from (N4), we have $p(x+z)\leq p(x)+p(z)$
    for all $x$, $z\in X$. Then, taking $z=y-x$, we get
    $p(y)\leq p(x)+p(y-x)\implies p(y)-p(x)\leq p(y-x)$.
    Similarly, we have $p(x)=p(x-y+y)\leq p(x-y)+p(y)
    \implies p(x)-p(y)\leq p(x-y)=p(y-x)$.
    Combining these two inequalities, we get
    $$|p(y)-p(x)|
    \leq p(y-x).$$
    
\end{mybox}

\item \textbf{Kreyszig p.76 / Problem 9.}
    If two norms $\| \cdot \|$ and $\| \cdot \|_0$
    on a vector space $X$ are equivalent,
    show that $(i)$ $\|x_n - x \| \to 0$ implies
    $(ii)$ $\|x_n -x \|_0 \to 0$ (and vice versa,
    of course).
\begin{mybox}
    
    If the two norms $\| \cdot \|$ and $\| \cdot \|_0$
    on a vector space $X$ are equivalent, then
    for some $c>0$, we have
    $$0\leq \|x_n-x\|_0\leq c\|x_n-x\|.$$
    Taking limit as $n\to\infty$, we see that, by squeeze
    theorem $\|x_n-x\|\to 0 \implies |x_n-x\|_0\to 0.$
    Similarly for some $c_1>0$, we have
    $$0\leq \|x_n-x\|\leq c_1\|x_n-x\|_0$$
    which after taking limit as $n\to\infty$
    proves the converse statement.
\end{mybox}

\item Let $X$ be a finite dimensional v.s. over
    $\mR$, with basis $\{e_1, ..., e_n\}$.
\begin{enumerate}
    \item Show that for any $1 \leq p \leq \infty$,
    the map $\| \cdot \|_{p} $ defined by
    $$ x = \sum_1^n x_i e_i \ \to \|x\|_p =
    \Big(\sum_1^n|x_i|^p \Big)^{1/p} , \ \ \text{for}
    \ \ 1 \leq p < \infty$$
    $$ x = \sum_1^n x_i e_i \ \to \|x\|_{\infty} =
    \max_{1 \leq i \leq n} |x_i| \ \ \text{for} \ \ p =
    \infty$$
    is a norm on $X$.
    \begin{mybox}

        \begin{enumerate}
            \item For $p<\infty$, $\|x\|_p=
            \left(\sum_1^n|x_i|^p\right)^{1/p}$ and each
            $|x|\geq 0$, so we have $\|x\|\geq 0$ for all
            $x\in X$. Similarly, for $p=\infty$, since
            $|x_i|\geq 0$, $\|x\|_\infty\geq 0$.

            \vspace*{2mm}
            \item If $\|x\|_p=
            \left(\sum_1^n|x_i|^p\right)^{1/p}=0$, then
            each $|x_i|=0\implies x_i=0\implies x=0$.
            And, if $x=0$ then each $|x_i|=0$ and so
            $\|x\|=0$.

            \vspace*{2mm}
            Similarly, for if $\|x\|_p=
            \max_{1\leq i\leq n}|x_i|=0$,
            then each $|x_i|$ must be 0 and so
            $x=0$.
            And, if $x=0$ then each $|x_i|=0$ and so
            $\|x\|_\infty=0$.

            \vspace*{2mm}
            \item For $\alpha\in \rl$, $$\|\alpha x\|_p
            =\left(\sum_1^n|\alpha x_i|^p\right)^{1/p}
            =\left(\alpha^p\sum_1^n|x_i|^p\right)^{1/p}
            =|\alpha|\|x\|.$$
            $$\|\alpha x\|_\infty=\max_{1\leq i\leq n}
            |\alpha x_i|=|\alpha|\max_{1\leq i\leq n}|x_i|
            =|\alpha|\|x\|_\infty.$$

            \vspace*{2mm}
            \item If another element $y\in X$ is given
            by $y=\sum_{i=1}^n{y_ie_i}$, then we take
            sequences $x'=\{x_i'\}_1^\infty$ and
            $y'=\{y_i'\}_1^\infty$ in $l^p$ such that
            $x_i'=x_i$ and $y_i'=y_i$ for $i=1,\ldots,n$
            and $x_i'=y_i'=0$ for $i>n$. Then, we have
            $$\left(\sum_{i=1}^n{|x_i|^p}\right)^{1/p}=
            \left(\sum_{i=1}^n{|x_i+y_i|^p}\right)^{1/p}$$
            and by Minkowski's inequality, we have
            $$\|x+y\|_p=
            \left(\sum_{i=1}^n{|x_i+y_i|^p}\right)^{1/p}
            =\left(\sum_{i=1}^\infty{|x_i'+y_i'|^p}\right)^{1/p}
            \leq \left(\sum_{i=1}^\infty{|x_i|^p}\right)^{1/p}
            +\left(\sum_{i=1}^\infty{|y_i|^p}\right)^{1/p}$$
            Hence $\|x+y\|_p\leq \|x\|_p+\|y\|_p$ for
            $1\leq p<\infty$.

            \vspace*{2mm}
            For $p=\infty$, we have $$\|x+y\|_p=
            \max\{|x_1|,\ldots,|x_n|,|y_1|,\ldots,|y_n|\}
            \leq \max\{|x_1|,\ldots,|x_n|\} +
            \max\{|y_1|,\ldots,|y_n|\}$$
            So, $\|x+y\|_p\leq \|x\|_p+\|y\|_p$ for
            $p=\infty$.
        \end{enumerate}
        
    \end{mybox}

    \item Show that for $1 \leq p \leq  \infty$,
    $(X, \| \cdot\|)$ is separable.
    \begin{mybox}

        We take the set $M$ consisting of elements of $X$
    with rational coordinates given by
        $$M=\left\{x\in X:\ x= \sum_{i=1}^n{\lambda_ie_i
        ,\ \lambda_i\in \mathbb{Q}}\right\}.$$
        $M$ is countable since $M$ is a countable
        union of countable sets. To show that $M$ is dense
        in $X$, we show that every open neighborhood of
        an arbitrary point contains a point of $M$.
        For a point $y=\sum_{i=1}^n{y_ie_i}$ we note that
        its $\varepsilon$-neighborhood is given by
        $$N_\varepsilon(y)=\left\{x\in X: \ 
        \|y-x\|<\varepsilon\right\}.$$
        For $1\leq p<\infty$,
        $$N_\varepsilon(y)=\left\{x\in X: \ 
        \sum_{i=1}^n{|y_i-x_i|^p}
        <\varepsilon^p\right\}.$$
        Then since rational numbers are dense in $\rl$,
        we can choose rational numbers $x_i$ such that
        $|y_i-x_i|<\varepsilon/n^{1/p}$. Then clearly,
        $x=\sum_{i=1}^n{x_ie_i}\in N_\varepsilon(y)$.
        Hence, $X$ is separable.

        \vspace*{2mm}
        Now, or $p=\infty$,
        $$N_\varepsilon(y)=\left\{x\in X: \ 
        \max_{i=\overline{1,\ldots,n}}\{|y_i-x_i|\}
        <\varepsilon\right\}.$$
        Then, we can choose each rational number
        $x_i$ such that
        $x_i\in (y_i, y_i+\varepsilon)$. Then
        $x\in N_\varepsilon(y)$ and hence $M$ is dense in
        $X$. So, $X$ is separable.
    \end{mybox}
\end{enumerate}

 
\item Prove that any vector space can be normed.
\begin{mybox}

    We note that every vector space $V$ has a Hamel basis
    $B$ such that every element $x$ of the vector space
    $V$ can be written as a linear combination of finitely
    many elements of $B$ with non-zero scalars
    as coefficients. Then, if for an element $x\in V$,
    $b_1,\ldots, b_k$ are the finitely many elements
    of $B$ with non-zero coefficients, then
    $$x=\lambda_1 b_1+\cdots+\lambda_k b_k$$
    and we define the norm of $x$ to be
    $$\|x\|=\max_{i=\overline{1, k}}\ |\lambda_i|.$$
    Clearly, $\|x\|\geq 0$ and if $\|x\|=0$ then
    $x=0$. Also, $x=0\implies \|x\|=0$ and $\|\alpha x\|
    =\max_{i=\overline{1, k}}\ |\alpha\lambda_i|
    =|\alpha|\|x\|$. Furthermore, if $y=\beta_1 c_1+
    \cdots +\beta_n c_n$ is another element of $V$,
    then $x+y=\lambda_1 b_1+\cdots+\lambda_k b_k+
    \beta_1 c_1+\cdots +\beta_n c_n$. And
    $$\|x+y\|=\max\{\lambda_1,\ldots,\lambda_k,\beta_1,
    \ldots,\beta_n\}\leq \max\{\lambda_1,\ldots,\lambda_k\}
    +\max\{\beta_1,\ldots,\beta_n\}=\|x\|+\|y\|.$$
    We see that $\|\cdot\|$ satisfies all the conditions
    of a norm on the vector space $V$.
\end{mybox}
 
\item Let $X$ be a finite dimensional normed space.
    Prove that any closed and bounded subset is compact.
\begin{mybox}

    Let $M$ be a closed and bounded subset of the
    finite dimensional vector space $X$ with dim $X=n$
    and basis $\{e_1,\ldots,e_n\}$. Let $\{x_i\}_1^\infty$
    be a sequence in $M$, then each $x_i$ has a unique
    linear representation
    $$x_i=\lambda_i^1 e_1+\cdots+\lambda_i^n e_n.$$
    By the lower bound theorem, we have
    $\|x_i\|\geq c\sum_{k=1}^n{|\lambda_i^k|}$ for some
    $c>0$ and since
    $M$ is a bounded set we have, for some real positive
    number $c_2$, $\|x_i\|\leq c_2$. Taking
    $c_1=\sum_{k=1}^n{|\lambda_i^k|}$, we get
    $$c_1\leq\|x_i\|\leq c_2.$$
    Then for each $k$, we have
    $c_1\leq\|\lambda_i^k\|\leq c_2$
    and for a fixed $k$, we have the sequence
    $\{\lambda_i^k\}_1^\infty$ of real numbers
    which is bounded. By Bolzano-Weierstrass theorem,
    we know that this sequence has a convergent
    subsequence which converges to a point $\beta^k$ in
    $\rl$. And hence, we have a corresponding
    convergent subsequence $\{x_{i_k}\}$ of $\{x_i\}$
    which converges to $z=\sum{\beta_i e_i}$. Since, $M$
    is closed, this limit point is in $M$. So, $M$ is compact.

\end{mybox}
 
\item Prove Riesz's Lemma.
\begin{mybox}

    \begin{thm}[Riesz's Lemma]

        Let $Y$ and $Z$ be subspaces of a normed space
        $X$ (of any dimension), and suppose that $Y$ is
        closed and is a proper subset of $Z$. Then for
        every real number $\theta$ in the interval
        $(0,1)$ there is a $z\in Z$ such that
        $$\|z\|=1, \hspace*{10mm} \|z-y\|\geq \theta
        \text{ for all } y\in Y.$$
    \end{thm}
    \begin{proof}
        For $v\in Z-Y$, we denote the distance between
        $v$ and the subspace $Y$ by
        $a=\inf_{y\in Y}{\|v-y\|}$. Since $v\notin Y$,
        $a>0$ and for some $\theta\in (0,1)$, there
        exists a $y_0\in Y$ such that
        $$a\leq \|v-y_0\|\leq a/\theta.$$
        Let $z=c(v-y_0)$ where $c=1/\|v-y_0\|$. Then
        $\|z\|=1$ and
        $$\|z-y\|=\|c(v-y_0)-y\|=c\|v-y_0-c^{-1}y\|
        =c\|v-y_1\|$$
        where $y_1=y_0+c^{-1}y$ which is in $Y$ (since
        $Y$ is a subspace). Hence $\|v-y_1\|\geq a$ and
        $$\|z-y\|=c\|v-y_1\|\geq \frac{a}{\|v-y_0\|}
        \geq a/(a/\theta)=\theta.$$
    \end{proof}
        
\end{mybox}
 
\item If a normed space $X$ has the property that the
    closed unit ball $\overline{B}(0,1)$ is compact,
    then $X$ is finite dimensional.
\begin{mybox}

    We show that if the closed unit ball $\overline{B}$
    is compact, and
    $X$ is of infinite dimension, then we get a
    contraction. Let $x_1\in X$ such that $\|x\|=1$
    for some norm $\|\cdot\|$. Then the span of $x_1$
    is a subspace $M_1$ of dimension 1 and hence,
    is closed and
    proper subspace of $X$. Then, by Riesz's Lemma,
    taking $\theta=1/2$, there exists $x_2\in X$ of norm
    1 such that $\|x_2-x_1\|\geq 1/2$. The elements
    $x_1$ and $x_2$ now generate a 2 dimensional
    closed and proper subspace $M_2$ of $X$ and again,
    by Riesz's lemma, there exists an $x_3$ of norm 1 such
    that $\|x_3-x_1\|\geq 1/2$ and $\|x_3-x_2\|\geq 1/2$.
    Then inductively, we obtain a sequence
    $\{x_n\}_1^\infty$ in $\overline{B}$
    such that for all integers $m\neq n$, we have
    $$\|x_m-x_n\|\geq \frac{1}{2}.$$
    But this sequence cannot have a convergent subsequence
    even though $\overline{B}$ is compact. Hence, our
    assumption that the dimension of $X$ is infinite
    cannot be true.\qed
\end{mybox}

\end{enumerate}
\end{document}