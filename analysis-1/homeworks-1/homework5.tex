\documentclass[12pt]{article}
\usepackage[]{blindtext}
\usepackage[letterpaper, total{216mm, 279mm}]{}
\usepackage{amssymb,amsmath,amsfonts,verbatim}
\usepackage[breakable, skins]{tcolorbox}
\usepackage[parfill]{parskip}
\usepackage[english]{babel}
\usepackage{mathtools, amsthm}
\usepackage{amsfonts}
\usepackage{amssymb}
\usepackage{mathrsfs}
\usepackage{verbatim}

\newtheorem{notes}{Notes}[section]
\newtheorem{prob}[notes]{Problems}
\newtheorem{thm}[]{Theorem}
\newtheorem{cor}[thm]{Corollary}
\newtheorem{lem}[thm]{Lemma}
\newtheorem{defn}[notes]{Definition}
\newtheorem{rem}[notes]{Remark}
\newtheorem{prop}[thm]{Proposition}

\newcommand{\rl}{\mathbb{R}}
\newcommand{\id}{\text{id}}
\newcommand{\dprime}{{\prime\prime}}
\newcommand{\xprime}{X^\prime}
\newtcolorbox{mybox}[2][]{
    arc=0mm, enhanced, frame hidden, breakable
}
\newcommand{\qedbox}{$\hfill\blacksquare$}
\newcommand{\mR}{\mathbb{R}}
\newcommand{\mQ}{\mathbb{Q}}
\newcommand{\ds}{\displaystyle}
\newcommand{\al}{\alpha}
\newcommand{\cD}{\mathcal{D}}
\newcommand{\cR}{\mathcal{R}}
\newcommand{\cN}{\mathcal{N}}
\newcommand{\inn}[2]{\left\langle #1, #2 \right\rangle}

\setcounter{MaxMatrixCols}{10}

\setlength{\topmargin}{-.65in}
\setlength{\textwidth}{195mm} 
\setlength{\textheight}{240mm}
\setlength{\oddsidemargin}{-15mm} 
\setlength{\evensidemargin}{-15mm}
\parindent=0pt

\title{Analysis I \\
\large Homework 5
}
\author{Nutan Nepal}
\newcommand{\packpledge}{
    $\text{{\bf Pack Pledge:} I have neither given nor
    received unauthorized aid on this
    test or assignment.}$}

\begin{document}
\maketitle
\packpledge\\
\makebox[\linewidth]{\rule{200mm}{1pt}}
\vspace{1mm}

\begin{enumerate}

\item Let $1 < p < \infty$.
Show that the dual space of $l^p$ is $l^q$. 

\begin{mybox}

    Taking the Schauder basis $(e_k)=(\delta_{kj})$ for
    the space $l^p$, we see that every $x\in l^p$ has a
    unique representation $x=\sum_{i=1}^\infty{\xi_i e_i}$.
    For the linear and bounded operator $f\in (l^p)'$, we
    have
    $$f(x)=\sum_{k=1}^\infty{\xi_k \gamma_k}\hspace*{10mm}
    \text{where }\gamma_k=f(e_k).$$
    We consider $x_n=(\xi_k^n)$ with
    $$\xi_k^n=\begin{dcases}
        |\gamma_k|^q/\gamma_k &\text{ if }k\leq n
        \text{ and } \gamma_k\neq 0\\
        0 &\text{if } k>n \text{ or } \gamma_k=0
    \end{dcases}$$
    where $q$ is the conjugate of $p$. Then we obtain
    $f(x_n)=\sum_{k=1}^\infty{\xi_k^n\gamma_k}=
    \sum_{k=1}^n{|\gamma_k|^q}$ and
    \begin{align*}
        f(x_n)\leq \|f\|\|x_n\|=&\|f\|\left(\sum_{k=1}
        ^\infty{|xi_k^n}|^p\right)^{1/p}\\
        =&\|f\|\left(\sum_{k=1}^n{|\gamma_k|^{(q-1)p
        }}\right)^{1/p}\\
        =&\|f\|\left(\sum_{k=1}^n{|\gamma_k|^q}\right)
        ^{1/p}.
    \end{align*}
    So we have, $\sum_{k=1}^n{|\gamma_k|^q}\leq
    \|f\|\left(\sum_{k=1}^n{|\gamma_k|^q}\right)^{1/p}
    \implies \left(\sum_{k=1}^n{|\gamma_k|^q}\right)^{1/q}
    \leq\|f\|$. Letting $n\longrightarrow \infty$, we see
    that $(\gamma_k)=(f(e_k))\in l^q$ since the infinite
    sum is bounded. Now, conversely, for any sequence
    $b=(\beta_k)\in l^q$, we define the correspondinf
    bounded linear functional $g\in l^p$ by $g(x)
    =\sum_{k=1}^\infty{\xi_k\beta_k}$ where $x=(\xi_k)
    \in l^p$. $g$ is linear and bounded by Holder's
    inequality. Thus $g\in (l^p)'$.

    \vspace*{3mm}
    Now, we show that the norm of the functional $f$ is
    the norm on the space $l^q$. We have
    $$|f(x)|=\left|\sum_{k=1}^\infty{\xi_k\gamma_k}
    \right|\leq\left(\sum_{k=1}^\infty{|\xi_k\|^p}
    \right)^{1/p}\left(\sum_{k=1}^\infty{|\gamma_k\|^q}
    \right)^{1/q}=\|x\|\left(\sum_{k=1}^\infty
    {|\gamma_k\|^q}\right)^{1/q}.$$
    Taking sup over all $x$ of norm 1 we obtain $\|f\|
    \leq \left(\sum_{k=1}^\infty
    {|\gamma_k\|^q}\right)^{1/q}$. But we also have 
    $\left(\sum_{k=1}^n{|\gamma_k|^q}\right)^{1/q}
    \leq\|f\|$ so $\left(\sum_{k=1}^n
    {|\gamma_k|^q}\right)^{1/q}=\|f\|$. Thus we see that
    the mapping of $(l^p)'$ to $l^q$ is linear and
    bijective and also norm preserving.
\end{mybox}


\item Prove the completion theorem for inner
product spaces.
\begin{mybox}

    \begin{thm}[Completion]
        For any inner product space $X$ there exists a
        Hilbert space
        $H$ and an isomorphism $A$ from $X$ into a dense
        subspace $W\subset H$. The space $H$
        is unique except for isomorphisms.

    \end{thm}
    \begin{proof}
        By completion of Banach spaces, there exists
        a Banach space $H$ and an isometry $A$ from $X$
        onto a dense subspace $W$ of $H$. For continuity,
        $A$ preserves sums and scalar multiplications and
        hence, $A$ is an isomorphism of normed spaces.
        By continuity of inner product, we can define
        an inner product on $H$ by
        $\inn{\hat{x}}{\hat{y}}=\lim_{n\to\infty}
        {\inn{x_n}{y_n}}$ where $\{x_n\}$ and $\{y_n\}$
        are sequences in $X$ converging to $x$ and
        $y$ respectively and they are representatives of
        $\hat{x}$ and $\hat{y}$ in $H$. Since the inner
        product is continuous, the
        parallelogram and polarization identities are
        also preserved and hence,
        we see that $A$ is an isomorphism of
        inner product spaces from $X$ onto $W$.

        The space $H$ is unique except for isomorphisms
        by the completion of Banach spaces theorem.
    \end{proof}
\end{mybox}
 
 
\item \textbf{Kreyszig p.135 / Problem 4.}
If an inner product space $X$ is real, show that the
condition $\|x\|= \|y\|$ implies $\langle x + y, x - y
\rangle = 0$.
What does this mean geometrically if $X = \rl^2$? What
does the condition imply if $X$ is complex?
\begin{mybox}

    We have $\langle x + y, x - y \rangle = 
    \langle x, x \rangle - \langle x, y \rangle+
    \langle y, x \rangle- \langle y, y \rangle=
    \|x\|^2-\|y\|^2- \langle x, y \rangle+
    \langle x, y \rangle=0.$ If $X=\rl^2$, then we see
    that $x+y$ and $x-y$ are orthogonal for all $x$ and
    $y$.
\end{mybox}

\item \textbf{Kreyszig p.135 / Problem 6.}
Let $x\neq 0$ and $y\neq 0$.
\begin{enumerate}
    \item If $x\perp y$, show that $\{x, y\}$ is a
    linearly independent set.
    \item Extend the result to mutually
    orthogonal nonzero
    vectors $x_1,\ldots, x_m$.
\end{enumerate}

\begin{mybox}

    \begin{enumerate}
        \item If $x\perp y$ then for the equation
        $\lambda_1 x+\lambda_2 y=0$ we have,
        $$0=\inn{\lambda_1 x+\lambda_2 y}{y}=
        \inn{\lambda_1 x}{y}+\inn{\lambda_2 y}{y}=
        \lambda_2\|y\|^2\implies \lambda_2=0
        \hspace*{10mm}\text{and,}$$
        $$0=\inn{\lambda_1 x+\lambda_2 y}{x}=
        \inn{\lambda_1 x}{x}+\inn{\lambda_2 y}{x}=
        \lambda_1\|x\|^2\implies \lambda_1=0.$$
        Thus, $x$ and $y$ are linearly independent.

        \vspace*{3mm}
        \item Using above method, we see that if
        $\sum_{i=1}^m{\lambda_i x_i}=0$, then for any
        $i\in \overline{\{1,\ldots, m\}},$
        $$0=\inn{\sum_{i=1}^m{\lambda_i x_i}}{x_i}
        =\lambda_i\|x_i\|^2\implies \lambda_i=0.$$
        Thys the mutually orthogonal vectors $x_1,\ldots
        ,x_m$ are linearly independent when $x_i\neq 0$
        for all $i$.
    \end{enumerate}
\end{mybox}
 
\item \textbf{Kreyszig p.135 / Problem 10.}
Let $z_1$ and $z_2$ denote complex numbers.
Show that $\langle z_1,z_2\rangle = z_1\overline{z}_2$
defines an inner product, which yields
the usual metric on
the complex plane. Under what condition do we have
orthogonality?
\begin{mybox}

    $\inn{z_1}{z_2}=z_1\overline{z}_2$. We check that
    this definition satisfies the axioms:
    \begin{enumerate}
        \item $\inn{x+y}{z}=(x+y)\cdot\overline{z}
        =x\overline{z}+y\overline{z}=\inn{x}{z}+
        \inn{y}{z}$,

        \item $\inn{\alpha x}{y}=(\alpha x)\overline{y}
        =\alpha\inn{x}{y}$,

        \item $\inn{x}{y}=x\overline{y}=\overline
        {y\overline{x}}=\overline{\inn{y}{x}}$,

        \item $\inn{x}{x}=x\overline{x}=\|x\|^2$, where
        $\|\cdot\|$ denotes the usual metric on the complex
        plane. So, $\inn{x}{x}\geq 0$ and $\inn{x}{x}=0
        \iff x=0$.
    \end{enumerate}

    The inner product yields the usual metric by taking
    $\|\cdot\|=\inn{\cdot}{\cdot}^{1/2}$. For two complex
    numbers $z_1=x_1+iy_1$ and $z_2=x_2+iy_2$ where $x_j$
    and $y_j$ are real numbers, we have
    $\inn{z_1}{z_2}=(x_1+iy_1)(x_2-iy_2)=
    (x_1x_2+y_1y_2)+i(y_1x_2-x_1y_2)$. So we have
    orthogonality where the two equations
    $x_1x_2+y_1y_2=0$ and $y_1x_2-x_1y_2=0$ satisfy.
\end{mybox}
 
\item \textbf{Kreyszig p.141 / Problem 8.}
Show that in an inner product space, $x\perp y$ if and
only if $\|x+ay\|\geq\|x\|$ for all scalars $a$.
\begin{mybox}

    If $x\perp y$, we have $\|x+ay\|^2=\langle x+ay,
    x+ay\rangle=\langle x,x \rangle+\langle x,ay\rangle+
    \langle ay, x \rangle+ \langle ay, ay \rangle$. Thus,
    $$\|x+ay\|^2=\|x\|^2+2\mathfrak{Re}(\overline{a}
    \langle x, y
    \rangle)+a\overline{a}\|y\|^2\geq \|x\|^2.$$
    Taking square root, we get the required result.

    \vspace*{3mm}
    Now, if $\|x+ay\|\geq\|x\|$, we have
    $$\|x+ay\|^2=\|x\|^2+2\mathfrak{Re}(\overline{a}
    \langle x, y
    \rangle)+a\overline{a}\|y\|^2\geq \|x\|^2
    \implies 2\mathfrak{Re}(\overline{a}
    \langle x, y
    \rangle)+|a|^2\|y\|^2\geq 0.$$
    If $\inn{x}{y}=k\neq 0$, we can choose $a=-k$
    to get $-2k^2+k^2\|y\|^2\geq 0$. This gives rise to
    a contradiction when $\|y\|<\sqrt{2}$. Hence $\inn{x}
    {y}=0$.
\end{mybox}
 
\item
\textbf{Kreyszig p.141 / Problem 10. (Zero operator)}
Let $T: X\to X$ be a bounded linear operator on a complex
inner product space $X$. If $\langle Tx, x\rangle = 0$
for all $x\in X$, show that $T=0$.

Show that this does not hold in the case of a real inner
product space. Hint: Consider a rotation of the Euclidean
plane.
\begin{mybox}

    Let $x=\alpha p+q$ for some $p,q\in X$ and a
    scalar $\alpha$, then we get $$\inn{Tx}{x}=
    \inn{\alpha Tp+Tq}{\alpha p+q}=\alpha\overline{\alpha}
    \inn{Tp}{p}+
    \alpha\inn{Tp}{q}+
    \overline{\alpha}\inn{Tq}{p}+
    \inn{Tq}{q}=\alpha\inn{Tp}{q}+
    \overline{\alpha}\inn{Tq}{p}.$$
    Taking $\alpha=1$, we get $\inn{Tp}{q}+\inn{Tq}{p}=0$
    and taking $\alpha=i$, we get
    $\inn{Tp}{q}-\inn{Tq}{p}=0$. Solving these two equations
    we get $\inn{Tp}{q}=0$ and $\inn{Tq}{p}=0$. Since these
    are true for all $p$ and $q$ we take $q=Tp$ to get
    $\|Tp\|^2=\inn{Tp}{q}=0$. Thus $T$ must equal 0.

    \vspace*{3mm}
    If we take the linear operator on the real inner product
    space $\rl^2$ defined by the matrix
    $$T=\left[\begin{array}{cc}
        0 &-1\\
        1 &0
    \end{array}\right],$$
    we see that $\inn{T(x,y)}{(x,y)}=\inn{(-y,x)}{(x,y)}
    =-yx+xy=0.$ But $T\neq 0$.
\end{mybox}
 
\item \textbf{Kreyszig p.150 / Problem 7.}
Let $A$ and $B\supset A$ be nonempty subsets of an inner
product space $X$. Show that
\begin{enumerate}
    \item $A\subset A^{\perp\perp}$
    \item $B^{\perp}\subset A^{\perp}$
    \item $A^{\perp\perp\perp}=A^{\perp}$
\end{enumerate}

\begin{mybox}

    \begin{enumerate}
        \item If $x\in A$, then $\inn{x}{y} =0$ for all
        $y\in A^\perp$ and $x\perp A^\perp$. So $x\in
        A^{\perp\perp}.$

        \item  If $x\in B^\perp$, $x\perp B$. Since
        $B\supset A$, $x\in A^\perp$. Thus $B^\perp
        \subset A^\perp$. 

        \item  From (a), $A^\perp\subset A^{\perp\perp
        \perp}$. Also from (a) and (b), $A^{\perp\perp}
        \supset A\implies A^{\perp}\supset
        A^{\perp\perp\perp}$. Thus,
        $A^{\perp\perp\perp}=A^{\perp}$.
    \end{enumerate}
\end{mybox}


\item \textbf{Kreyszig p.150 / Problem 8.}
Show that the annihilator $M^{\perp}$ of a set
$M\neq\emptyset$ in an inner product space $X$ is a closed
subspace of $X$.
\begin{mybox}

    Let $\{x_n\}$ be a sequence in $M^\perp$ which
    converges to some $x\in X$. Then for
    all $n$ and all $y\in M$, $\inn{x_n}{y}=0$. By the
    continuity of inner product, we have,
    $$0=\lim_{n\to\infty}{\inn{x_n}{y}}=\inn{x}{y}.$$
    Thus $x\in M^\perp$ and $M^\perp$ is a closed subspace.
\end{mybox}

 
\item \textbf{Kreyszig p.150 / Problem 10.}
If $M\neq\emptyset$ is any subset of a Hilbert space $H$,
show that $M^{\perp\perp}$ is the smallest
closed subspace
of $H$ which contains $M$, that is, $M^{\perp\perp}$
is contained in any closed subspace $Y\subset H$ such
that $Y\supset M$.
\begin{mybox}

    Let $Y\supset M$ be a closed subspace of $H$. Then
    $Y=Y^{\perp\perp}$. By
    \textbf{Kreyszig p.150 / Problem 7}, $Y^\perp
    \subset M^\perp$ and $Y^{\perp\perp}\supset
    M^{\perp\perp}$. Thus, $M^{\perp\perp}$ is contained
    in the closed subspace $Y$ which contained $M$.
\end{mybox}
 
\item Let $Y$ be a closed subspace of a Hilbert space
$H$. Show that the projection operator $P: H \to Y$
is a bounded linear operator.
\begin{mybox}

    The projection operator $P$ maps a point $x\in H
    =Y\bigoplus Y^\perp$
    to the point $y\in Y$ such that $x=y+z$ for some
    $z\in Y^\perp$. We note that this representation is
    unique and hence the map $P$ is well-defined.

    \vspace*{3mm} If $p=y_1+z_1$ and $p_2=y_2+z_2$ are
    two points in $H$ written as the sum in
    $Y\bigoplus Y^\perp$ then
    $\alpha p_1+p_2=\alpha y_1+y_2+z_1+z_2$. Since $z_1
    +z_2\in Y^\perp$ and $\alpha y_1+y_2\in Y$ we see that
    $P(\alpha p_1+p_2)=\alpha y_1+y_2=\alpha P(p_1)+
    P(p_2)$. Hence, $P$ is linear.

    \vspace*{3mm} To show that $P$ is bounded we first
    note that if $x=y+z$ as above, then
    $$\|x\|^2=\inn{y+z}{y+z}=\|y\|^2+\|z\|^2.$$
    Then $\|P(x)\|^2=\|x\|^2-\|z\|^2\leq\|x\|^2$. Hence,
    $P$ is bounded.
\end{mybox}

\item For any subset $M \neq \emptyset$ of a Hilbert
space $H$, $\text{Span}(M)$ is dense in $H$
if and only if
$M^{\perp} = \{0\}$.
\begin{mybox}

    Assume that $\text{Span}(M)$ is dense in $H$ and let
    $x\in M^\perp$. There exists a sequence $(x_n)$ in
    $\text{Span}(M)$ such that $x_n\longrightarrow x$.
    We have $\inn{x_n}{x}=0$ for all $n$ and by continuity
    of inner product, $\lim_{n\to\infty}{\inn{x_n}{x}}
    =\inn{x}{x}=0\implies x=0.$ Hence $M^\perp=\{0\}$.

    \vspace*{3mm}
    Now, if $M^\perp=\{0\}$, then since $M\oplus M^\perp=
    H$, the subspace $\text{Span}(M)$ is dense in $H$.
\end{mybox}

\item \textbf{Kreyszig p.194 / Problem 6.}
Show that Theorem 3.8-1 defines an isometric
bijection $T: H\to H'$, $z \mapsto f_z = \langle\cdot, z
\rangle$ which is not linear but conjugate linear,
that is, $\alpha z +\beta v \mapsto \overline{\alpha}f_z
+\overline{\beta }f_v.$
\begin{mybox}

    We see that $T(\alpha z +\beta v)(x)=
    \inn{x}{\alpha z +\beta v}=\inn{x}{\alpha z}
    +\inn{x}{\beta v}=\overline{\alpha}\inn{x}{z}
    +\overline{\beta }\inn{x}{v}
    =\overline{\alpha}T(z)(x)
    +\overline{\beta }T(v)(x)$. Thus
    $\alpha z +\beta v \mapsto \overline{\alpha}f_z
    +\overline{\beta }f_v$ and the map is conjugate
    linear and isometric. The fact that $T$ is a bijection
    comes from the statement of Theorem 3.8-1.
\end{mybox}

\item \textbf{Kreyszig p.194 / Problem 7.}
Show that the dual space $H'$ of a Hilbert space $H$
is a Hilbert space with inner product $\langle\cdot,\cdot
\rangle_1$ defined by
$$\langle f_z,f_v\rangle_1=\overline{\langle z,v\rangle
}=\langle v,z\rangle,$$
where $f_z(x)=\langle x,z\rangle$, etc.
\begin{mybox}

    By Riesz's Theorem, we know that every linear bounded
    functional on $H$ can be represented as $f_z$
    for some $z\in H$. Also, $f_x$ is a linear and bounded
    functional for all $x\in H$. Here, $f_{x+y}(p)=\inn{p}
    {x+y}=f_x(p)+f_y(p)$ and $f_{\alpha x}(p)=\inn{p}
    {\alpha x}=\overline{\alpha}f_x(p)$.
    Now, we show that the
    given inner product satisfies the axioms:

    \begin{enumerate}
        \item $\inn{f_x+f_y}{f_z}_1=\inn{z}{x+y}=
        \inn{z}{x}+\inn{z}{y}=\inn{f_x}{f_z}_1+
        \inn{f_y}{f_z}_1$,

        \item $\inn{\alpha f_x}{f_y}_1=\inn{f_{
        \overline{\alpha} x}}{f_y}
        =\inn{y}{\overline{\alpha} x}=\alpha\inn{f_x}{f_y}
        _1$,

        \item $\inn{f_x}{f_y}_1=\inn{y}{x}=\overline{
        \inn{x}{y}}=\overline{\inn{f_y}{f_x}}_1$,

        \item $\inn{f_x}{f_x}_1=\inn{x}{x}\geq 0$
        and $\inn{x}{x}=0 \iff x=0$ so $\inn{f_x}{f_x}=0
        \iff f_x=0.$
    \end{enumerate}

    Since $x+y\mapsto f_x+f_y$, we also see that the map
    is isometric.
\end{mybox}

\end{enumerate}
\end{document}