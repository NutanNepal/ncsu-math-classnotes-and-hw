\documentclass[12pt]{article}
\usepackage[]{blindtext}
\usepackage[letterpaper, total{216mm, 279mm}]{}
\usepackage{amssymb,amsmath,amsfonts,verbatim}
\usepackage[breakable, skins]{tcolorbox}
\usepackage[parfill]{parskip}

\newtcolorbox{mybox}[2][]{
    arc=0mm, enhanced, frame hidden, breakable
}
\newcommand{\qed}{$\hfill\blacksquare$}

\setcounter{MaxMatrixCols}{10}

\pagestyle{empty}
\setlength{\topmargin}{-.65in}
\setlength{\textwidth}{190mm} 
\setlength{\textheight}{240mm}
\setlength{\oddsidemargin}{-15mm} 
\setlength{\evensidemargin}{-15mm}
\parindent=0pt


\newcommand{\mR}{\mathbb{R}}
\newcommand{\ds}{\displaystyle}
\newcommand{\al}{\alpha}

\begin{document}

{\large \textbf{MA 515 - Analysis I}}\hfill
\textbf{Homework 1} \\
{\bf Pack Pledge:} I have neither given nor received unauthorized aid on this
test or assignment.\\
{\bf Name:} Nutan Nepal\hfill \\
 \underline{}\hrulefill
\vspace{.1in}

\begin{enumerate}

\item Choose either $d_1$ or $d_2$ below and show that it is a metric on $\mathbb{R}^n$. 
 $$d_1(x,y) = max\{|x_i-y_i|\} \ \ \text{and} \ \ d_2(x,y) = \sum_{i=1}^n |x_i - y_i| \ \ \text{(taxicab metric)}.$$
\begin{mybox}

  \begin{enumerate}
    \item[i.]
      Since $d_2$ is a finite sum of positive numbers,
      $0\leq d_2(x,y)<\infty$.
    \item[ii.]
      $d_2(x,y)=d_2(y,x)$ since $|x_i-y_i|=|y_i-x_i|$
      for all $i$.
    \item[iii.]
      $d_2(x,x)=0$ since it is the sum of zeros.
    \item[iv.]
      Since we have $|x_i-z_i|=|(x_i-y_i)+(y_i-z_i)|
      \leq |x_i-y_i|+|y_i-z_i|$ (using triangle inequality
      for each $x_i, y_i, z_i \in \mathbf{R}$), we get
      $$d_2(x,z)=\sum_{i=1}^n |x_i - z_i|\leq
      \sum_{i=1}^n{\left(|x_i - y_i|+|y_i-z_i|\right)}
      =d_2(x,y)+d_2(y,z).$$
      Hence, $d_2$ also satisfies the triangle inequality
      for $x,y,z\in \mathbb{R}^n$.   
  \end{enumerate}
  Thus, $d_2$ is a metric in $\mathbb{R}^n$.\qed
\end{mybox}

 \item Let $C[0,1]$ be the space of continuous functions on $[0,1]$. \\
 Show that $\ds d(f,g) = \int_0^1|f(x) - g(x)| dx$ is a metric on $C[0,1]$
 \begin{mybox}

For all $f\in C[0,1]$, $f$ is bounded. Let $|f(x)|<M$ and
$|g(x)|<N$ for $x\in[0,1]$.
  \begin{enumerate}
    \item[i.]
      Then $0\leq\int_0^1{|f(x)-g(x)| dx}\leq
      \int_0^1{|f(x)|+|g(x)| dx}\leq (M+N)(1-0)<\infty$.
      Hence
      $0\leq d(f,g)<\infty$.
    \item[ii.]
      $d(f,g)=d(g,f)$ since $|f(x)-g(x)|=|g(x)-f(x)|$
      for all $x\in[0,1]$.
    \item[iii.]
      $d(f,f)=0$ since it is the integration of zero function.
    \item[iv.]
      For each $x\in [0,1]$ and $f,g,h\in C[0,1]$, we have
      $|f(x)-h(x)|=|(f(x)-g(x))+(g(x)-h(x))|
      \leq |f(x)-g(x)|+|g(x)-h(x)|$ (using triangle inequality
      for real numbers). Then
      $$d(f,h)=\int_{0}^1{ |f(x) - h(x)| dx}\leq
      \int_{0}^1{\left(|f(x) - g(x)|+|g(x)-h(x)|\right) dx}
      =d(f,g)+d(g,h).$$
      Hence, $d$ satisfies the triangle inequality
      for $f,g,h\in C[0,1]$.
  \end{enumerate}
  Thus, $d$ is a metric on $C[0,1] $.\qed
\end{mybox}
  
\item   Let $x = \{x_n\}_1^{\infty}$ be a sequence. 
\begin{enumerate} 
\item True or False: If $x \in l^p$ for some $1 \leq p < \infty$, then  $x_n \to 0$ as $n \to \infty$. Justify your answer.
\begin{mybox}

  True. If $x\in l^p$ for some $1\leq p<\infty$, then
  $\left(\sum_1^{\infty}{|x_i|^p}\right)^{1/p}<\infty$.
  Hence $\sum_i^{\infty}{|x_i|^p}$ is a convergent series
  and $|x_i|^p\to 0$ as $i\to \infty$ which implies that
  $x_i\to 0$ as $i\to\infty$.
\end{mybox}
\item True or False: If $x_n \to 0$ as $n \to \infty$, then $x_n \in l^p$, for some $1 \leq p < \infty$. Justify your answer.
\begin{mybox}

  False. The sequence given by $x= \{x_i=\frac{1}{\log (i+1)}\}_1^{\infty}
  \to 0$ as $i\to \infty$ but the sum $\sum_2^{\infty}{
    |x_i|^p
  }$ does not converge for any $1\leq p<\infty$.
  So, $x\notin l^p$ for any $1\leq p<\infty$.
\end{mybox}
 \end{enumerate}
 
 
 \item Let $a, b \geq 0$, and $p \geq 1$. Prove that 
 $$(a+b)^p \leq 2^{p-1} (a^p + b^p)$$
  Use the hints from class.

\begin{mybox}

  Let $f(x)=x^p$, $f:[0, \infty)\to \mathbf{R}$ and
    $p\geq 1$. Since $f$ is a $convex$ function, we have
    $$f(\alpha x +(1-\alpha)y)\leq\alpha f(x)+(1-\alpha)
    f(y)\hspace{8mm}\text{for } \alpha\in [0,1].$$
    Taking $\alpha=1/2$, we get
    \begin{align*}
        f\left(\frac{a}{2}+\frac{b}{2}\right)
        &\leq \frac{f(a)}{2}+\frac{f(b)}{2}\\
        \text{or, }\hspace{3mm}
        \frac{1}{2^p}f\left(a+b\right)
        &\leq \frac{1}{2}\left(f(a)+f(b)\right)\\
        \text{or, }\hspace{9.5mm} (a+b)^p
        &\leq 2^{p-1}\left(a^p+b^p\right).
    \end{align*}\qed
  \end{mybox}

\item For $p >1$, let $q$ be its conjugate, i.e. $\ds \frac{1}{p} + \frac{1}{q} = 1$. Prove the following inequality:
  $$\ds u \cdot v \leq \frac{1}{p} u^p + \frac{1}{q} v^q, \ \ \forall u, v \geq 0$$
  Use the hints from class. 
\begin{mybox}

  If either $u$ or $v$ equals $0$, then the inequality
    follows immediately. Suppose $u>0$, $v>0$ and let
    $f(x) = e^x$. Since $f$ is a $convex$ function,
    \begin{align*}
        u\cdot v &= \exp\left({\log{u}+\log{v}}\right)\\
        &= f\left(\frac{1}{p}\log{u^p} +\frac{1}{q}
        \log{v^q}\right)\\
        &\leq \frac{1}{p}f(\log{u^p})+\frac{1}{q}f(\log{v^q})\\
        &= \frac{u^p}{p}+\frac{v^q}{q}.
    \end{align*}
    \qed
\end{mybox}

    
  \item Prove Holder's Inequality for Sums. Use the hints from class. 
  
  \begin{mybox}
    
    \textbf{Holder's inequality:}
    Let $p,q \geq 1$ be conjugate exponents. Let
    $x=\{x_{i}\}_{1}^{\infty}\in l^{p}$
    and $y=\{y_{i}\}_{1}^{\infty}\in l^{q}$. Then
    \begin{enumerate}
      \item[a.]
            $xy=\{x_{i}y_{i}\}_{1}^{\infty} \in l^{1}$ and
      \item[b.]
            $\sum_{1}^{\infty}{|x_{i}y_{i}|}\leq
            (\sum_{1}^{\infty}{|x_{i}|^{p}})^{\frac{1}{p}}
            \cdot
            (\sum_{1}^{\infty}{|y_{i}|^{q}})^{\frac{1}{q}}.$
  \end{enumerate}
    Let $u_i=\frac{x_i}{\left(\sum_{1}^{\infty}
    {|x_i|^{p}}\right)^{1/p}}$ and 
    $v_i=\frac{y_i}{\left(\sum_{1}^{\infty}
    {|y_i|^{q}}\right)^{1/q}}$. Then by Young's inequality,
    \begin{align*}
        u_{i}\cdot v_{i} &= \frac{x_i}{
            \left(\sum_{1}^{\infty}
            {|x_i|^{p}}\right)^{1/p}
            }
            \cdot \frac{y_i}{\left(\sum_{1}^{\infty}
            {|y_i|^{q}}\right)^{1/q}}\\
            &\leq \frac{x_i^p}{
            p\sum_{1}^{\infty}
            {|x_i|^{p}}
            } +
            \frac{y_i^q}{q\sum_{1}^{\infty}
            {|y_i|^{q}}}
    \end{align*}
    Let $m = \left(\sum_{1}^{\infty}{|x_i|^{p}}\right)^{1/p}$
    and $n = \left(\sum_{1}^{\infty}{|y_i|^{q}}\right)^{1/q}$.
    Then from above we have
    \begin{align*}
        \sum_{1}^{\infty}{|x_{i}y_{i}|} = 
    mn\sum_{1}^{\infty}{|u_{i}v_{i}|} \leq
    mn\sum_1^\infty{\left|\frac{1}{pm^p}x_i^p +
    \frac{1}{qn^q}\cdot y_i^q\right|}&\leq mn\left(\frac{1}{pm^p}
    \cdot\sum_1^\infty\left|x_i^p\right|+\frac{1}{qn^q}
    \sum_1^\infty\left|y_i^q\right|\right)\\
    &=mn\left(\frac{1}{pm^p}\cdot m^p +\frac{1}{qn^q}\cdot n^q
    \right) = mn
    \end{align*}
    Hence $\sum_{1}^{\infty}{|x_{i}y_{i}|}\leq mn = 
    \left(\sum_{1}^{\infty}{|x_i|^{p}}\right)^{1/p}\cdot
    \left(\sum_{1}^{\infty}{|y_i|^{q}}\right)^{1/q}$ which proves
    (b). Since $0\leq\sum_{1}^{\infty}{|x_{i}y_{i}|}<\infty$, we also
    have (a) by definition.\qed
  \end{mybox}
  \item Prove Minkowski's Inequality for Sums.  Use the hints from class. 
  
  \begin{mybox}

    \textbf{Minkowski's inequality :} Let $p\geq 1$ and
    $x=\{x_{i}\}_{1}^{\infty} \in l^p$ and
    $y=\{y_{i}\}_{1}^{\infty} \in l^p$. Then

    \begin{enumerate}
        \item[a.]
              $x+y =
              \{x_{i} + y_{i}\}_{1}^{\infty} \in l^{p}$ and
        \item[b.]
        $\left(\sum_{1}^{\infty}{|x_{i} + y_{i}|^p}\right)^
        \frac{1}{p}\leq
        \left(\sum_{1}^{\infty}{|x_{i}|^{p}}\right)^{\frac{1}{p}}
        +
        \left(\sum_{1}^{\infty}{|y_{i}|^{p}}\right)^{\frac{1}{p}}.$
    \end{enumerate}
    First we show that $x+y\in l^p$ by showing that
    $$\left(\sum_{1}^{\infty}{|x_{i} + y_{i}|^p}\right)^{1/p}
    <\infty$$
    We have,
    $$\sum_{1}^{\infty}{|x_{i} + y_{i}|^p}\leq
    \sum_i^{\infty}{\left(|x_i|+|y_i|\right)^p}\leq
    2^{p-1}\left(\sum_i^{\infty}{|x_i|^p}+
    \sum_i^{\infty}{|y_i|^p}\right)<\infty.$$

    Now, since $x,y\in l^p$, $d_p(x,y)<\infty$.
    If $p=1$ then the Minkowski inequality follows from the
    triangle inequality of real numbers. Let $p>1$ then
    \begin{align}\sum_{1}^{\infty}{|x_{i} + y_{i}|^p}=
    \sum_{1}^{\infty}{|x_i+y_1||x_{i} + y_{i}|^{p-1}}
    &\leq\sum_{1}^{\infty}{\left(|x_i||x_{i} + y_{i}|^{p-1}
    +|y_i||x_{i} + y_{i}|^{p-1}\right)}\label{line1}\\
    &=\sum_{1}^{\infty}{\left(|x_i||x_{i} + y_{i}|^{p-1}\right)}
    +\sum_{1}^{\infty}{\left(|y_i||x_{i} + y_{i}|^{p-1}\right)}
    \label{line2}
    \end{align}
    Now let $q$ be the conjugate exponent of $p$, then we have
    $\frac{1}{p}+\frac{1}{q}=1 \iff p+q=pq \iff p=p(q-1)$. Then,
    at line
    \ref{line2}

    $$\left(\sum_{1}^{\infty}{|x_i+y_i|^{(p-1)q}}\right)^{1/q}
    =\left(\sum_{1}^{\infty}{|x_i+y_i|^{p}}\right)^{1/q}
    <\infty$$

    which shows that $\{|x_i+y_i|^{p-1}\}_i^{\infty}\in
    l^q$. Then by Holder's inequality,

    \begin{align}\sum_{1}^{\infty}{|x_i||x_{i} + y_{i}|^{p-1}}
      &\leq \left(\sum_{1}^{\infty}{|x_i|^p}\right)^{1/p}\cdot
      \left(\sum_{1}^{\infty}{|x_i+y_i|^{(p-1)q}}\right)^{1/q}\\
      &=\left(\sum_{1}^{\infty}{|x_i|^p}\right)^{1/p}\cdot
      \left(\sum_{1}^{\infty}{|x_i+y_i|^{p}}\right)^{1/q}
      \label{line4}
    \end{align}

    Using the results from line \ref{line2} and line \ref{line4} on line
    \ref{line1},
    \begin{align}
      \sum_{1}^{\infty}{|x_{i} + y_{i}|^p}
      &\leq \sum_{1}^{\infty}{\left(|x_i||x_{i} + y_{i}|^{p-1}\right)}
      +\sum_{1}^{\infty}{\left(|y_i||x_{i} + y_{i}|^{p-1}\right)}\\
      \text{or,}\hspace*{5mm}
      \sum_{1}^{\infty}{|x_{i} + y_{i}|^p}
      &\leq\left(\sum_{1}^{\infty}{|x_i+y_i|^{p}}\right)^{1/q}
      \cdot\left(\left(\sum_{1}^{\infty}{|x_i|^p}\right)^{1/p}
      +\left(\sum_{1}^{\infty}{|y_i|^p}\right)^{1/p}\right)
    \end{align}
    Dividing both sides by
    $\left(\sum_{1}^{\infty}{|x_i+y_i|^{p}}\right)^{1/q}$,
    we get the Minkowski's inequality (since $1-\frac{1}{q}=\frac{1}{p}$).
    
    \qed
  \end{mybox}
  \item For $1 \leq p < \infty$, let $\ds l^p = \{x = \{x_i\}_1^{\infty} \ | \ \sum_1^{\infty}|x_i|^p < \infty \}$. For any $x, y \in l^p$, define 
  $$d_p(x,y) = \Big( \sum_1^{\infty}|x_i - y_i|^p \Big)^{1/p}$$
  Prove that $(l^p, d_p)$ is a metric space.

  \begin{mybox}
    
      \begin{enumerate}
        \item[i.]
              Since $d_p(x,y)$ is the $pth$ root of a sum of positive
              numbers, $d_p\geq 0$. Also from Minkowski inequality (a.),
              we have $d_p<\infty$.
              
        \item[ii.]
              $d_p(x,y)=d_p(y,x)$ since $|x_i-y_i| = |y_i-x_i|$
              for all $i$.
        \item[iii.] 
              $d_p(x,x)=0$ since $|x_i-x_i|=0$ for all $i$.
        \item[iv.]
              The triangle inequality for $d_p$ follows from the Minkowski
              inequality (b.)
              \begin{align*}
                \left(\sum_{1}^{\infty}{|x_{i} - z_{i}|^p}\right)^
              \frac{1}{p}&\leq
              \left(\sum_{1}^{\infty}{|x_{i}-y_i|^{p}}\right)^{\frac{1}{p}}
              +
              \left(\sum_{1}^{\infty}{|y_{i}-z_i|^{p}}\right)^{\frac{1}{p}}\\
              \text{or, }\ \ d_p(x,z)&\leq d_p(x,y)+d_p(y,z)
              \end{align*}\qed
    \end{enumerate}
    
  \end{mybox}
  
  \item Prove Jensen's Inequality for Sums. Use the hints from class. 
  \begin{mybox}
    
    $$\left(\sum_{i=1}^{\infty}{|x_i|^{p_2}}\right)^{1/{p_2}}
    \leq\left(\sum_{i=1}^{\infty}{|x_i|^{p_1}}\right)^
    {1/{p_1}}\hspace*{10mm}\forall 1\leq p_1<p_2<\infty$$
  
    
    Let $|y_i|=|x_i|^{p_1}$. Then we need to show that
    $\left(\sum_{i=1}^{\infty}{|y_i|^{p_2/p_1}}\right)
    ^{{p_1/p_2}}
    \leq\sum_{i=1}^{\infty}{|y_i|}$. First we show that
    this is true for a finite sequence $\{x_i\}_1^{n}$
    using induction on $n$. Then we take the limit as
    $n\to \infty$ to prove Jensen's inequality.

    \setlength{\parskip}{3mm}
    When $n=1$, $(|y_1|^{p_2/p_1})^{p_1/p_2}=y_1$.
    (True)

    \setlength{\parskip}{3mm}
    Let $H(k):\left(\sum_{i=1}^{k}{|y_i|^{p_2/p_1}}\right)^{p_1/{p_2}}
    \leq\sum_{i=1}^{k}{|y_i|}$ be true for some integer
    $k>1$. Then
    \begin{align*}
      \left(\sum_{i=1}^{k+1}{|y_i|^{p_2/p_1}}\right)^{p_1/{p_2}}
      &=\left(\sum_{i=1}^{k}{|y_i|^{p_2/p_1}}
      +|y_{k+1}|^{p_2/p_1}\right)^{p_1/{p_2}}&\\
      &\leq \left(\sum_{i=1}^{k}{|y_i|^{p_2/p_1}}\right)
      ^{p_1/p_2}
      +\left(|y_{k+1}|^{p_2/p_1}\right)^{p_1/{p_2}}
      &\text{[by Minkowski inequality]}\\
      &\leq \sum_{i=1}^{k}{|y_i|}+|y_{k+1}|
      &\text{[by induction hypothesis]}\\
      &=\sum_{i=1}^{k+1}{|y_i|}
    \end{align*}
    Hence $H(k)\implies H(k+1)$ which proves that $H(n)$ is
    true for all $n\in \mathbb{Z}$. Taking the limit
    as $n\to\infty$ we get the required Jensen's inequality.
  \qed
\end{mybox}
  \item Show that $l^1 \subset l^2$ without using Jensen's inequality. Then show that inclusion is strict, i.e., find an element in $l^2$ that is not in $l^1$. 
  \begin{mybox}

    Let $x\in l^1$ then $0\leq\sum_1^\infty{|x_1|}<\infty$
    which implies that the sequence $x$ converges to $0$. Let $N\in\mathbb{Z}$
    such that $x_i<1$ for all $i>N$. Then for $i>N$, we have
    $|x_i|^2<|x_i|$. Hence,
    $$0\leq \sum_{i=1}^{\infty}{|x_i|^2}
    \leq \sum_{i=1}^{N}{|x_i|^2}+
    \sum_{i=N+1}^{\infty}{|x_i|^2}<\sum_{i=1}^{N}{|x_i|^2}+
    \sum_{i=N+1}^{\infty}{|x_i|}<\infty.$$
    So, $l^1\subset l^2$.

    \setlength{\parskip}{3mm}
    The harmonic series given by the sequence
    $x=\{x_i=\frac{1}{i}\}_1^{\infty}$ does not converge.
    However $$\sum_{i=1}^{\infty}{\frac{1}{i^2}}
    =\frac{\pi^2}{6}<\infty.$$
    Here we see that $x\notin l^1$ but $x\in l^2$. Hence,
    the inclusion is strict.
    \qed
  \end{mybox}
 
  \end{enumerate}

\end{document}
