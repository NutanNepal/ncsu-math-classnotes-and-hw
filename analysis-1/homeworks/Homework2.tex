\documentclass[12pt]{article}
\usepackage[]{blindtext}
\usepackage[letterpaper, total{216mm, 279mm}]{}
\usepackage{amssymb,amsmath,amsfonts,verbatim}
\usepackage[breakable, skins]{tcolorbox}
\usepackage[parfill]{parskip}
\usepackage[english]{babel}
\usepackage{mathtools, amsthm}
\usepackage{amsfonts}
\usepackage{amssymb}
\usepackage{mathrsfs}
\usepackage{verbatim}

\newtheorem{notes}{Notes}[section]
\newtheorem{prob}[notes]{Problems}
\newtheorem{thm}{Theorem}
\newtheorem{cor}[thm]{Corollary}
\newtheorem{lem}[thm]{Lemma}
\newtheorem{defn}[notes]{Definition}
\newtheorem{rem}[notes]{Remark}
\newtheorem{prop}[thm]{Proposition}

\newcommand{\rl}{\mathbb{R}}
\newcommand{\id}{\text{id}}
\newcommand{\dprime}{{\prime\prime}}
\newcommand{\xprime}{X^\prime}
\newtcolorbox{mybox}[2][]{
    arc=0mm, enhanced, frame hidden, breakable
}
\newcommand{\qedbox}{$\hfill\blacksquare$}
\newcommand{\mR}{\mathbb{R}}
\newcommand{\mQ}{\mathbb{Q}}
\newcommand{\ds}{\displaystyle}
\newcommand{\al}{\alpha}

\setcounter{MaxMatrixCols}{10}

\setlength{\topmargin}{-.65in}
\setlength{\textwidth}{190mm} 
\setlength{\textheight}{240mm}
\setlength{\oddsidemargin}{-15mm} 
\setlength{\evensidemargin}{-15mm}
\parindent=0pt

\title{Analysis I \\
\large Homework 2
}
\author{Nutan Nepal}
\newcommand{\packpledge}{
    $\text{{\bf Pack Pledge:} I have neither given nor
    received unauthorized aid on this
    test or assignment.}$}

\begin{document}
\maketitle
\packpledge\\
\makebox[\linewidth]{\rule{200mm}{1pt}}
\vspace{1mm}

\begin{enumerate}

\item Let $(X, \rho)$ and $(Y, \sigma)$ be metric spaces,
    and let $f: (X, \rho) \to (Y, \sigma)$ be a map
    such that $f^{-1}(V)$ is open in $X$, for all
    $V$ open in $Y$. Show that $f$ is continuous on $X$. 

\begin{mybox}

    Let $V$ be the open set around the image $f(x)$ of a
    point $x\in X$. Then there exists an open set $f^{-1}
    (V)\in X$ such that $x\in f^{-1}(V)$.
    Let $V$ be the open ball
    $B_{x,\varepsilon}(f(x))$ around the point $f(x)$
    for some given $\varepsilon>0$.
    Then $f^{-1}(B_{x,\varepsilon}(f(x)))$ is also open in
    $X$. Since $x$ is in the open set
    $f^{-1}(B_{x,\varepsilon}(f(x)))$, we can find a
    $\delta>0$ such that $x\in B_\delta(x)\subset
    f^{-1}(B_{x,\varepsilon}(f(x)))$. This means that
    for all $z\in X$ such that
    $d(z,x)<\delta$ we have $d(f(z),f(x))<\varepsilon$.
    Hence $f$ is continuous a4 $x$. Since we can
    do this at every point $x\in X$, we see that $f$ is
    continuous on $X$.
\end{mybox}


\item ({\bf Continuous mapping}) Show that a
    mapping $T: X\to Y$ is continuous
    if and only if the inverse image of any closed
    set $M\subset Y$ is a closed
    set in $X$.
\begin{mybox}

    We first note that for any set $A\subset Y$,
    $f^{-1}(A)=\{x\in X:\ f(x)\in A\}$ and
    $f^{-1}(Y\char`\\A)=\{x\in X:\ f(x) \notin A\}$.
    Then clearly, $f^{-1}(A)$ and $f^{-1}(Y\char`\\A)$
    are disjoint sets of $X$. Particularly,
    $$f^{-1}(Y\char`\\A)=X\char`\\f^{-1}(A).$$

    \vspace*{3mm}
    ($\Longleftarrow$)
    Let $V$ be a closed set of $Y$. Then there exists
    a closed set $f^{-1}(V)$ in  $X$. Since all open
    sets can be written as a complement of closed set,
    we see that for any open set $Y\char`\\V$ in $Y$
    there exists an open set $X\char`\\f^{-1}(V)$ in
    $X$. Since $X\char`\\f^{-1}(V)=f^{-1}(Y\char`\\A)$,
    the preimage of any open set of $Y$ is open in
    $X$. Hence $f$ is continuous.

    \vspace*{3mm}
    ($\Longrightarrow$)

\end{mybox}
 
 
\item Assume that $f : (\mR^2, d_1 =  \
    \text{Euclidean metric}) \to \mR$ is continuous at
    $x \in \mR^2$. Show that
    $f : (\mR^2, d_2 =  \ \text{taxicab metric})
    \to \mR$ is also continuous at $x$.
\begin{mybox}

    Since $f : (\mR^2, d_1 =  \
    \text{Euclidean metric}) \to \mR$ is continuous,
    we know that $f^{-1}(V)$ is open in $\rl^2$, for all
    $V$ open in $\rl$. Our proof will be complete if we
    can show that every open set $U$ in $(\rl^2,d_1)$
    is also
    open with respect to the taxicab metric.

    \vspace*{3mm}
    Let $V$ be open in $(\mR^2, d_1)$. Then for
    every point $(p,q)\in X$ there exists an open ball
    $B_\delta((p,q))\subset V$ containing $(p,q)$.
    $$B_\delta((p,q))=\{(x,y)\in\mR^2:
    \ (x-p)^2+(y-q)^2<\delta^2\}$$
\end{mybox}

\item Show that the discrete metric space $(X,d)$ is
    separable iff $X$ is countable. 
\begin{mybox}

    If $X$ is countable, then $X$ is the countable
    dense subset in $X$. So, it is separable.

    \vspace*{3mm}
    Now, we show that if $(X,d)$ is separable, then
    it is countable. Let $Y\subset X$ be
    the countable dense subset of $X$.
    Then every open set of $X$ must contain a point
    from $Y$. We note that all the singleton sets of
    $X$ are open in $X$ and so Y must contain all the
    points from the singleton sets. Hence we get
    $X\subset Y\implies X=Y$. So, $X$ is countable.
\end{mybox}
 
\item Show that $l^p$, with $1 \leq p <
    \infty$ is separable.
\begin{mybox}

    We will show that the set $M$ of sequences in $l^p$
    with finite non-zero rational terms is
    the countable dense
    subset of $l^p$ with $1\leq p<\infty$. That is
    $M$ has all the sequences $x$ of the form
    $$(x_1,x_2,\ldots,x_n,0,0,\ldots)$$
    with each $x_i\in\mQ$.
    First, we show
    that $M$ is countable. Note that for a fixed $n$, the
    set of sequences in $l^p$ with rational terms
    and all but the first $n$
    terms zero is countable. Then $M$ is the countable
    union of countable sets and, hence, is countable.

    \vspace*{3mm}
    Now, we show that the set $M$ is dense in $l^p$
    with $1 \leq p <\infty$. Let $x=\{x_i\}_1^\infty
    \in l^p$. Then $\sum_1^\infty{|x_i|^p}$ is
    convergent and so for every
    $\varepsilon>0$, there exists $N\in\mathbb{N}$ such
    that,
    $$\sum_{N+1}^\infty{|x_i|^p}<\varepsilon^p/2.$$
    Then, for the first $N$ terms, since rational numbers
    are dense in $\mR$, we can choose rational numbers
    $y_i$ such that $|x_i-y_i|^p<\varepsilon^p/2N$.
    So $(y_1,\ldots,y_N,0,0,\ldots)$ is a point in $M$ and
    we see that
    $$d(x,y)=\left(\sum_{i=1}^\infty{|x_i-y_i|^p}\right)
    ^{1/p}=\left(\sum_{i=1}^N{|x_i-y_i|^p}
    +\sum_{i=N+1}^\infty{|x_i-y_i|^p}\right)
    ^{1/p}<(\varepsilon^p/2+\varepsilon^p/2)^{1/p}
    =\varepsilon$$
    So, for every $\varepsilon>0$, we can find a point
    of $M$ in the $\varepsilon$-neighborhood of every
    point $x\in l^p$. Hence $l^p$ with
    $1 \leq p <\infty$ is separable.
\end{mybox}
 
\item Show that $l^{\infty}$ is not separable.
\begin{mybox}

    Let $x=\{x_i\}_{i=0}^\infty$ be the sequence
    of zeros and ones. Then since the sequence
    is bounded, it is in $l^\infty$. Now we
    will show that the set of such sequences $x$
    are uncountably many and have disjoint
    open neighborhoods for some radius.

    \vspace*{3mm}
    For each sequence $x$ we associate a real
    number $y$ whose binary
    representation is
    $$\sum_1^{\infty}{\frac{x_i}{2^i}}.$$
    Then for each $y\in [0,1]$, there exists
    a unique sequence of zeros and ones as each
    $y$ has a unique binary representation. Since
    there are uncountably many $y$, the sequences
    associated with them are also uncountable.
    The metric on $l^\infty$ given by
    $$d(x,y)=\sup_{i\in \mathbb{N}}{|x_i-y_i|}$$
    implies that any two distict binary sequences
    $\{x_i\}_{i=0}^\infty$ must be 1 distance
    apart. Then we can take $r=1/2$ to get the
    disjoint neighborhoods in $l^\infty$
    associated with each sequence
    $\{x_i\}_{i=0}^\infty$.

    \vspace*{3mm}
    Now, if $M$ is any dense set in $l^\infty$,
    then every open set in $l^\infty$ must contain
    a point of $M$. So, for each disjoint
    open neighborhood constructed above, $M$
    contains a point in the neighborhood. Hence,
    any dense set $M$ is uncountable in $l^\infty$
    and $l^\infty$ is not separable.
\end{mybox}
 
\item Let $\{x_n\}_{n \geq 1}$ be a sequence in a m.s.
    $(X, d)$ which converges to $x$. Show that
    $\{x_n\}_{n \geq 1}$ is a bounded sequence.
    Then let $\{y_n\}_{n \geq 1}$ be a sequence in
    $(X, d)$ which converges to $y$. Show that
    $\lim_{n \to \infty} d(x_n, y_n) = d(x,y)$. 
\begin{mybox}

    Since the sequence $\{x_n\}_{n \geq 1}$ is converges
    to x,
    for every $\varepsilon>0$, we can find a $N$ such that
    $d(x_n,x)<\varepsilon$ for all $n\geq N$. If we take
    $\varepsilon=1$, we see $x_n\leq|x|+1$ for all
    $n\geq N$. Let $M=\max\{x_1,\ldots,x_{N-1}\}$. Then
    $$x_n\leq \max\{M,|x|+1\}\ \ \text{for all }n\in
    \mathbb{Z}.$$
    Hence, $\{x_n\}_{n \geq 1}$ is bounded.

    \vspace*{3mm}
    If $\{x_n\}_{n \geq 1}$ converges to $x$ and
    $\{y_n\}_{n \geq 1}$ converges to $y$, then for every
    $\varepsilon>0$, there exists $N_1$ and $N_2$ such
    that $d(x,x_n)<\varepsilon/2$ for $n>N_1$ and
    $d(y,y_n)<\varepsilon/2$ for $n>N_2$. Then
    for $n>\max\{N_1, N_2\}$,
    $$d(x_n,y_n)\leq d(x,x_n)+d(x,y_n)\leq
    d(x,x_n)+d(y,y_n)+d(x,y)\leq\varepsilon+d(x,y).$$
    As $n\to\infty$, $d(x_n,y_n)\to d(x,y)$.
\end{mybox}
 
\item Show that any nonempty set $A\subset (X, d)$
    is open if and only if it is a
    union of open balls.
\begin{mybox}

    If $A$ is open in $X$ then for all $x\in A$ we exists
    an open ball $B_{\delta_x}(x)$ such that
    $x\in B_{\delta_x}(x)\subset A$.
    Let $$B=\bigcup_{x\in A}{B_{\delta_x}(x)}.$$
    Clearly, $B\subset A$ since each $B_{\delta_x}(x)$
    is contained in $A$. Also, since $B$ contains all the
    points of $A$, we have $A=B$. So, $A$ is a union
    of open balls.
    
    \vspace*{3mm}
    Now, if $A=\bigcup{B_\alpha}$ is a union of open
    balls $B_\alpha$ then
    for each $x\in A$, we have $x \in B_\alpha$ for some
    $B_\alpha$. That means that for every $x\in A$ we have
    some open ball $B_\alpha$ such that
    $x\in B_\alpha\subset A$. Hence $A$ is open in $X$.
\end{mybox}


\item Let $(X, \rho)$ be a metric space, $E \subset X$, and $x \in X$. Prove that the following are equivalent:
\begin{enumerate}
\item $x \in \overline{E}$
\item $B(x,r) \cap E \neq \emptyset$, $\forall r > 0$
\item $\exists \ \{x_n\} \in E$ s.t. $x_n \to x$
\end{enumerate}  
\begin{mybox}

    (a) $\implies$ (b).
    Let $x\in \overline{E}=X\char`\\(X\char`\\E)^\circ$.
    Then $x\notin (X\char`\\E)^\circ$. Negating this
    statement, we see that for all $r>0$,
    $B(x,r)\cap E\neq \emptyset$.
    
    \vspace*{3mm}
    (b) $\implies$ (c).
    We first define the radii $r_n=1/n$ of the open balls
    around the point $x\in X$. Since
    $B(x,r) \cap E \neq \emptyset$, $\forall r > 0$,
    we can take a sequences of points 
    $\{x_i\}_1^\infty$ in $E$ such that
    $x_i\in B(x,r_i)$ for each $i$.
    Then this is the sequence in $E$
    which converges to the point $x$.

    \vspace*{3mm}
    (c) $\implies$ (a). Since $x_n\to x$, for
    every $\varepsilon>0$
    we can find a $N$ such that $d(x,x_n)<\varepsilon$
    all $n>N$. That is, every open ball around
    $x$ contains a point of $E$. So $x\notin 
    (X\char`\\E)^\circ\implies x\in
    X\char`\\(X\char`\\E)^\circ$. Thus $x$ is in the
    closure $\overline{E}$.

\end{mybox}

 
\item If $d_1$ and $d_2$ are metrics on the same set
    $X$ and there are positive
    numbers $a$ and $b$ such that for all $x$, $y\in X$,
    $$ad_1(x, y)\leq d_2(x,y)\leq bd_1(x,y),$$
    show that the Cauchy sequences in $(X,d_1)$
    and $(X,d_2)$ are the same.
\begin{mybox}

    Let $\{x_n\}_1^\infty$ be a Cauchy sequence
    in $(X,d_1)$, then for every $\varepsilon'>0$,
    we can find an integer $N$ such that
    $d_1(x_m,x_n)<\varepsilon'$ for all $m$, $n>N$.
    Then taking $\varepsilon=\varepsilon'/b$ we
    have $d_2(x_m,x_n)\leq\varepsilon$. Hence
    the sequence $\{x_n\}_1^\infty$ is Cauchy in
    $(X,d_2)$.

    \vspace*{3mm}
    Now, let $\{y_n\}_1^\infty$ be a Cauchy sequence
    in $(X,d_2)$, then for every $\varepsilon'>0$,
    we can find an integer $N$ such that
    $d_2(y_m,y_n)<\varepsilon'$ for all $m$, $n>N$.
    Then taking $\varepsilon=a\varepsilon'$ we
    have $ad_1(y_m,y_n)\leq a\varepsilon
    \implies d_1(y_m,y_n)\leq \varepsilon$. Hence
    the sequence $\{y_n\}_1^\infty$ is Cauchy in
    $(X,d_1)$.
\end{mybox}
 
\item Show that $l^p$, with $1 \leq p < \infty$ is
    complete.
\begin{mybox}

    Let $\{\{x_i^n\}_{i=1}^\infty\}_{n=1}^\infty$
    be a Cauchy sequence in
    the space $l^p$ where for each $n$,
    $\{x_i^n\}_{i=1}^\infty$ is a convergent
    sequence in $\rl$. Then for every $\varepsilon>0$,
    there exists an $N$ such that
    $$d(x_i^m,x_i^n)
    =\left(\sum_{i=1}^\infty{
        |x_i^m-x_i^n|^p
    }\right)^{1/p}<\varepsilon$$
    for all $m$, $n>N$. Then for each fixed $i$,
    the term
    $|x_i^m-x_i^n|^p<\varepsilon^p\implies
    |x_i^m-x_i^n|<\varepsilon$. So 
    $\{x_i^j\}_1^\infty$ is a Cauchy
    sequence of real
    numbers for each fixed $i$ and it converges
    to some number that we can call $x_i$.
    We define $x=\{x_i\}_{i=1}^\infty$ and show that
    this is the limit of our sequence
    $\{\{x_i^n\}_{i=1}^\infty\}_{n=1}^\infty$ in
    $l^p$.

    From above, we have
    $$\sum_{i=1}^k{
        |x_i^m-x_i^n|^p
    }<\varepsilon^p$$ for all $m$, $n>N$. Then as 
    $n\to \infty$, we have (by definition) for $m>N$
    $$\sum_{i=1}^k{
        |x_i^m-x_i|^p
    }\leq\varepsilon^p.$$
    Then as $k\to \infty$, we have
    $$\sum_{i=1}^\infty{
        |x_i^m-x_i^n|^p
    }\leq\varepsilon^p$$

\end{mybox}
 
\item Prove that $(\mR, d(x,y) = |x-y|)$ is complete.\\ 
    Hint: Follow the steps provided in class.
    You can use Bolzano-Weierstrass Theorem without
    proving it.
\begin{mybox}

    First, we show that every Cauchy sequence in 
    $(\rl,d)$ is bounded.
    Let $\{x_n\}_1^\infty$ be a Cauchy sequence
    in $(\rl,d)$, then for every $\varepsilon>0$,
    we can find an integer $N$ such that
    $d(x_m,x_n)=|x_m-x_n|<\varepsilon$ 
    for all $m$, $n>N$. Then when $m=N+1$
    and $\varepsilon=1$, we see
    that $|x_m|-|x_{N+1}|\leq|x_m-x_{N+1}|
    <\varepsilon$. So $|x_m|\leq |x_{N+1}| +1$
    for all $m>N$. Hence the sequence
    $\{x_n\}_1^\infty$ is bounded by $M$ where
    $M=\max\{|x_1|,\ldots,|x_N|, |x_{N+1}|+1\}$.
    Then, since $\{x_n\}_1^\infty$ is bounded,
    by Bolzano-Weierstrass Theorem, we know that
    $\{x_n\}_1^\infty$ has a convergent
    subsequence $\{x_{n_i}\}_{i=1}^\infty$.
    Let $\{x_{n_i}\}_{i=1}^\infty\to x$ in $\rl$.

    \vspace*{3mm}
    Now, we show that our sequence
    $\{x_n\}_1^\infty$ itself converges to the
    point $x\in \rl$. In the Cauchy sequence
    $\{x_n\}_1^\infty$, we have, for all
    $\varepsilon>0$, there exists
    $N\in \mathbb{N}$ such that
    $|x_m-x_n|<\varepsilon/2$ for all $m$, $n>N$.
    Similarly, in the convergent subsequence
    $\{x_{n_i}\}_{i=1}^\infty$, there exists
    $N'\in \mathbb{N}$ such that
    $|x_{n_i}-x|<\varepsilon/2$ for all
    $i>N'$. Then for all $\varepsilon>0$, there
    exists $N,N'\in \mathbb{N}$ such that,
    for all $n>N$ and $i>N'$ with $n_i>N$, we have
    $$|x_n-x|\leq |x_n-x_{n_i}|+|x_{n_i}-x|
    <\varepsilon/2+\varepsilon/2=\varepsilon.$$
    Hence $\{x_{n}\}_{n=1}^\infty$ converges in
    $(\rl,d)$ and $\rl$ is complete.
    
\end{mybox}
 
\item Prove that $(\mQ, d(x,y) = |x-y|)$ is incomplete.
\begin{mybox}

    We take the sequence $\{x_n\}_1^\infty$ of rational
    numbers given by
    $x_n=\left(1+\frac{1}{n}\right)^n$. Note that this
    is a convergent sequence in $\rl$ with the same
    metric and hence it is
    Cauchy in $\mQ$ too. However, it does not converge
    to any number in $\mQ$. So, $\mQ$ is not complete.
\end{mybox}
 
\item Prove that $\Big(\text{C}[-1,1], d(f,g) =
    \int_{-1}^1|f(t) - g(t)| \ dt\Big)$ is incomplete. \\
    Hint: Follow the steps provided in class.
\begin{mybox}

    We take the sequence $\{f_n\}_1^\infty$
    of piecewise defined functions in $\text{C}[-1,1]$
    such that
    $$f_n(x)=
    \begin{dcases}
        1,   & x>\frac{1}{n}\\
        nx,  & -\frac{1}{n}\leq x\leq \frac{1}{n}\\
        -1,  & x<\frac{1}{n}
    \end{dcases}$$

    We  first observe that
    $d(f_m,f_n)=2 \times$ area of the
    triangle with base $\left|\frac{1}{m}-
    \frac{1}{n}\right|$ and height 1
    =$\left|1/m-1/n\right|$. Then,
    the sequence is Cauchy
    since for every $\varepsilon>0$ we have
    an integer $N>1/\varepsilon$ such that
    for all $m$, $n>N$,
    $d(f_m,f_n)<\varepsilon$.

    \vspace*{3mm}
    Now, for every $g\in C[0,1]$, we have
    $$d(f_n,g)=\int_{-1}^1{|f_n(t) - g(t)|} dt
    = \int_{-1}^{-\frac{1}{n}}{|-1-g(t)|dt}
    + \int_{-\frac{1}{n}}^{\frac{1}{n}}
    {|f_n(t)-g(t)|dt}
    + \int_{{\frac{1}{n}}}^1{|1-g(t)|dt}$$
    Since $d\geq 0$ for all $g\in C[0,1]$,
    $d(f_n,g)\to 0$ implies that each
    integral on the right should also approach
    0. Then we should have
    $$g(t)=-1\ \ \text{ for }t\in [-1,0)
    \ \ \text{ and }\ \
    g(t)=1\ \ \text{ for }t\in (0,1].$$
    But then $g$ cannot be continuous so we have
    a contradiction. So, the Cauchy sequence
    $\{f_n\}_1^\infty$ does not converge in
    $C[0,1]$ and the space is not complete.
\end{mybox}
 
\item Determine whether or not the discrete metric
    space is complete. Justify your answer.
\begin{mybox}

    Let $\{x_i\}_{i=0}^\infty$ be a Cauchy
    sequence in the discrete metric space. Then
    for every $\varepsilon>0$ there exists an $N$
    such that for all $m$, $n>N$, we have
    $$d(x_m,x_n)<\varepsilon.$$
    If we take $\varepsilon<1$ then we see that
    $x_m=x_n=x$ for all $m$, $n>N$ for some $N$.
    So the sequence $\{x_i\}_{i=0}^\infty$
    converges to $x$ and the space is
    complete.
\end{mybox}
 
\item Prove the Completion of a Metric Space Theorem. 
\begin{mybox}

    \begin{thm}[Completion of a Metric Space]

    For a metric space $X = (X, d)$ there
    exists a complete metric space $X^\prime
    =(X^\prime, d^\prime)$
    which has a subspace $W$ that is
    isometric with $X$ and is dense in $X^\prime$.
    This space $X^\prime$ is unique except for
    isometries, that is, if $\tilde{X}$ is
    any complete metric space having a dense
    subspace $\tilde{W}$ isometric with $X$,
    then $X^\prime$ and
    $\tilde{X}$ are isometric.
    \end{thm}

    \begin{proof}
        We prove the theorem in the following steps.

        \vspace*{2mm}
        \begin{enumerate}
            \item We first define a relation $\sim$ on Cauchy
            sequences of $X$ and show that it is a
            well-defined equivalence relation. For 
            the Cauchy sequences
            $x= \{x_i\}_1^\infty$,
            $y= \{y_i\}_1^\infty$ in $X$, we
            say that $x \sim y$ if
            $$\lim_{n\to\infty}{d(x_n,y_n)}=0.$$
            Clearly, this relation is symmetric and
            reflexive. For transitivity we see that
            if $x\sim y$ and $y\sim z$ then
            $$0\leq\lim_{n\to\infty}{d(x_n,z_n)}\leq
            \lim_{n\to\infty}{d(x_n,z_n) +d(y_n,z_n)}
            =0.$$
            Hence $\sim$ is an equivalence relation on the
            Cauchy sequences of $X$. Now, let $\xprime$ be the
            set of all equivalence classes of Cauchy
            sequences on $X$ and define the function
            $d^\prime:\xprime\to \rl$ by 
            $$d^\prime(x^\prime,y^\prime)=
            \lim_{n\to\infty}{d(x_n,y_n)}$$
            where $x^\prime$ and $y^\prime$ are the equivalence
            classes of $x$ and $y$ respectively. We show that this
            limit exists and the function $d^\prime$ is well
            defined. We have,
            \begin{align*}
                d(x_n,y_n)&\leq d(x_n,x_m)+d(x_m,y_m)
                +d(y_n,y_m)\\
                \text{or, }\ |d(x_n,y_n)-d(x_m,y_m)|
                &\leq d(x_n,x_m)+d(y_n,y_m).
            \end{align*}
            Taking limit as $m$, $n$ go to $\infty$
            on both sides we obtain,
            $$\lim_{n\to\infty}{|d(x_n,y_n)-d(x_m,y_m)|}
            \leq \lim_{n\to\infty}{\left[d(x_n,x_m)+
            d(y_n,y_m)\right]}=0$$
            Hence the limit $d^\prime(x^\prime,y^\prime)=
            \lim_{n\to\infty}{d(x_n,y_n)}$ exists.
    
            \vspace*{3mm}
            Now, if $x\sim x^\prime$ and $y\sim y^\prime$, then
            $$d(x_n,y_n)\leq d(x_n,x^\prime_n)+
            d(x_n^\prime,y_n^\prime)
            +d(y_n,y_n^\prime)$$
            And as before,
            $$0\leq \lim_{n\to\infty}{|d(x_n,y_n)-
            d(x_n^\prime,y_n^\prime)|}
            \leq \lim_{n\to\infty}{\left[d(x_n,x^\prime_n)+
            (y_n,y_n^\prime)\right]}=0$$
            which implies that $\lim_{n\to\infty}{d(x_n,y_n)}
            =\lim_{n\to\infty}{d(x_n^\prime,y_n^\prime)}$.
            Hence $d^\prime$ is a well-defined function on
            $\xprime$.
    
            \vspace*{3mm}
            We now show that $d^\prime$ is a metric on
            $\xprime$. Clearly $0\leq d^\prime<\infty$ since
            the limit exists and $d^\prime(x^\prime,
            x^\prime)=0$. Furthermore,
            $$d^\prime(x^\prime,
            y^\prime)=0\implies x\sim y\implies x^\prime
            \sim y^\prime.$$
            And,
            $$d^\prime(x^\prime,z^\prime)=
            \lim_{n\to\infty}{d(x_n,z_n)}\leq
            \lim_{n\to\infty}{d(x_n,y_n)}+
            \lim_{n\to\infty}{d(y_n,z_n)}
            =d^\prime(x^\prime,y^\prime)+
            d^\prime(y^\prime,z^\prime).$$
            So, $d^\prime$ satisfies the definition of a metric.
    
            \vspace*{3mm}
            \item Now, we construct an isometry
            $T:X\to W$ where $W$ is a dense subset of
            $\xprime$. Let $T$ be a function that takes
            each element to the
            equivalence
            class $x^\prime$ in $\xprime$ of the
            Cauchy sequence $\{x\}_1^\infty=
            (x,x,x,\ldots)$ associated with that element.
            Then $T$ is an isometry since for each
            $x$, $y\in X$,
            $$d^\prime(T(x),T(y))=\lim_{n\to\infty}
            {d(x_n,y_n)}=d(x,y).$$
            We note that isometry is injective map and
            if $W=T(X)$, then $T:X\to W$ is surjective. So,
            $W$ and $X$ are isometric.
    
            \vspace*{3mm}
            We need to show that $W$ is dense in $\xprime$.
            Let $x^\prime\in \xprime$ be the equivalence
            class of the Cauchy sequence $\{y_i\}_1^\infty$.
            Then for every $\varepsilon>0$, there exists $N
            \in \mathbb{N}$ such that
            $d(y_n,y_m)<\varepsilon$ for all $m$, $n>N$.
            Let $z=y_{N+1}$. Then, if $z^\prime$ is the image
            of $z$ under $T$,
            $$d^\prime(y^\prime,z^\prime)=
            \lim_{n\to\infty}{d(y_n-z)}<\varepsilon.$$
            Thus we see that every open neighborhood
            around the point $x^\prime$ in $\xprime$
            contains a point $z^\prime$ of $W$ and so,
            $W$ is dense in $\xprime$.

            \vspace*{3mm}
            \item We now show that $\xprime$ is a complete
            metric space. Let $\{x_i^\prime\}_1^\infty$
            be a Cauchy sequence in $\xprime$. Since
            $W$ is dense in $\xprime$, every open
            neighborhood in $\xprime$ contains a point of $W$.
            So, for each $i$,
            there exists $z_i^\prime\in W$ such that
            $$d^\prime(x_i^\prime,z_i^\prime)<1/n$$
            Then
            $$d^\prime(z_j^\prime,z_i^\prime)\leq
            d^\prime(z_j^\prime,x_j^\prime)+
            d^\prime(x_j^\prime,x_i^\prime)+
            d^\prime(x_i^\prime,z_i^\prime)<
            1/j+1/i+
            d^\prime(x_j^\prime,x_i^\prime)$$
            So, since $\{x_i^\prime\}_1^\infty$
            is Cauchy, as $i$, $j\to \infty$,
            $d^\prime(z_j^\prime,z_i^\prime)\to
            0$. So the sequence
            $\{z_i^\prime\}_1^\infty$ is Cauchy in $\xprime$.
            Since $T$ is an isometry,
            we see that the sequence
            $T^{-1}(\{z_i^\prime\}_1^\infty)=(
                T^{-1}(z_1^\prime),T^{-1}(z_2^\prime),
                T^{-1}(z_3^\prime),
                \ldots
            )=(z_1,z_2,z_3,\ldots)$ is Cauchy in $X$.
            Let $x^\prime$ be the equivalence class of
            the Cauchy sequence $(z_1,z_2,z_3,\ldots)$.
            We now show that $x^\prime$ is the limit
            of our Cauchy sequence
            $\{x_i^\prime\}_1^\infty$ in $\xprime$.
            We have
            $$d^\prime(x_i^\prime,x^\prime)
            \leq d^\prime(x_i^\prime,z_i^\prime)+
            d^\prime(x^\prime,z_i^\prime)
            <1/n+d^\prime(x^\prime,z_i^\prime)$$
            for $z_i^\prime\in W$. Then, since
            $x^\prime$ is the equivalence class of
            the Cauchy sequence $(z_1,z_2,z_3,\ldots)$
            and $z_i^\prime$ is the equivalence
            class of $(z_i,z_i,z_i,\ldots)$,
            $d^\prime(x^\prime,z_i^\prime)=
            \lim_{n\to\infty}{d(z_n,z_i)}$. Then

            $$d^\prime(x_i^\prime,x^\prime)<
            1/n+\lim_{n\to\infty}{d(z_n,z_i)}.$$
            The right hand side goes to zero as $n,i\to
            \infty$. So the Cauchy sequence
            $\{x_i^\prime\}_1^\infty$ is convergent in
            $\xprime$ and $\xprime$ is complete.

            \vspace*{3mm}
            \item Now, we show that the space $\xprime$ is
            unique upto isometry. If $(\tilde{X},\tilde{d})$
            is another space that contains a dense
            subset $\tilde{W}$ isometric to $X$, then
            for any $\tilde{x}$, $\tilde{y}\in
            \tilde{X}$, we have sequences $\{\tilde{x}_n\}
            _1^\infty$
            and $\{\tilde{y}_n\}
            _1^\infty$ in $\tilde{W}$ such that
            $\tilde{x}_n\to \tilde{x}$ and 
            $\tilde{y}_n\to \tilde{y}$ with
            $$\tilde{d}(\tilde{x},\tilde{y})=
            \lim_{n\to\infty}
            {\tilde{d}(\tilde{x}_n,\tilde{y}_n)}.$$
            Since $W$ and $\tilde{W}$ are isometric
            and the closure of $W$ in $\xprime$ is
            $\xprime$ itself, $\xprime$ and $\tilde{X}$
            must be isometric.
            

        \end{enumerate}
    \end{proof}

\end{mybox}
 
 
\end{enumerate}
\end{document}