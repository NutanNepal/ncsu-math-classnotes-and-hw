\documentclass[12pt]{article}
\usepackage[]{blindtext}
\usepackage[letterpaper, total{216mm, 279mm}]{}
\usepackage{amssymb,amsmath,amsfonts,verbatim}
\usepackage[breakable, skins]{tcolorbox}
\usepackage[parfill]{parskip}
\usepackage[english]{babel}
\usepackage{mathtools, amsthm}
\usepackage{amsfonts}
\usepackage{amssymb}
\usepackage{mathrsfs}
\usepackage{verbatim}
\usepackage{graphicx}

\usepackage{listings}
\usepackage{color}

\definecolor{dkgreen}{rgb}{0,0.6,0}
\definecolor{gray}{rgb}{0.5,0.5,0.5}
\definecolor{mauve}{rgb}{0.58,0,0.82}

\lstset{frame=tb,
  language=Java,
  aboveskip=3mm,
  belowskip=3mm,
  showstringspaces=false,
  columns=flexible,
  basicstyle={\small\ttfamily},
  numbers=none,
  numberstyle=\tiny\color{gray},
  keywordstyle=\color{blue},
  commentstyle=\color{dkgreen},
  stringstyle=\color{mauve},
  breaklines=true,
  breakatwhitespace=true,
  tabsize=3
}


\newcommand{\rl}{\mathbb{R}}
\newcommand{\id}{\text{id}}
\newcommand{\dprime}{{\prime\prime}}
\newcommand{\xprime}{X^\prime}
\newcommand{\mR}{\mathbb{R}}
\newcommand{\ds}{\displaystyle}
\newcommand{\al}{\alpha}
\newcommand{\iv}{\mathbf{I}(V)}
\newcommand{\is}{\mathbf{I}(S)}
\newcommand{\qedbox}{$\hfill\blacksquare$}
\newcommand{\nindent}{.5pt}
\newcommand{\mdeg}{\text{multideg}}
\newcommand{\lt}{\text{LT}}
\newcommand{\lm}{\text{LM}}
\newcommand{\lc}{\text{LC}}
\newcommand{\znz}{\mathbb{Z}^n_{\geq 0}}

\newcommand{\noverline}[1]{
    \kern\nindent\overline{\kern-\nindent#1\kern-\nindent}\kern\nindent}

\newtcolorbox{mybox}[0]{
    arc=0mm, enhanced, frame hidden, breakable,
}

\setlength{\topmargin}{-.65in}
\setlength{\textwidth}{174mm} 
\setlength{\textheight}{230mm}
\setlength{\oddsidemargin}{-5mm}

\title{Computer Algebra 522\\
\large Homework 2
}
\author{Nutan Nepal}

\begin{document}

\maketitle
\makebox[\linewidth]{\rule{180mm}{1pt}}
\vspace{.1in}


Programming Exercise 1
\begin{mybox}
    \begin{lstlisting}
        S.<x,y> = PolynomialRing(QQ,order='deglex')
        def multidiv(f, listf):
            listofqs = []
            p=f
            for fs in listf:
                q, r = p.quo_rem(fs)
                listofqs.append(q)
                p=r
            return (listofqs, p)

        def buchbergerAlgorithm(listf):
            listg = listf
            while True:
                gTemp = listg
                for p in gTemp:
                    for q in gTemp:
                        if p != q:
                            gamma = lcm(p.lt(),q.lt())
                            s = p*gamma.quo_rem(p.lt())[0] - q*gamma.quo_rem(q.lt())[0]
                            r = multidiv(s, gTemp)[1]
                            if r != 0:
                                listg.append(r)
                if listg == gTemp:
                    break

            return listg

        f = x^3-2*x*y
        g = x^2*y-2*x*y^2+x
        buchbergerAlgorithm([f,g])

    \end{lstlisting}
\end{mybox}

Programming Exercise 2
\begin{mybox}
    \begin{lstlisting}
        def minimalizeGB(listg):
            gTemp = listg
            for gi in gTemp:
                gi = gi*(1/gi.lc())
            for gj in gTemp:
                for gi in gTemp:
                    if gi != gj and (gi.lt()).quo_rem(gj.lt())[1]==0:
                        gTemp.remove(gi)

            return gTemp


        def reducedGB(listg):
            gTemp = minimalizeGB(listg)
            reducedBasis = []
            for gi in gTemp:
                temp = 0
                while True:
                    s = gi.lt()
                    if s == 0: break
                    flag = 0
                    for gj in gTemp:
                        if gi != gj and s.quo_rem(gj.lt())[1] == 0: flag = 1
                    if not flag: temp = temp + s
                    gi = gi - s
                reducedBasis.append(temp)
                
            return reducedBasis
    \end{lstlisting}
\end{mybox}
\end{document}