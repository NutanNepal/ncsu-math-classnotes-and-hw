\documentclass[12pt]{article}
\usepackage[]{blindtext}
\usepackage[letterpaper, total{216mm, 279mm}]{}
\usepackage{amssymb,amsmath,amsfonts,verbatim}
\usepackage[breakable, skins]{tcolorbox}
\usepackage[parfill]{parskip}
\usepackage[english]{babel}
\usepackage{mathtools, amsthm}
\usepackage{amsfonts}
\usepackage{amssymb}
\usepackage{mathrsfs}
\usepackage{verbatim}

\newtheorem{notes}{Notes}[section]
\newtheorem{prob}[notes]{Problems}
\newtheorem{thm}{Theorem}[section]
\newtheorem{cor}[thm]{Corollary}
\newtheorem{lem}[thm]{Lemma}
\newtheorem{defn}[notes]{Definition}
\newtheorem{rem}[notes]{Remark}
\newtheorem{prop}[thm]{Proposition}

\newcommand{\rl}{\mathbb{R}}
\newcommand{\id}{\text{id}}
\newcommand{\dprime}{{\prime\prime}}
\newcommand{\xprime}{X^\prime}
\newtcolorbox{mybox}[2][]{
    arc=0mm, enhanced, frame hidden, breakable
}
\newcommand{\qedbox}{$\hfill\blacksquare$}

\setcounter{MaxMatrixCols}{10}

\pagestyle{empty}
\setlength{\topmargin}{-.65in}
\setlength{\textwidth}{190mm} 
\setlength{\textheight}{240mm}
\setlength{\oddsidemargin}{-15mm} 
\setlength{\evensidemargin}{-15mm}
\parindent=0pt

\title{Introduction to Manifold Theory \\
\large Homework 1
}
\author{Nutan Nepal}
\newcommand{\mR}{\mathbb{R}}
\newcommand{\ds}{\displaystyle}
\newcommand{\al}{\alpha}

\begin{document}
\maketitle
 \underline{}\hrulefill
\vspace{.1in}

\begin{enumerate}

\item Prove that the open disks $D_r(p)$ are 
open subsets of $\rl^n$.

\begin{mybox}

  To prove that the open disks $D_r(p)$
  are open subsets of $\rl^n$, we show that
  for every point $x\in D_r(p)$ we have another
  open disk $D_{\epsilon}(x)$, $\epsilon>0$ such that
  $D_{\epsilon}(x)\subset D_r(p)$.

  \setlength{\parskip}{3mm}
  For any $x\in D_r(p)$ such that $\delta=d(x,p)<r$,
  we take $0<\epsilon<r-\delta$. Then we see that
  for all $y\in D_{\epsilon}(x)$
  $$d(p,y)\leq d(p,x)+d(x,y)<\delta+\epsilon<
  \delta+r-\delta=r.$$
  Hence, $y\in D_r(p)$ for all $y\in D_{\epsilon}(x)$
  which implies that $D_{\epsilon}(x)\subset D_r(p)$.
  So, $D_r(p)$ is an open subset.
\end{mybox}


\item Prove the second part of Proposition 2.17
 (a function $f : \rl^n\to \rl^m$ is continuous
 everywhere if and only if for all open subsets
 $V$ of $\rl^m$, the preimage $f^{-1}(V)$ of $V$
 under $f$ is open in $\rl^n$).
 
\begin{mybox}

  We prove for $\Leftarrow$ : If for all open subsets
  $V$ of $\rl^m$ the preimage $f^{-1}(V)$ of $V$
  under $f$ is open in $\rl^n$, then $f$ is continuous.

  \setlength{\parskip}{3mm}
  Let $V\subset \rl^m$ be an open subset such that
  $f(x)\in V$. Then we have an open disk
  $D_{\epsilon}(f(x)) \subset V$.
  As the disk $D_{\epsilon}(f(x))$ is open
  in $\rl^m$, we have an open set
  $f^{-1}(D_{\epsilon}(f(x)))
  \subset \rl^n$ which contains $x$. Then we can find
  a $\delta>0$ such that
  $D_{\delta}(x)\subset f^{-1}(D_{\epsilon}(f(x)))$.
  That is, for every $\epsilon$-ball around $f(x)$,
  we can find a $\delta$-ball around $x$ such that
  $$y\in D_{\delta}(x)\implies f(y)\in D_{\epsilon}(f(x))$$
  for some $y$. Hence $f$ is continuous.
\end{mybox}
  
\item  Show that a composition of continuous functions
$\rl^n\xrightarrow{f}\rl^m\xrightarrow{g}\rl^k$ is continuous.

\begin{mybox}

  Since $g$ is continuous, we have $g^{-1}(V)$ open for
  all open set $V\subset \rl^k$. Similarly we have $f$
  continuous, so $f^{-1}(U)$ is open for all open sets
  $U\subset \rl^m$. Then, $f^{-1}(g^{-1}(V))=
  (g\circ f)^{-1}(V)$ is open for all open sets $V$ in
  $\rl^k$. Hence the composition is continuous.
\end{mybox}
 
 
\item Show that a function $f : X \to Y$ between sets is
invertible if and only if it is bijective.

\begin{mybox}

  A function $f:X\to Y$ is invertible if there exists a
  function $g:Y\to X$ such that $f\circ g= \id_Y$ and
  $g\circ f=\id_X$.
  \begin{enumerate}
    \item[i.]
      $f$ invertible $\implies$ $f$ bijective\\
      Note that since $\id_Y$ is surjective, $f$ must be
      surjective. Now for injectivity, we observe that
      if $f(x) =f(y)$ then $$x=(g\circ f)(x)
      =g(f(x))=g(f(y))=(g\circ f)(y)=y.$$
      Hence, $f$ is bijective.
    \item[ii.]
      $f$ bijective $\implies$ $f$ invertible\\
      Since $f$ is both injective and surjective,
      we define a function $g:Y\to X$ by
      $g(y)=x$ whenever $f(x)=y$. Note that $g$ is 
      well-defined since there exists only one $y$ for
      each $x$. Then, for all
      $x\in X$,
      $g(f(x)) = g(y)=x \implies g\circ f=\id_X$.
      Similarly, for all $y\in Y$, $f(g(y))=f(x)=y
      \implies f\circ g=\id_Y$. So, $f$ is invertible.
  \end{enumerate}
\end{mybox}


\item Show that the product topology on a product
$X \times Y$ of topological spaces is a valid
topology.

\begin{mybox}

  $X$ and $Y$ are topological spaces.
  We define a set $U\subset X\times Y$ to be open if
  for all $(x,y)\in U$ we have open neighborhoods $U_x
  \subset X$ and $U_y\subset Y$ such that
  $U_x\times U_y\subset U$.

  \setlength{\parskip}{3mm}
  \begin{enumerate}
    \item[i.] Clearly the null set $\phi$ and the whole set
      $X\times Y$ are open since $X$ is open in $X$ and
      $Y$ is open in $Y$.

    \item[ii.] Arbitrary union $\bigcup
    {U_\alpha}$ of open sets is open.
    
    \setlength{\parskip}{2mm}
    Let $(x,y)$ be an
    arbitrary point in $\bigcup{U_\alpha}$, then
    $(x,y)\in U_i$ for some $i$. Then, by definition,
    there are open neighborhoods $U_x
    \subset X$ and $U_y\subset Y$ such that
    $U_x\times U_y\subset U_i \subset \bigcup{U_\alpha}$.

    \item[iii.] Finite intersection $U_i\cap U_j$ of
      open sets is open.
      
      \setlength{\parskip}{2mm}
      Let $(x,y)$ be an arbitrary
      point of $U_i\cap U_j$, then $(x,y)\in U_i$ and
      $(x,y)\in U_j$. Then, by definition,
      there are open neighborhoods $U_{ix}, U_{jx}
      \subset X$ and $U_{iy}, U_{jy}\subset Y$ such that
      $U_{ix}\times U_{iy}\subset U_i$ and
      $U_{jx}\times U_{jy}\subset U_j$. Then
      $$U_i\cap U_j
      \supset \left(U_{ix}\times U_{iy}\right)
      \cap\left(U_{jx}\times U_{jy}\right)
      =\left(U_{ix}\cap U_{jx}\right)
      \times\left(U_{iy}\cap U_{jy}\right)
      \ni (x,y).$$
      Since $\left(U_{ix}\times U_{jx}\right)$
      and $\left(U_{iy}\times U_{jy}\right)$ are open
      in $X$ and $Y$ respectively, we see that
      $U_i \cap U_j$ is open.

  \end{enumerate}
\end{mybox}

    
\item Verify the three basic properties of closed sets
that correspond to the three axioms for open sets.
  
\begin{mybox}
  
  We define a set $V\subset X$ to be closed if its
  complement $V^c$ is open in $X$.

  \setlength{\parskip}{2mm}
  \begin{enumerate}
    \item[i.] The null set $\phi$ and the whole set
      $X$ are closed.

      \setlength{\parskip}{2mm}
      $\phi^c=X$ and $X^c=\phi$ which are open in $X$.
    \item[ii.] Arbitrary intersection $\bigcap{V_\alpha}$
      of closed sets is closed.

      \setlength{\parskip}{2mm}
      Here we use the set-theoretic fact that
      \begin{equation}
        \left(\bigcup U_\beta\right)^c=
      \bigcap U_\beta^c \label{setfact}
      \end{equation}
      where $\{U_\beta\}$ is the collection of
      indexed sets. Since each sets $V_\alpha$
      are closed, we write $V_\alpha$ as $U_\alpha^c$
      where $U_\alpha$ is an open set of $X$.
      Then from \ref{setfact} we have
      \begin{equation}
        \bigcap V_\alpha=\bigcap U_\alpha^c=
        \left(\bigcup U_\alpha\right)^c
      \end{equation}
      Hence, since $\bigcup U_\alpha$ is open in $X$,
      $\bigcap V_\alpha$ must be closed.

      \item[iii.] Finite union $V_i\cup V_j$ of closed sets
        is closed.

        \setlength{\parskip}{3mm}
         We have $V_i\cup V_j=U_i^c\cup U_j^c=(U_i\cap U_j)
         ^c$. Since finite intersection of open sets are
         open, we observe that $V_i\cup V_j$ is the complement
         of an open set. Hence $V_i\cup V_j$ is closed.
  \end{enumerate}
\end{mybox}


\item Show that if we have
$X^{\prime\prime}\subset X^{\prime}\subset X$,
then the “subspace of a subspace” topology on
$X^{\prime\prime}$ is the same as the “subspace of the biggest space”
topology on $X^{\prime\prime}$.

\begin{mybox}
  
Suppose $(X,\tau)$ is a topological space.
The subspace topology on $X^{\prime}$ is given by
$$\tau^{\prime}=\{U^{\prime}\subset X^{\prime}:
U^\prime = U\cap X^\prime\text{ for some }U\in\tau\}$$
and the subspace topology on $X^{\prime\prime}$
induced by $\tau^\prime$ is given by
$$\tau^{\prime\prime}=\{U^{\prime\prime}\subset
X^{\prime\prime}:U^{\prime\prime} = U^\prime\cap
X^{\prime\prime}\text{ for some }U^\prime\in\tau^\prime\}.$$
We need to show that $\tau^{\prime\prime}$ is equal to the
the subspace topology on $X^{\prime\prime}$ induced by
$\tau$
$$T=\{U^{\prime\prime}\subset
X^{\prime\prime}:U^{\prime\prime} = U\cap
X^{\prime\prime}\text{ for some }U\in\tau\}.$$

Let $A\in \tau^{\prime\prime}$, then $A=U^\prime\cap
X^{\dprime}$ for some $U^\prime\in\tau^\prime$. Since
$U^\prime=U\cap X^\prime$ for some $U\in\tau$ we have,
$A=U\cap X^\prime\cap X^\dprime=U\cap X^\dprime
\in T$. Hence $\tau^\dprime\subset T$. Similarly,
let $B\in T$, then $B=U\cap X^\dprime$ for some $U\in
\tau$. Since we can write $X^\dprime$ as $X^\prime\cap
X^\dprime$ we have $B=U\cap X^\prime\cap X^\dprime=
U^\prime\cap X^\dprime\in \tau^\dprime$ for some $U^\prime
$ in $\tau^\prime$. Hence $T\subset \tau^\dprime$ which
gives $T=\tau^\dprime$ ending our proof.
\end{mybox}
  

\item Prove the characterization of closed sets of a subspace
as intersections with closed sets of the larger space:
Let $X$ be a topological space and let $X^\prime\subset
X$ be a subspace. Show that a subset $E\subset X^\prime$
is closed if and only if $E = F\cap X^\prime$ for some
closed subset $F$ of $X$.
\begin{mybox}
  
If $E\subset\xprime$ is closed, then $\xprime\char`\\E$ is open
in $\xprime$. Since all open sets of $\xprime$ are in
the form of $U\cap\xprime$ for some open set $U$ of $X$
we have, $\xprime\char`\\E=U\cap\xprime$.
$$E=\xprime\char`\\(\xprime\char`\\E)=\xprime
\char`\\(U\cap\xprime)=\xprime\char`\\U=\xprime\cap U^c
=\xprime\cap F$$
for a closed subset $F$ of $X$.

\setlength{\parskip}{3mm}
If $E = F\cap X^\prime$ for some closed subset $F$
of $X$, then for the open set $U=F^c$ of $X$,
$E=(X\char`\\U)\cap \xprime=(X\cap\xprime)
\char`\\(U\cap\xprime)=\xprime\char`\\(U\cap\xprime)$.
Since $U\cap \xprime$ is an open set of $\xprime$, we see
that $E=\xprime\char`\\(U\cap\xprime)$
is closed in $\xprime$.
\end{mybox}

\end{enumerate}
\end{document}