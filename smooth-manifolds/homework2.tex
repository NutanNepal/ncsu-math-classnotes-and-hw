\documentclass[12pt]{article}
\usepackage[]{blindtext}
\usepackage[letterpaper, total{216mm, 279mm}]{}
\usepackage{amssymb,amsmath,amsfonts,verbatim}
\usepackage[breakable, skins]{tcolorbox}
\usepackage[parfill]{parskip}
\usepackage[english]{babel}
\usepackage{mathtools, amsthm}
\usepackage{amsfonts}
\usepackage{amssymb}
\usepackage{mathrsfs}
\usepackage{verbatim}

\newtheorem{notes}{Notes}[section]
\newtheorem{prob}[notes]{Problems}
\newtheorem{thm}{Theorem}[section]
\newtheorem{cor}[thm]{Corollary}
\newtheorem{lem}[thm]{Lemma}
\newtheorem{defn}[notes]{Definition}
\newtheorem{rem}[notes]{Remark}
\newtheorem{prop}[thm]{Proposition}

\newcommand{\rl}{\mathbb{R}}
\newcommand{\id}{\text{id}}
\newcommand{\dprime}{{\prime\prime}}
\newcommand{\xprime}{X^\prime}
\newtcolorbox{mybox}[2][]{
    arc=0mm, enhanced, frame hidden, breakable
}
\newcommand{\qedbox}{$\hfill\blacksquare$}

\setcounter{MaxMatrixCols}{10}

\pagestyle{empty}
\setlength{\topmargin}{-.65in}
\setlength{\textwidth}{190mm} 
\setlength{\textheight}{240mm}
\setlength{\oddsidemargin}{-15mm} 
\setlength{\evensidemargin}{-15mm}
\parindent=0pt

\title{\textbf{Introduction to Manifold Theory} \\
\large Homework 2
}
\author{Nutan Nepal}
\newcommand{\mR}{\mathbb{R}}
\newcommand{\ds}{\displaystyle}
\newcommand{\al}{\alpha}

\begin{document}
\maketitle
\makebox[\linewidth]{\rule{200mm}{1pt}}
\vspace{1mm}

\begin{enumerate}

\item Do Exercise 2.6 (show that for topological spaces
    $X$, $Y$, $Z$, the “rearrange-the-parentheses” map
    from $(X\times Y)\times Z$ to $X\times(Y\times Z)$
    is a homeomorphism).

\begin{mybox}

Let the function $f:(X\times Y)\times Z \to 
X\times(Y\times Z)$ be defined as
$$f((x,y),z)=f(x,(y,z))$$
where $x$, $y$ and $z$ are respective points of
the topological spaces. We see that the map is
clearly bijective and hence invertible.

\vspace*{2mm}
Now, for each open set $U_x\times (U_y\times U_z)$,
the preimage of $f$
is given by $(U_x\times U_y)\times U_z$
which is open in $(X\times Y)\times Z$. Similarly,
for each open set $(U_x\times U_y)\times U_z$,
the preimage of $f^{-1}$
is given by $U_x\times (U_y\times U_z)$
which is open in $X\times (Y\times Z)$.
Hence $f$ and $f^{-1}$ are both continuous and the
"rearrange the parentheses" map is homeomorphism.
\end{mybox}


\item Do Exercise 2.7 (show that the product topology
    and the usual topology on $\mathbb{R}^n$ agree).
 
\begin{mybox}

    Suppose $\mathscr{P}$ be the product topology and
    $\mathscr{T}$ be the usual topology in $\mathbb{R}^n$.
    Let $U$ be open in the product topology, then
    for all $x=(x_1,\ldots,x_n) \in U$ there exists 
    open neighborhoods $U_i\in \mathbb{R}$ such that
    $x_i\in U_i$ and $U_1\times \cdots 
    \times U_n \subset U$. Then for all $x_i$, there
    exists an open interval $(x_i-\delta_i,x_i+\delta_i)$
    for some $\delta_i>0$.
    Let $\delta=\min\{\delta_i\}$ taken over
    all $i$ from 1 to $n$. Clearly, $\delta>0$  and 
    $x\in B_\delta(x)\subset U_1\times \cdots 
    \times U_n \subset U$. This shows that
    $\mathscr{P}\subset \mathscr{T}$.

    \vspace*{2mm}
    Now let $U$ be open with respect to the usual
    topology. Then for all $x\in U$, there exists an
    open ball $B_\delta(x)$ containing $x$ such that
    $B_\delta(x)\subset U$ for some $\delta>0$.
    Let $\delta_i = \delta/\sqrt{2}$. Then each $x_i$
    is contained in the interval
    $U_i= (x_i-\delta_i,x_i+\delta_i)$ and we see that
    $B_\delta(x) \supset U_1\times \cdots 
    \times U_n$. Then $x\in U_1\times \cdots 
    \times U_n \subset B_\delta(x) \subset U$. Hence
    $U$ is open in the product topology and
     $\mathscr{P}\supset \mathscr{T}$. So we see that the
     two topologies agree.

\end{mybox}
 
 
\item The following exercises are about the
“line with two origins” of Example 2.44, which we will
call $X$.
\begin{enumerate}
    \item[(a)] Show that the construction in Example 2.44
        defines a topology on $X$.
        \begin{mybox}

            The construction in Example 2.44 is reproduced
            below:

            \vspace*{2mm}
            Let $\mathscr{B}$ be the set of subsets of $X$
            that have one of the following two forms:
        \begin{enumerate}
            \item[i.] open intervals $(a, b)
                \subset \mathbb{R}$ (with $a$ and $b$
                finite and $a < b$);
            \item[ii.] sets of the form $((a, b)
                \char`\\{0})\cup {\overline{0}}$
                whenever $a < 0 < b$.
        \end{enumerate}
        Then we declare a subset $U$ of $X$ to be open
        if, for all $x\in U$, there exists a
        subset $B$ of $\mathscr{B}$ with $x \in B$
        and $B \subset U$.

        \vspace*{2mm}
        Let $\mathscr{T}$ be the collection of open sets
        as defined above. We now show that it is a topology.

        \begin{enumerate}
            \item[a.] Clearly, $\phi \in \mathscr{T}$ and
                also $X\in \mathscr{T}$.
            \item[b.] Let $A=\bigcup_i{U_i}$ be the union
                of arbitrary collection of indexed open sets.
                For all $x\in A$ then there exists a $U_i$ such
                that $x\in U_i$. So, there exists a
                subset $B$ of $\mathscr{B}$ with $x \in B$
                and $B \subset U_i\subset A$. Hence, A is open.
            \item[c.] Let $A=U_1\cap U_2$ be the finite
                intersection of open sets of $X$. For any
                $x\in A$ we see that $x\in U_1$ and 
                $x\in U_2$. Then
                there exists a
            subset $B_1$ of $\mathscr{B}$ with $x \in B_1$
            and $B_1 \subset U_1$ and there exists a
            subset $B_2$ of $\mathscr{B}$ with $x \in B_2$
            and $B_2 \subset U_2$. If $x\neq \bar{0}$
            then the problem reduces to $\mathbb{R}$ which implies
            that $A$ is open. If $x=\bar{0}$ then we see that
            $B_1\cap B_2$ is the intersection of open intervals and
            $\bar{0}$ which is again open in $X$.
        \end{enumerate}

        Thus $X$ is a topological space with the topology
        $\mathscr{T}$.
        \end{mybox}

    \item[(b)] Show that with this topology, $X$ is locally
        homeomorphic to $R$.
        \begin{mybox}

            For any point $x\neq \overline{0}$ in $X$,
            we observe that
            there is an open ball $(x-\delta,x+\delta)$
            around $x$ for some $\delta>0$.
            Since any open intervals of
            $\mathbb{R}$ are homeomorphic to $\mathbb{R}$
            itself, we see that $X$ is locally
            homeomorphic
            $\mathbb{R}$ for every point
            $x\neq \overline{0}$.

            \vspace*{2mm}
            Now, when $x=\overline{0}$ we take
            $Y=(-\delta,0)\cup(0,\delta)\cup
            \{\overline{0}\}$ and define a function
            $f:Y\to \mathbb{R}$ by $f(\overline{0})
            =0$ and $f(y)=\tan{(\pi y/2\delta)}$.
            We see that $f$ is invertible, continuous
            and has a continuous inverse and hence is a
            homeomorphism. Thus, $X$ is locally
            homeomorphic to $\mathbb{R}$.
        \end{mybox}

    \item[(c)] Show that $X$ is not Hausdorff.
        \begin{mybox}

            For every $\epsilon>0$, the neighborhood
            $N_\epsilon(0)$
            of the point 0 intersects with the neighborhood
            around the point $\overline{0}$
            non-trivially. So, $X$ is
            not Hausdorff.
        \end{mybox}

\end{enumerate}

\end{enumerate}
\end{document}