\documentclass[12pt]{article}
\usepackage[]{blindtext}
\usepackage[letterpaper, total{216mm, 279mm}]{}
\usepackage{amssymb,amsmath,amsfonts,verbatim}
\usepackage[breakable, skins]{tcolorbox}
\usepackage[parfill]{parskip}
\usepackage[english]{babel}
\usepackage{mathtools, amsthm}
\usepackage{amsfonts}
\usepackage{amssymb}
\usepackage{mathrsfs}
\usepackage{verbatim}

\newtheorem{notes}{Notes}[section]
\newtheorem{prob}[notes]{Problems}
\newtheorem{thm}{Theorem}[section]
\newtheorem{cor}[thm]{Corollary}
\newtheorem{lem}[thm]{Lemma}
\newtheorem{defn}[notes]{Definition}
\newtheorem{rem}[notes]{Remark}
\newtheorem{prop}[thm]{Proposition}

\newcommand{\rl}{\mathbb{R}}
\newcommand{\id}{\text{id}}
\newcommand{\dprime}{{\prime\prime}}
\newcommand{\xprime}{X^\prime}
\newtcolorbox{mybox}[2][]{
    arc=0mm, enhanced, frame hidden, breakable
}
\newcommand{\qedbox}{$\hfill\blacksquare$}

\setcounter{MaxMatrixCols}{10}

\pagestyle{empty}
\setlength{\topmargin}{-.65in}
\setlength{\textwidth}{190mm} 
\setlength{\textheight}{240mm}
\setlength{\oddsidemargin}{-15mm} 
\setlength{\evensidemargin}{-15mm}
\parindent=0pt

\title{\textbf{Introduction to Manifold Theory} \\
\large Homework 2
}
\author{Nutan Nepal}
\newcommand{\mR}{\mathbb{R}}
\newcommand{\ds}{\displaystyle}
\newcommand{\al}{\alpha}

\begin{document}
\maketitle
\makebox[\linewidth]{\rule{200mm}{1pt}}
\vspace{1mm}

\begin{enumerate}

\item Do Exercise 2.6 (show that for topological spaces
    $X$, $Y$, $Z$, the “rearrange-the-parentheses” map
    from $(X\times Y)\times Z$ to $X\times(Y\times Z)$
    is a homeomorphism).

\begin{mybox}

\vspace{30mm}
\end{mybox}


\item Do Exercise 2.7 (show that the product topology
    and the usual topology on $\mathbb{R}^n$ agree).
 
\begin{mybox}

\vspace*{30mm}
\end{mybox}
 
 
\item The following exercises are about the
“line with two origins” of Example 2.44, which we will
call $X$.
\begin{enumerate}
    \item[(a)] Show that the construction in Example 2.44
        defines a topology on $X$.
        \begin{mybox}

            \vspace*{30mm}
        \end{mybox}

    \item[(b)] Show that with this topology, $X$ is locally
        homeomorphic to $R$.
        \begin{mybox}

            \vspace*{30mm}
        \end{mybox}

    \item[(c)] Show that $X$ is not Hausdorff.
        \begin{mybox}

            \vspace*{30mm}
        \end{mybox}

\end{enumerate}

\end{enumerate}
\end{document}