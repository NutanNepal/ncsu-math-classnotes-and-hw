\question{
    The following exercises are about the
“line with two origins” of Example 2.44, which we will
call $X$.
\begin{enumerate}
    \item[(a)] Show that the construction in Example 2.44
        defines a topology on $X$.
    \item[(b)] Show that with this topology, $X$ is locally
        homeomorphic to $R$.
    \item[(c)] Show that $X$ is not Hausdorff.
\end{enumerate}
}

\begin{solution}
    The construction in Example 2.44 is reproduced
            below:

            \vspace*{2mm}
            Let $\mathscr{B}$ be the set of subsets of $X$
            that have one of the following two forms:
        \begin{enumerate}
            \item[i.] open intervals $(a, b)
                \subset \mathbb{R}$ (with $a$ and $b$
                finite and $a < b$);
            \item[ii.] sets of the form $((a, b)
                \char`\\{0})\cup {\overline{0}}$
                whenever $a < 0 < b$.
        \end{enumerate}
    \begin{enumerate}
    \item[(a)] We declare a subset $U$ of $X$ to be open
        if, for all $x\in U$, there exists a
        subset $B$ of $\mathscr{B}$ with $x \in B$
        and $B \subset U$.

        \vspace*{2mm}
        Let $\mathscr{T}$ be the collection of open sets
        as defined above. We now show that it is a topology.

        \begin{enumerate}
            \item[a.] Clearly, $\phi \in \mathscr{T}$ and
                also $X\in \mathscr{T}$.
            \item[b.] Let $A=\bigcup_i{U_i}$ be the union
                of arbitrary collection of indexed open sets.
                For all $x\in A$ then there exists a $U_i$ such
                that $x\in U_i$. So, there exists a
                subset $B$ of $\mathscr{B}$ with $x \in B$
                and $B \subset U_i\subset A$. Hence, A is open.
            \item[c.] Let $A=U_1\cap U_2$ be the finite
                intersection of open sets of $X$. For any
                $x\in A$ we see that $x\in U_1$ and 
                $x\in U_2$. Then
                there exists a
            subset $B_1$ of $\mathscr{B}$ with $x \in B_1$
            and $B_1 \subset U_1$ and there exists a
            subset $B_2$ of $\mathscr{B}$ with $x \in B_2$
            and $B_2 \subset U_2$. If $x\neq \bar{0}$
            then the problem reduces to $\mathbb{R}$ which implies
            that $A$ is open. If $x=\bar{0}$ then we see that
            $B_1\cap B_2$ is the intersection of open intervals and
            $\bar{0}$ which is again open in $X$.
        \end{enumerate}

        Thus $X$ is a topological space with the topology
        $\mathscr{T}$.

    \item[(b)] For any point $x\neq \overline{0}$ in $X$,
            we observe that
            there is an open ball $(x-\delta,x+\delta)$
            around $x$ for some $\delta>0$.
            Since any open intervals of
            $\mathbb{R}$ are homeomorphic to $\mathbb{R}$
            itself, we see that $X$ is locally
            homeomorphic
            $\mathbb{R}$ for every point
            $x\neq \overline{0}$.

            \vspace*{2mm}
            Now, when $x=\overline{0}$ we take
            $Y=(-\delta,0)\cup(0,\delta)\cup
            \{\overline{0}\}$ and define a function
            $f:Y\to \mathbb{R}$ by $f(\overline{0})
            =0$ and $f(y)=\tan{(\pi y/2\delta)}$.
            We see that $f$ is invertible, continuous
            and has a continuous inverse and hence is a
            homeomorphism. Thus, $X$ is locally
            homeomorphic to $\mathbb{R}$.

    \item[(c)] 
    For every $\epsilon>0$, the neighborhood
    $N_\epsilon(0)$
    of the point 0 intersects with the neighborhood
    around the point $\overline{0}$
    non-trivially. So, $X$ is
    not Hausdorff.
    \end{enumerate}
\end{solution}