\question{
    Do Exercise 3.1 (show that if
    $U\subset \rl^n$ and $V\subset \rl^m$
    are open, then a function $f:U\to V$
    is smooth if
    and only if each of its component
    functions $f^i : U\to R$ are smooth).
}

\begin{solution}
    If $f(x_1,\ldots,x_n)=
    (f^1(x_1,\ldots,x_n),\ldots,f^m(x_1,\ldots,x_n))$
    then for $i\in \{1,2,\ldots,n\}$, the first-order
    partial derivative at $p$ is given by the limit
    \begin{equation}
        \lim_{t\to 0}{\frac{f(p+te_i)-f(p)}{t}}
        =\lim_{t\to 0}{\frac{
            (f^1(p+te_i)-f^1(p),\ldots,f^j(p+te_i)-f^j(p)
            ,\ldots,f^m(p+te_i)-f^m(p))
        }{t}}
    \end{equation}
    Then for each $i\in \{1,2,\ldots,n\}$
    and $j\in \{1,2,\ldots,m\}$,
    the partial derivative exists at
    $p\in U$ iff the
    limit
    \begin{equation}
        \lim_{t\to 0}{\frac{
            f^j(p_1,\ldots,p_i+t,\ldots,p_n)
            -f^j(p_1,\ldots,p_n)
        }{t}}
    \end{equation}
    exists at $p$. But the limit on equation (2)
    is the partial derivative of the component
    function
    $f^j$ at $x_i$. Hence, the derivative of $f$ exists at
    $p$ iff each of its component functions
    are differentiable. The partial derivative at
    a point $p$, again, is a function $g:U\to \rl^m$.
    Then, as above, we see that the partial
    derivatives of $g$ exist iff each of its
    component functions are differentiable.

    \vspace*{3mm}
    If $f:U\to V$ is smooth then all $k^{th}$-order
    partial derivatives exist on $U$ for all $k$.
    Then, inductively, from above,
    all $k^{th}$-order
    partial derivatives of each component functions
    also exist on $U$ for all $k$.
    Similarly, if all $k^{th}$-order
    partial derivatives of each component functions
     exist on $U$ for all $k$,
    then $f$ is also smooth.
\end{solution}