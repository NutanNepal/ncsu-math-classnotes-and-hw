\documentclass[12pt]{article}
\usepackage[]{blindtext}
\usepackage[letterpaper, total{216mm, 279mm}]{}
\usepackage{amssymb,amsmath,amsfonts,verbatim}
\usepackage[breakable, skins]{tcolorbox}
\usepackage[parfill]{parskip}
\usepackage[english]{babel}
\usepackage{mathtools, amsthm}
\usepackage{amsfonts}
\usepackage{amssymb}
\usepackage{mathrsfs}
\usepackage{verbatim}

\newtheorem{notes}{Notes}[section]
\newtheorem{prob}[notes]{Problems}
\newtheorem{thm}{Theorem}[section]
\newtheorem{cor}[thm]{Corollary}
\newtheorem{lem}[thm]{Lemma}
\newtheorem{defn}[notes]{Definition}
\newtheorem{rem}[notes]{Remark}
\newtheorem{prop}[thm]{Proposition}

\newcommand{\germs}{C_x^\infty}
\newcommand{\rl}{\mathbb{R}}
\newcommand{\id}{\text{id}}
\newcommand{\dprime}{{\prime\prime}}
\newcommand{\xprime}{X^\prime}
\newtcolorbox{mybox}[2][]{
    arc=0mm, enhanced, frame hidden, breakable
}
\newcommand{\qedbox}{$\hfill\blacksquare$}

\setcounter{MaxMatrixCols}{10}

\setlength{\topmargin}{-.65in}
\setlength{\textwidth}{195mm} 
\setlength{\textheight}{240mm}
\setlength{\oddsidemargin}{-15mm} 
\setlength{\evensidemargin}{-15mm}
\parindent=0pt

\title{Introduction to Manifold Theory \\
\large Homework 6
}
\author{Nutan Nepal}
\newcommand{\mR}{\mathbb{R}}
\newcommand{\ds}{\displaystyle}
\newcommand{\al}{\alpha}

\begin{document}
\maketitle
\makebox[\linewidth]{\rule{200mm}{1pt}}
\vspace{1mm}

\begin{enumerate}

\item Do Exercise 3.7: show that if $X$ is a smooth
    manifold and $f \colon X \to \rl$ is a smooth
    function, then \textbf{any} coordinate representation
    of $f$ is smooth.

\begin{mybox}

    Since $X$ is a smooth $n$-manifold, it is equipped with
    a maximal atlas, say, $\mathcal{A}$ and since $f$ is
    smooth, for all $p\in X$, there exists some chart
    $(U,\varphi)$ with $p\in U$ in some smooth
    atlas $\mathcal{B}$
    representing the smooth structure on $X$,
    such that the composite function
    $$f\circ\varphi^{-1}:D_\varphi^\circ \to \rl $$
    from an open neighborhood $D_\varphi^\circ\subset\rl^n$
    of $\varphi(p)$ to $\rl$ is smooth. We need to show
    that, if $\mathcal{C}$ is any smooth atlas on $X$, then
    for all $(U,\alpha)\in\mathcal{C}$ and $p\in X$ with
    $p\in U$, the map $f\circ \alpha^{-1}:\rl^n\to\rl$ is
    smooth.

    \vspace*{3mm}
    We have $\mathcal{A}\supset \mathcal{B}$ and
    $\mathcal{A}\supset\mathcal{C}$ since $\mathcal{A}$ is
    the maximal atlas of the equivalence class of the
    atlases that $X$ is equipped with.
    Then, for every point $p\in X$,
    there exists $(U,\varphi)\in\mathcal{B}$ and
    $(V,\alpha)\in \mathcal{C}$ with $p\in U, V$
    such that the transition maps $\alpha\circ\varphi^{-1}$
    and $\varphi\circ\alpha^{-1}$ are smooth. Then, the
    function
    $f\circ\alpha^{-1}=(f\circ\varphi^{-1})\circ
    (\varphi\circ\alpha^{-1})$ is the composite of smooth
    functions and hence, is smooth itself. So, any
    coordinate representation of $f$ is smooth.
\end{mybox}

\item Do Exercise 3.8: define the standard vector space
    structure on the stalk of smooth functions
    $C^{\infty}_x$ at a point $x$ of a smooth manifold
    $X$, define the usual product of elements of
    $C^{\infty}_x$ (i.e. of germs of smooth functions
    at $x$), and prove that this product is compatible
    with the vector space structure.
 
\begin{mybox}

    We consider the functions $f_{[p]}, g_{[p]}:
    \germs\to\germs$ by $f_{[p]}([q])=[p\cdot q]$ and
    $g_{[p]}([q])=[q\cdot p]$ and show that they are
    linear maps between the $\rl$-vector spaces.
    Since the product of functions are commutative, we
    see that $f_{[p]}=g_{[p]}$ and it is enough to show
    that one of them, say $f_{[p]}$, is linear.

    \vspace*{3mm}
    We first note that for $q\in[q]$, $r\in[r]$,
    $\alpha\in\rl$ and $x\in X$,
    $$(q+r)(x)=q(x)+r(x)\hspace*{10mm}\text{and}
    \hspace*{10mm}(\alpha q)(x)=\alpha q(x)$$
    and hence, $q+r\in [q+r]=[q]+[r]$ and
    $\alpha q\in [\alpha q]=\alpha[q]$.
    Now,
    \begin{equation*}
        \begin{split}
            f_{[p]}(\alpha[q]+[r])&=f_{[p]}([\alpha q+r])\\
            &=[p(\alpha q +r)]\\
            &=[\alpha p\cdot q+p\cdot r]\\
            &=\alpha[p\cdot q]+[p\cdot r]=
            \alpha f_{[p]}([q])+f_{[p]}([r])
        \end{split}
    \end{equation*}
    Hence, the given functions are linear maps.
\end{mybox}

\item Do Exercise 3.9: for a point $x$ in a smooth
    manifold $X$, show that the set $\mathfrak{m}$ of
    ``germs vanishing at $x$'' gives a maximal ideal of
    $C^{\infty}_x$ and that any other ideal of
    $C^{\infty}_x$ is either equal to all of $C^{\infty}_x$
    or contained in $\mathfrak{m}$.

\begin{mybox}
    
    For any smooth function $f:V\to \rl$ defined on some
    open neighborhood of $x$, if $f(x)\neq 0$ then we
    know that, since $f$ is continuous, there exists some
    open neighborhood $U\ni x$ such that $f(y)\neq 0$ for
    all $y\in U$. Then, the function $1/f:U\to\rl$ defined
    by $(1/f)(x)=1/f(x)$ exists. We note that $1/f$
    is a smooth function and belongs to some germ $[1/f]$
    at $x$. We call such germs $[f]$ as \textbf{unit}
    elements of $C_x^\infty$.
    Then if $I$ is an ideal of the ring $C_x^\infty$
    containing $[f]$, then $[1/f][f]=[1]$ is in the ideal
    $I$. So, $I$ must be the whole ring $C_x^\infty$.

    \vspace*{3mm}
    Now, we take the set $I$ of all germs $[g]$ such that
    $g(x)=0$ and show that it is an ideal.
    For any element $[f]\in C_x^\infty$
    and $[g]\in I$, we have $[f][g]=[f\cdot g]$. Since,
    $(f\cdot g)(x)=f(x)\cdot g(x)=f(x)\cdot 0=0$, we see
    that $[f][g]\in I$. Hence, $I$ is an ideal of
    $C_x^\infty$. Furthermore, since $I$ contains
    \textbf{all} the non-unit elements of $C_x^\infty$,
    $I$ is the unique maximal ideal of $C_x^{\infty}$.
    Hence, $C_x^{\infty}$ is a local ring with the unique
    maximal ideal $I$ as defined above.
    
\end{mybox}

\end{enumerate}
\end{document}