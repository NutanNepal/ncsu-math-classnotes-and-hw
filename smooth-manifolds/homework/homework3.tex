\documentclass[12pt]{article}
\usepackage[]{blindtext}
\usepackage[letterpaper, total{216mm, 279mm}]{}
\usepackage{amssymb,amsmath,amsfonts,verbatim}
\usepackage[breakable, skins]{tcolorbox}
\usepackage[parfill]{parskip}
\usepackage[english]{babel}
\usepackage{mathtools, amsthm}
\usepackage{amsfonts}
\usepackage{amssymb}
\usepackage{mathrsfs}
\usepackage{verbatim}

\newtheorem{notes}{Notes}[section]
\newtheorem{prob}[notes]{Problems}
\newtheorem{thm}{Theorem}[section]
\newtheorem{cor}[thm]{Corollary}
\newtheorem{lem}[thm]{Lemma}
\newtheorem{defn}[notes]{Definition}
\newtheorem{rem}[notes]{Remark}
\newtheorem{prop}[thm]{Proposition}

\newcommand{\rl}{\mathbb{R}}
\newcommand{\id}{\text{id}}
\newcommand{\dprime}{{\prime\prime}}
\newcommand{\xprime}{X^\prime}
\newtcolorbox{mybox}[2][]{
    arc=0mm, enhanced, frame hidden, breakable
}
\newcommand{\qedbox}{$\hfill\blacksquare$}

\setcounter{MaxMatrixCols}{10}

\pagestyle{empty}
\setlength{\topmargin}{-.65in}
\setlength{\textwidth}{190mm} 
\setlength{\textheight}{240mm}
\setlength{\oddsidemargin}{-15mm} 
\setlength{\evensidemargin}{-15mm}
\parindent=0pt

\title{\textbf{Introduction to Manifold Theory} \\
\large Homework 3
}
\author{Nutan Nepal}
\newcommand{\mR}{\mathbb{R}}
\newcommand{\ds}{\displaystyle}
\newcommand{\al}{\alpha}

\begin{document}
\maketitle
\makebox[\linewidth]{\rule{200mm}{1pt}}
\vspace{1mm}

\begin{enumerate}

\item Do Exercise 3.1 (show that if
    $U\subset \rl^n$ and $V\subset \rl^m$
    are open, then a function $f:U\to V$
    is smooth if
    and only if each of its component
    functions $f^i : U\to R$ are smooth).

\begin{mybox}

    If $f(x_1,\ldots,x_n)=
    (f^1(x_1,\ldots,x_n),\ldots,f^m(x_1,\ldots,x_n)))$
    then for $i\in \{1,2,\ldots,n\}$, the first-order
    partial derivative at $p$ is given by the limit
    \begin{equation}
        \lim_{t\to 0}{\frac{f(p+te_i)-f(p)}{t}}
        =\lim_{t\to 0}{\frac{
            (0,\ldots,0,f^i(p_1,\ldots,p_i+t,\ldots,p_n)
            -f^i(p_1,\ldots,p_n)
            ,0,\ldots,0)
        }{t}}
    \end{equation}
    Then for each $i\in \{1,2,\ldots,n\}$,
    the partial derivative exists at
    $p\in U$ iff the
    limit
    \begin{equation}
        \lim_{t\to 0}{\frac{
            f^i(p_1,\ldots,p_i+t,\ldots,p_n)
            -f^i(p_1,\ldots,p_n)
        }{t}}
    \end{equation}
    exists at $p$. But the limit on equation (2)
    is the derivative of the component function
    $f^i$. Hence, the derivative of $f$ exists at
    $p$ iff each of its component functions
    are differentiable. The partial derivative at
    a point $p$ is a function $g:U\to \rl^m$.
    Then, as above, we see that the partial
    derivatives of $g$ exist iff each of its
    component functions are differentiable.

    \vspace*{3mm}
    If $f:U\to V$ is smooth then all $k^{th}$-order
    partial derivatives exist on $U$ for all $k$.
    Then, inductively, from above,
    all $k^{th}$-order
    partial derivatives of each component functions
    also exist on $U$ for all $k$.
    Similarly, if all $k^{th}$-order
    partial derivatives of each component functions
     exist on $U$ for all $k$,
    then $f$ is also smooth.
\end{mybox}


\item Check that Definition 3.6 gives an
    equivalence relation (a binary relation
    that is reflexive, symmetric,
    and transitive) on the set of smooth
    atlases on a given topological manifold
    $X$.
 
\begin{mybox}

    Let $\mathcal{A}=\{(U_\alpha,\varphi_\alpha):
    \alpha\in A\}$ and 
    $\mathcal{B}=\{(V_\beta,\psi_\beta):
    \beta\in B\}$ be smooth atlases on the
    topological manifold $X$ for some indexed
    set $A$ and $B$. We say that $\mathcal{A}
    \sim\mathcal{B}$ if their union is a smooth
    atlas on $X$. The reflexive ($\mathcal{A}
    \sim\mathcal{A}$) and symmetric
    ($\mathcal{A}
    \sim\mathcal{B}\implies\mathcal{B}
    \sim\mathcal{A}$) properties are obvious.
    We now prove for the transitivity of the
    relation $\sim$.

    \vspace*{3mm}
    If $\mathcal{A}\sim\mathcal{B}$ and $\mathcal{B}
    \sim\mathcal{C}$, with
    $\mathcal{C}=\{(W_\gamma,\zeta_\gamma):
    \gamma\in C\}$ for some indexed set $C$, then
    for all $\alpha\in A$ and $\beta
        \in B$ such that $U_\alpha\cap V_\beta$
        is non-empty, the map
        \begin{equation}
            \psi_\beta\circ\varphi_\alpha^{-1}:
            \varphi_\alpha(U_\alpha\cap V_\beta)
            \to \psi_\beta(
                U_\alpha\cap V_\beta
            )
        \end{equation} is smooth, and
        for all $\gamma\in C$ and $\beta
        \in B$ such that $W_\gamma\cap V_\beta$
        is non-empty, the map
        \begin{equation}
            \zeta_\gamma\circ\psi_\beta^{-1}:
            \psi_\beta(W_\gamma\cap V_\beta)
            \to \zeta_\gamma(
                W_\gamma\cap V_\beta
            )
        \end{equation} is smooth.
        Then, we take all $\alpha\in A$ and
        $\gamma\in C$ such that $U_\alpha\cap
        W_\gamma$ is non-empty.
        For each $x\in U_\alpha\cap
        W_\gamma$, we take a chart $(V,\psi)\in
        \mathcal{B}$ that contains $x\in X$.
        Then from (3) and (4), we get the composition
        of smooth maps
        $$\zeta_\gamma\circ\psi^{-1}
        \circ \psi\circ\varphi_\alpha^{-1}
        =\zeta_\gamma\circ\varphi_\alpha^{-1}$$
        from $\varphi_\alpha(U_\alpha\cap
        W_\gamma)\to \zeta_\gamma(U_\alpha\cap
        W_\gamma)$ which is smooth. Analogously,
        we can show that the inverse map
        $$\varphi_\alpha\circ\zeta_\gamma^{-1}:
        \zeta_\gamma(U_\alpha\cap
        W_\gamma)
        \to \varphi_\alpha(
            U_\alpha\cap W_\gamma
        )$$ is also smooth. Hence, this proves
        transitivity and that $\sim$ is an
        equivalence relation.

\end{mybox}
 
 
\item Do Exercise 3.2 (Let $X$ and $Y$ be
topological manifolds equipped with smooth
atlases $\mathcal{A}$ and $\mathcal{B}$
respectively.
Show that $\{U\times V: U\in \mathcal{A},
V\in\mathcal{B}\}$ is a smooth
atlas on the topological manifold
$X\times Y$).

\begin{mybox}

    Let $\mathcal{A}=\{(U_\alpha,\varphi_\alpha):
    \alpha\in A\}$ and 
    $\mathcal{B}=\{(V_\beta,\psi_\beta):
    \beta\in B\}$ be smooth atlases on the
    topological
    manifolds $X$ and $Y$ respectively
    for some indexed set
    $A$ and $B$. Then the product of the smooth
    atlases is defined by
    $$\mathcal{A\times B}=\{(U_\alpha\times
    V_\beta,\varphi_\alpha\times \psi_\beta):
    \alpha\in A, \ \beta\in B\}.$$
    
    \begin{enumerate}
        \item If $X$ is $m$-manifold and $Y$ is $n$-manifold,
        then $\varphi_\alpha\times \psi_\beta:
        U_\alpha\times V_\beta\to \rl^{m+n}$ is
        given by $$(\varphi_\alpha\times \psi_\beta)
        (x,y)=(\varphi_\alpha^1(x),\ldots,
        \varphi_\alpha^m(x),
        \psi_\beta^1(y),\ldots,\psi_\beta^n(y))$$
        for all $x\in U_\alpha$ and $y\in V_\beta$.
        Since, each component functions are smooth,
        we see that $\varphi_\alpha\times \psi_\beta$
        is a smooth function on the product
        topology.

        \vspace*{3mm}
        \item Each $\varphi_\alpha$ is a
        homeomorphism from $U_\alpha$ to an open
        disk $D_\alpha\subset\rl^m$ and 
        $\psi_\beta$ is a
        homeomorphism from $V_\beta$ to an open
        disk $D_\beta\subset\rl^n$. Clearly,
        $D_\alpha\times D_\beta$ is an open
        disk in $\rl^{m+n}$. We define
        $\varphi_\alpha^{-1}\times \psi_\beta
        ^{-1}:D_\alpha\times D_\beta\to
        U_\alpha\times V_\beta$ by
        $$(\varphi_\alpha^{-1}\times \psi_\beta
        ^{-1})(z_1,\ldots,z_m,z_{m+1},\ldots,z_{m+n})
        =(\varphi^{-1}(z_1,\ldots,z_m,z_{m+1}),
        \psi_\beta^{-1}(z_{m+1},\ldots,z_{m+n}))$$
        Since each component functions are
        continuous, we see that
        $\varphi_\alpha^{-1}\times \psi_\beta
        ^{-1}$ is the continuous inverse of the
        map $\varphi_\alpha\times \psi_\beta$.
        Hence $\varphi_\alpha\times \psi_\beta$
        is a homeomorphism from
        $U_\alpha\times V_\beta$ to an open
        disk in $\rl^{m+n}$.

        \vspace*{3mm}
        \item Since every point of $X$ is in
        at least one $U_\alpha$ and every point
        of $Y$ is in $V_\beta$, every point of
        $X\times Y$ is in some $U_\alpha\times
        V_\beta$ (by defnition of the product
        topology).

        \vspace*{3mm}
        \item If $(U_\alpha\times V_\beta)
        \cap (U_{\alpha^\prime}\times V_{\beta
        ^\prime})$ is non-empty,
        then the transition map
        $(\varphi_{\alpha^\prime}
        \times\psi_{\beta^\prime})
        \circ(\varphi_\alpha\times\psi_\beta)
        ^{-1}:(\varphi_\alpha\times\psi_\beta)
        ((U_\alpha\times V_\beta)
        \cap (U_{\alpha^\prime}\times V_{\beta
        ^\prime}))\to 
        (\varphi_{\alpha^\prime}
        \times\psi_{\beta^\prime})
        ((U_\alpha\times V_\beta)
        \cap (U_{\alpha^\prime}\times V_{\beta
        ^\prime}))$ is given by
        $(\varphi_{\alpha^\prime}
        \times\psi_{\beta^\prime})
        \circ(\varphi_\alpha\times\psi_\beta)
        ^{-1}(z_1,\ldots,z_m,z_{m+1},\ldots,
        z_{m+n})=$ 
        
        $$(\varphi_{\alpha^\prime}\circ
        \varphi_\alpha^{-1}(z_1),\ldots,
        \varphi_{\alpha^\prime}\circ
        \varphi_\alpha^{-1}(z_{m}),
        \psi_{\beta^\prime}\circ
        \psi_\beta^{-1}(z_{m+1}),\ldots
        \psi_{\beta^\prime}\circ
        \psi_\beta^{-1}(z_{m+n}))$$
        Since each component functions are smooth,
        we see that the transition map is smooth.
        
    \end{enumerate}
    Hence the product of atlases is an atlas
    in the product of topological manifolds.
\end{mybox}

\item Define $f:\rl^2\to\rl^3$ by
$$f(u,v) =\left(\cos(u^2v)-e^{u-v},\ 
    \frac{u^2-3}{u^2+v^2},\ e^{e^{uv}}\right)$$
Compute the Jacobian matrix of $f$.
        
\begin{mybox}

    $$(Jf)_{(u,v)}=\left[
        \begin{array}{cc}
            -2uv\sin(u^2v)-e^{u-v}
            &u^2\sin(u^2v)+e^{u-v}\\
            \frac{2uv^2-6u}{(u^2+v^2)^2}
            &\frac{2v(3-u^2)}{u^2+v^2}\\
            ve^{e^{uv}+uv}
            &ue^{e^{uv}+uv} 
        \end{array}
    \right].$$
\end{mybox}

\end{enumerate}
\end{document}