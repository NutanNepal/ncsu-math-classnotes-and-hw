\documentclass[12pt]{article}
\usepackage[]{blindtext}
\usepackage[letterpaper, total{216mm, 279mm}]{}
\usepackage{amssymb,amsmath,amsfonts,verbatim}
\usepackage[breakable, skins]{tcolorbox}
\usepackage[parfill]{parskip}
\usepackage[english]{babel}
\usepackage{mathtools, amsthm}
\usepackage{amsfonts}
\usepackage{amssymb}
\usepackage{mathrsfs}
\usepackage{verbatim}

\newcommand{\rl}{\mathbb{R}}
\newcommand{\id}{\text{id}}
\newcommand{\dprime}{{\prime\prime}}
\newcommand{\xprime}{X^\prime}
\newtcolorbox{mybox}[2][]{
    arc=0mm, enhanced, frame hidden, breakable
}
\newcommand{\qedbox}{$\hfill\blacksquare$}

\setcounter{MaxMatrixCols}{10}

\setlength{\topmargin}{-.65in}
\setlength{\textwidth}{190mm} 
\setlength{\textheight}{240mm}
\setlength{\oddsidemargin}{-15mm} 
\setlength{\evensidemargin}{-15mm}
\parindent=0pt

\title{\textbf{Introduction to Manifold Theory} \\
\large Homework 8
}
\author{Nutan Nepal}
\newcommand{\mR}{\mathbb{R}}
\newcommand{\ds}{\displaystyle}
\newcommand{\al}{\alpha}
\newcommand{\atlasA}{\mathcal{A}}

\begin{document}
\maketitle
\makebox[\linewidth]{\rule{200mm}{1pt}}
\vspace{1mm}

\begin{enumerate}

\item Do Exercise 6.9: show that the matrix representative
of a total derivative $Df_p$ of a smooth map $f : X\to Y$,
in the coordinate bases for $TX_p$ and $TY_{f(p)}$
arising from charts $\varphi$ near $p$ and $\psi$ near
$f(p)$, is the Jacobian matrix of
$\psi\circ f\circ\varphi^{-1}$ at $\varphi(p)$.

\begin{mybox}


\end{mybox}

\item Do Exercise 6.11: show that the usual identification
of a finite-dimensional vector
space $V$ with its double dual $V^{**}$ is an isomorphism.
\begin{mybox}
    
    The usual identification is given by the map $T:V\to
    V^{**}$ that takes $v\in V$ to the linear functional
    $T(v)\in V^{**}$ defined by
    $$(T(v))(\alpha)=\alpha(v)$$
    for all $\alpha\in V^*$. We now check that the map
    $T$ is linear and bijective to show that it is an
    isomorphism of vector spaces.
    \begin{enumerate}
        \item For $p\in\rl$ and $x$, $y\in V$, we have
            $(T(px+y))(\al)=\al(px+y)=p(T(x))(\al)+(T(y))
            (\al)$ for all $\al\in V^*$. Thus, $T$ is linear.

        \item $(T(v))(\al)=\al(v)=0$ for all $\al\in V^*$
            implies that $v=0$. Thus the kernel of $T$
            is trivial and hence $T$ is injective.

        \item The dimension of $V$, $V^*$ and $V^{**}$ are
            all same and since $T$ is injective, it follows
            that it is also surjective.
    \end{enumerate}
    Hence, $T$ is an isomorphism of vector spaces.
\end{mybox}

\item Do Exercise 6.12: show that, given a basis for a
finite-dimensional vector space $V$ ,
the corresponding “dual basis” as defined in this
exercise is a basis for the dual space $V^*$.

\begin{mybox}
    
    If $\beta=\{e_1, e_2,\ldots, e_n\}$ is a basis for the
    finite dimensional vector space $V$, then we show
    that the set $\beta^*=\{e_1^*,\ldots,e_n^*\}$ with
    $e_i^*$ defined by
    $$e_i^*(a^1e_1+\cdots+a^ne_n)=a^i$$
    is a dual basis for the dual space $V^*$.

    \vspace*{2mm}
    First we show that the given set is linearly
    independent. Indeed if $\sum_{i=1}^n{a^ie_i^*}=0$,
    then for each $k\in\{1,\ldots, n\}$, we see that
    $(\sum_{i=1}^n{a^ie_i^*})(e_k)=a^k=0.$ Thus the
    set $\beta^*$ is linearly independent. Furthermore,
    since the dimension of the dual space $V^*$ is
    equal to $n$ we see that the set $\beta^*$
    containing $n$ linearly independent element is
    a basis for $V^*$.
\end{mybox}

\item Do Exercise 6.14: roughly, show that the “transforms
like a covector” condition makes
the definition of an abstract element $\alpha\in T^* X_p$
given in the exercise well-defined.

\begin{mybox}
    
\end{mybox}
\end{enumerate}

I could not wrap my head around ques. 1 and 4. Sorry for the
incomplete submission.
\end{document}