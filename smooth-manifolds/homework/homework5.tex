\documentclass[12pt]{article}
\usepackage[]{blindtext}
\usepackage[letterpaper, total{216mm, 279mm}]{}
\usepackage{amssymb,amsmath,amsfonts,verbatim}
\usepackage[breakable, skins]{tcolorbox}
\usepackage[parfill]{parskip}
\usepackage[english]{babel}
\usepackage{mathtools, amsthm}
\usepackage{amsfonts}
\usepackage{amssymb}
\usepackage{mathrsfs}
\usepackage{verbatim}

\newtheorem{notes}{Notes}[section]
\newtheorem{prob}[notes]{Problems}
\newtheorem{thm}{Theorem}[section]
\newtheorem{cor}[thm]{Corollary}
\newtheorem{lem}[thm]{Lemma}
\newtheorem{defn}[notes]{Definition}
\newtheorem{rem}[notes]{Remark}
\newtheorem{prop}[thm]{Proposition}

\newcommand{\rl}{\mathbb{R}}
\newcommand{\id}{\text{id}}
\newcommand{\dprime}{{\prime\prime}}
\newcommand{\xprime}{X^\prime}
\newtcolorbox{mybox}[2][]{
    arc=0mm, enhanced, frame hidden, breakable
}
\newcommand{\qedbox}{$\hfill\blacksquare$}

\setcounter{MaxMatrixCols}{10}

\setlength{\topmargin}{-.65in}
\setlength{\textwidth}{195mm} 
\setlength{\textheight}{240mm}
\setlength{\oddsidemargin}{-15mm} 
\setlength{\evensidemargin}{-15mm}
\parindent=0pt

\title{Introduction to Manifold Theory \\
\large Homework 5
}
\author{Nutan Nepal}
\newcommand{\mR}{\mathbb{R}}
\newcommand{\ds}{\displaystyle}
\newcommand{\al}{\alpha}

\begin{document}
\maketitle
\makebox[\linewidth]{\rule{200mm}{1pt}}
\vspace{1mm}

\begin{enumerate}

\item Do Exercise 3.6: for part 1, for each of
    the six types of quadric surfaces listed in
    the problem, say which you expect to be
    regular level sets and thus have a smooth
    manifold structure. For part 2, give an
    equation defining an example of each of these
    types of quadric surfaces, view the defining
    equation as the level set of a function 
    \[f \colon \rl^3 \to \rl,\]
    and determine algebraically whether the
    level set is regular.

\begin{mybox}

    We expect ellipsoids, elliptic paraboloids,
    hyperbolic paraboloids, hyperboloids of one
    sheet and hyperboloids of two sheets to be
    regular level sets and thus a smooth manifolds.

    \begin{enumerate}
        \item Ellipsoid:
            $$\frac{x^2}{4}+\frac{y^2}{1}+\frac{z^2}{9}
            =1$$
            This is an example of an ellipsoid in
            $\rl^3$ and we can view this as a
            level set of the function
            $f:\rl^3\to\rl$ defined by
            $$f(x,y,z)=
            \frac{x^2}{4}+\frac{y^2}{1}+\frac{z^2}{9}-1$$
            at level 0. The Jacobian given by
            $$\left[\begin{array}{ccc}
                2x/4 &2y &2z/9 
            \end{array}\right]$$
            is 0 only when $(x,y,z)=(0,0,0)$ but
            this point is not our set. So, the Jacobian
            always has rank $\geq 1$ and hence, every point
            is regular and the ellipsoid is has a
            smooth manifold structure.

        \vspace*{3mm}
        \item Elliptic paraboloids:
        $$z=x^2+y^2$$ is a level set of the function
        $f:\rl^3\to\rl$ defined by
        $$f(x,y,z)=
        x^2+y^2-z$$
        at level 0. The Jacobian given by
        $$\left[\begin{array}{ccc}
            2x &2y &1
        \end{array}\right]$$
        always more than 0. Hence this is a regular level
        set.

        \vspace*{3mm}
        \item Hyperbolic paraboloids:
        $$z=x^2-y^2$$ is a level set of the function
        $f:\rl^3\to\rl$ defined by
        $$f(x,y,z)=
        x^2-y^2-z$$
        at level 0. The Jacobian given by
        $$\left[\begin{array}{ccc}
            2x &-2y &-1
        \end{array}\right]$$
        always more than 0. Hence this is a regular level
        set.

        \vspace*{3mm}
        \item Hyperboloids of one sheet:
        $$1=x^2+y^2-z^2$$ is a level set of the function
        $f:\rl^3\to\rl$ defined by
        $$f(x,y,z)=
        x^2+y^2-z^2-1$$
        at level 0. The Jacobian given by
        $$\left[\begin{array}{ccc}
            2x &2y &-2z
        \end{array}\right]$$
        always more than 0 since the critical point
        $(0,0,0)$ is not in our level set.
        Hence this is a regular level
        set.

        \vspace*{3mm}
        \item Hyperboloids of two sheet:
        $$1=-x^2-y^2+z^2$$ is a level set of the function
        $f:\rl^3\to\rl$ defined by
        $$f(x,y,z)=
        -x^2-y^2+z^2-1$$
        at level 0. The Jacobian given by
        $$\left[\begin{array}{ccc}
            -2x &-2y &2z
        \end{array}\right]$$
        always more than 0 since the critical point
        $(0,0,0)$ is not in our level set.
        Hence this is a regular level
        set.

        \vspace*{3mm}
        \item Double cone:
        $$0=x^2+y^2-z^2$$ is a level set of the function
        $f:\rl^3\to\rl$ defined by
        $$f(x,y,z)=
        x^2+y^2-z^2$$
        at level 0. The Jacobian given by
        $$\left[\begin{array}{ccc}
            2x &2y &-2z
        \end{array}\right]$$
        is 0 when $(x,y,z)=(0,0,0)$. This point is in our
        set and hence the Jacobian has rank 0 at $(0,0,0)$.
        So, this is not a regular level set.
    \end{enumerate}
\end{mybox}


\item Do Exercise 3.5: let $\xi$ be the function
    from $n \times n$-matrix-space $\rl^{n^2}$ to
    itself sending a matrix $A$ to $AA^T$. Show that
    the Jacobian of $\xi$ at a point
    $A \in \rl^{n^2}$, viewed as a linear
    transformation from $\rl^{n^2}$ to $\rl^{n^2}$,
    sends a matrix $a \in \rl^{n^2}$ to
    \[(J\xi)_A(a) = a A^T + A a^T.\]
Exercise 3.4 may be useful.
 
\begin{mybox}

    We first note that the transpose of the
    sum of matrices is the sum of transposes of
    the matrices. That is
    $$(a+A)^T=a^T+A^T.$$
    Now, if $\mu:\rl^{n^2}\to\rl^{n^2}$ is the function
    taking a matrix to its transpose, then
    denoting the Jacobian at a point $a\in rl^{n^2}$
    by $T$ we have,

    \begin{align*}
        0=\lim_{t\to 0}
        {\frac{\|\mu(a+tA)-\mu(a)-T(tA)\|}
        {\|(tA)\|}}
        =&\lim_{t\to 0}
        {\frac{\|(a+tA)(a^T+tA^T)-aa^T
        -tT(A)\|}
        {\|(tA,tB)\|}}\\
        =&\lim_{t\to 0}
        {\frac{\|aa^T+tAa^T+taA^T+t^2AA^T-aa^T
        -tT(A)\|}
        {\|(tA,tB)\|}}\\
        =&\lim_{t\to 0}
        {\frac{|t|\|Aa^T+aA^T+tAA^T-T(A)\|}
        {|t|\|A\|}}\\
        =&\lim_{t\to 0}
        {\frac{\|Aa^T+aA^T+tAA^T-T(A)\|}
        {\|A\|}}\\
        =&\frac{\|Aa^T+aA^T-T(A)\|}
        {\|A\|}
    \end{align*}
    The above limit is 0 and hence the numerator
    \[\|Aa^T+aA^T-T(A)\|=0\implies
    Aa^T+aA^T=T(A)\]
    So $T(A)$ evaluated at the point $a$ is
    given by
    \[(J\xi)_A(a) = a A^T + A a^T.\]
\end{mybox}

\end{enumerate}
\end{document}