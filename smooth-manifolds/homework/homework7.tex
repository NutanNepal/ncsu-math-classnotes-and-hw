\documentclass[12pt]{article}
\usepackage[]{blindtext}
\usepackage[letterpaper, total{216mm, 279mm}]{}
\usepackage{amssymb,amsmath,amsfonts,verbatim}
\usepackage[breakable, skins]{tcolorbox}
\usepackage[parfill]{parskip}
\usepackage[english]{babel}
\usepackage{mathtools, amsthm}
\usepackage{amsfonts}
\usepackage{amssymb}
\usepackage{mathrsfs}
\usepackage{verbatim}

\newcommand{\rl}{\mathbb{R}}
\newcommand{\id}{\text{id}}
\newcommand{\dprime}{{\prime\prime}}
\newcommand{\xprime}{X^\prime}
\newtcolorbox{mybox}[2][]{
    arc=0mm, enhanced, frame hidden, breakable
}
\newcommand{\qedbox}{$\hfill\blacksquare$}

\setcounter{MaxMatrixCols}{10}

\setlength{\topmargin}{-.65in}
\setlength{\textwidth}{190mm} 
\setlength{\textheight}{240mm}
\setlength{\oddsidemargin}{-15mm} 
\setlength{\evensidemargin}{-15mm}
\parindent=0pt

\title{\textbf{Introduction to Manifold Theory} \\
\large Homework 7
}
\author{Nutan Nepal}
\newcommand{\mR}{\mathbb{R}}
\newcommand{\ds}{\displaystyle}
\newcommand{\al}{\alpha}
\newcommand{\atlasA}{\mathcal{A}}

\begin{document}
\maketitle
\makebox[\linewidth]{\rule{200mm}{1pt}}
\vspace{1mm}

\begin{enumerate}

\item Do Exercise 3.11: show that if a second-countable
    Hausdorff topological space $X$ admits an
    $n$-dimensional smooth atlas, then $X$ is an
    $n$-dimensional topological manifold
    (and thus a smooth manifold equipped with the
    equivalence class of this atlas).

\begin{mybox}

    It suffices to show that the topological space $X$ is
    locally homeomorphic to an open set of $\rl^n$. Let
    $x\in X$. Since $X$ admits an $n$-dimensional smooth
    atlas, we know that there exists a chart $(U,\varphi)$
    of open set $U$ and a smooth map $\varphi$
    with $x\in U$ such that $\phi(U)$ is an open disk
    in $\rl^n$. Then we see that $x$ has an open
    neighborhood $N_\varepsilon(x)$ such that
    $\varphi(N_\varepsilon(x))$ is an open disk in $\rl^n$.
    Hence, $X$ is locally homeomorphic to $\rl^n$ and is
    a smooth manifold.

\end{mybox}


\item Do Exercise 3.12: in imprecise terms,
    show that if a set $X$ has a "sets-and-bijections
    smooth atlas" then $X$ can be turned into a smooth
    manifold in a natural way.
 
\begin{mybox}

    We first check that the given topology $\mathscr{T}$
    is indeed a valid topology on $X$:

    \vspace*{2mm}
    \begin{enumerate}
        \item For all coordinate patch $(U,\varphi)$ in the
        atlas $\mathcal{A}$, $\varphi(\emptyset\cap U)
        =\emptyset$ and $\varphi(X\cap U)=\varphi(U)$ is open
        in $\rl^n$. Hence $\emptyset$ and $X$ are in the
        topology.

        \item If $V=\bigcup_{\alpha\in A}{U_\alpha}$ is an 
        arbitrary union of indexed (by $A$) open sets, then 
        $$\varphi(V\cap U)=\bigcup_{\alpha\in A}
        {\varphi(U_\alpha)}\cap \varphi(U).$$
        Since eaach $\varphi(U_\alpha)$ and
        $\varphi(U)$ are open in $\rl^n$,
        the arbitrary union $\bigcup_{\alpha\in A}
        {\varphi(U_\alpha)}$ is open and thus $V$ is open
        in $X$.

        \item If $V=U_1\cap U_2$ is intersection of open
        sets of $X$, then $\varphi(V)=\varphi(U_1)\cap
        \varphi(U_2)$ is open in $\rl^n$. Thus
        $V\in\mathscr{T}$.
    \end{enumerate}
    Thus the given topology is indeed a topology. If we
    start with finitely many $U_i$, then we see that all the
    open sets of $X$ are generated by these $U_i$. Thus $X$
    has finitely many (hence, countable) basis and is
    second countable.

    \vspace*{2mm}
    We note that every $U \subset X$ from the chart
    $(U,\varphi)$
    is open since $\varphi(U\cap U)=\varphi(U)$ is open
    in $\rl^n$. Here, $\varphi$ is a bijection and hence has
    an inverse $\varphi^{-1}$. Now we show that this is a
    homeomorphism by showing that both $\varphi$ and
    $\varphi^{-1}$ are continuous. Clealy for every open set
    $V$ in $\varphi(X)$ and a chart $(U,\psi)$ of $X$,
    $\psi\circ\varphi^{-1}$ is a transition map and is smooth.
    Hence $\varphi^{-1}(V)$ is open in $X$, Thus, $\varphi$
    is continuous. Similarly, For open set $V\subset
    \varphi^{-1}(X)$ and chart $(U,\psi)$, $\psi(V\cap U)$
    is open in $\rl^n$. Thus $\varphi\circ\psi^{-1}$
    being a transition function is smooth and thus,
    $\varphi(V)$ is open in $\rl^n$. So $\varphi^{-1}$ is
    also continuous.

\end{mybox}
 
 
\item Do Exercise 3.13: prove the analogue of Exercise
    3.11 where the charts in your smooth atlas are allowed
    to have general smooth manifolds (rather than just
    open disks) as codomains.

\begin{mybox}

    $\atlasA$ is given by the collection of pairs
    $$(\varphi^{-1}(V),\psi\circ\varphi)$$
    where $(V,\psi)$ is a coordinate patch from the atlas
    of the smooth $n$-manifold $X_{U,\varphi}$. $\varphi$ is
    a homeomorphism and hence is continuous. Thus $\varphi^{-1}
    (V)$ is open in $X$. Since
    every $p\in X$ is in some $(U,\varphi)$, $p$ is contained
    in some $\varphi^{-1}(V)$. For each $p\in X$, we have
    an open set $U$ containing $p$ and since $\varphi(U)$ is
    open in the smooth manifold $X_{U,\varphi}$, there is
    an open disk around $\varphi(p)$ contained in $\varphi(U)$.
    Thus we have the map
    $\psi\circ\varphi:X\to\rl^n$ such that the image of
    some open set $U$ of $X$ in $\rl^n$
    an open disk. Thus, $X$ is locally homeomorphic to
    $\rl^n$. (I could not prove for smoothness of transition
    functions.)

    %\vspace*{2mm}
    %We also note that for another pair
    %$(\varphi^{'-1}(V^{'}),\psi^{'}\circ\varphi^{'})
    %\in\atlasA$,
    %the transition map
    %$$(\varphi^{-1}\circ \psi^{-1})\circ (\psi^{'}\circ\varphi
    %^{'}):\varphi^{'-1}(V^{'})\cap\varphi^{-1}(V)
    %\to \varphi^{'-1}(V^{'})\cap\varphi^{-1}(V)$$
\end{mybox}

\item Do Exercise 3.14: prove the analogue of Exercise
    3.12 where the charts in your ``sets-and-bijections
    smooth atlas" are allowed to have general smooth
    manifolds (rather than just open disks) as codomains.

\begin{mybox}

    We note that this topology is the same as the topology
    in problem 3.12 with the modification that
    $rl^n$ is replaced by a general $n$-manifold. The proof
    follows similarly.

    \vspace*{2mm}
    We now check that for every given pair $(U,\varphi)$,
    $U$ is open in $X$ and $\varphi$ is a homeomorphism
    from $U$ to $X_{U,\varphi}$. Clearly, $U$ is open
    since $\varphi(U\cap U)=\varphi(U)$ is open
    in $X_{U,\varphi}$. Now, for every open set
    $V$ in $\varphi(X)$ and a chart $(U,\psi)$ of $X$,
    $\psi\circ\varphi^{-1}$ is a transition map and is smooth.
    Hence $\varphi^{-1}(V)$ is open in $X$, Thus, $\varphi$
    is continuous. Similarly, For open set $V\subset
    \varphi^{-1}(X)$ and chart $(U,\psi)$, $\psi(V\cap U)$
    is open in $X_{U,\varphi}$. Thus $\varphi\circ\psi^{-1}$
    being a transition function is smooth and thus,
    $\varphi(V)$ is open in $X_{U,\varphi}$.
    So $\varphi^{-1}$ is also continuous.
\end{mybox}

\end{enumerate}
\end{document}