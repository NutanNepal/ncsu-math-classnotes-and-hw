\documentclass[12pt]{article}
\usepackage[]{blindtext}
\usepackage[letterpaper, total{216mm, 279mm}]{}
\usepackage{amssymb,amsmath,amsfonts,verbatim}
\usepackage[breakable, skins]{tcolorbox}
\usepackage[parfill]{parskip}
\usepackage[english]{babel}
\usepackage{mathtools, amsthm}
\usepackage{amsfonts}
\usepackage{amssymb}
\usepackage{mathrsfs}
\usepackage{verbatim}

\newtheorem{notes}{Notes}[section]
\newtheorem{prob}[notes]{Problems}
\newtheorem{thm}{Theorem}[section]
\newtheorem{cor}[thm]{Corollary}
\newtheorem{lem}[thm]{Lemma}
\newtheorem{defn}[notes]{Definition}
\newtheorem{rem}[notes]{Remark}
\newtheorem{prop}[thm]{Proposition}

\newcommand{\rl}{\mathbb{R}}
\newcommand{\id}{\text{id}}
\newcommand{\dprime}{{\prime\prime}}
\newcommand{\xprime}{X^\prime}
\newtcolorbox{mybox}[2][]{
    arc=0mm, enhanced, frame hidden, breakable
}
\newcommand{\qedbox}{$\hfill\blacksquare$}

\setcounter{MaxMatrixCols}{10}

\setlength{\topmargin}{-.65in}
\setlength{\textwidth}{190mm} 
\setlength{\textheight}{240mm}
\setlength{\oddsidemargin}{-15mm} 
\setlength{\evensidemargin}{-15mm}
\parindent=0pt

\title{\textbf{Introduction to Manifold Theory} \\
\large Homework 7
}
\author{Nutan Nepal}
\newcommand{\mR}{\mathbb{R}}
\newcommand{\ds}{\displaystyle}
\newcommand{\al}{\alpha}

\begin{document}
\maketitle
\makebox[\linewidth]{\rule{200mm}{1pt}}
\vspace{1mm}

\begin{enumerate}

\item Do Exercise 3.11: show that if a second-countable
    Hausdorff topological space $X$ admits an
    $n$-dimensional smooth atlas, then $X$ is an
    $n$-dimensional topological manifold
    (and thus a smooth manifold equipped with the
    equivalence class of this atlas).

\begin{mybox}

\end{mybox}


\item Do Exercise 3.12: in imprecise terms,
    show that if a set $X$ has a "sets-and-bijections
    smooth atlas" then $X$ can be turned into a smooth
    manifold in a natural way.
 
\begin{mybox}

\end{mybox}
 
 
\item Do Exercise 3.13: prove the analogue of Exercise
    3.11 where the charts in your smooth atlas are allowed
    to have general smooth manifolds (rather than just
    open disks) as codomains.

\begin{mybox}

\end{mybox}

\item Do Exercise 3.14: prove the analogue of Exercise
    3.12 where the charts in your ``sets-and-bijections
    smooth atlas" are allowed to have general smooth
    manifolds (rather than just open disks) as codomains.

\begin{mybox}

\end{mybox}

\end{enumerate}
\end{document}