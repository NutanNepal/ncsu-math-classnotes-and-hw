\documentclass[12pt]{article}
\usepackage[]{blindtext}
\usepackage[letterpaper, total{216mm, 279mm}]{}
\usepackage{amssymb,amsmath,amsfonts,verbatim}
\usepackage[breakable, skins]{tcolorbox}
\usepackage[parfill]{parskip}
\usepackage[english]{babel}
\usepackage{mathtools, amsthm}
\usepackage{amsfonts}
\usepackage{amssymb}
\usepackage{mathrsfs}
\usepackage{verbatim}

\newtheorem{notes}{Notes}[section]
\newtheorem{prob}[notes]{Problems}
\newtheorem{thm}{Theorem}[section]
\newtheorem{cor}[thm]{Corollary}
\newtheorem{lem}[thm]{Lemma}
\newtheorem{defn}[notes]{Definition}
\newtheorem{rem}[notes]{Remark}
\newtheorem{prop}[thm]{Proposition}

\newcommand{\rl}{\mathbb{R}}
\newcommand{\id}{\text{id}}
\newcommand{\dprime}{{\prime\prime}}
\newcommand{\xprime}{X^\prime}
\newtcolorbox{mybox}[2][]{
    arc=0mm, enhanced, frame hidden, breakable
}
\newcommand{\qedbox}{$\hfill\blacksquare$}

\setcounter{MaxMatrixCols}{10}

\setlength{\topmargin}{-.65in}
\setlength{\textwidth}{195mm} 
\setlength{\textheight}{240mm}
\setlength{\oddsidemargin}{-15mm} 
\setlength{\evensidemargin}{-15mm}
\parindent=0pt

\title{\textbf{Introduction to Manifold Theory} \\
\large Homework 4
}
\author{Nutan Nepal}
\newcommand{\mR}{\mathbb{R}}
\newcommand{\ds}{\displaystyle}
\newcommand{\al}{\alpha}

\begin{document}
\maketitle
\makebox[\linewidth]{\rule{200mm}{1pt}}
\vspace{1mm}

\begin{enumerate}

\item Precisely specify a function $f$ and an 
element $c$ of the codomain of $f$ such that
the level set of $f$ at
level $c$ is the ellipsoid in
$\rl^3$ defined by the equation
$$x^2+2y^2+3z^2=4.$$

\begin{mybox}

    The given equation is the level set of
    the function $f:\rl^3\to\rl$ given by
    $$f(x,y,z)=x^2+2y^2+3z^2$$
    at level $c=4$.
\end{mybox}


\item Precisely specify a function $f$ and an
element $c$ of the codomain of $f$ such that the
level set of $f$ at
level $c$ is the graph of
\begin{align*}
    &g:\rl^2\to\rl\\
    &g(x,y)=x^3-y^4.
\end{align*}
 
\begin{mybox}

    The graph of $g$ is given by the set
    $$\{(x,y,z)\in\rl^3:\ z=x^3-y^4\}.$$
    Then, this graph of $g$ is the level
    set of the function $f:\rl^3\to\rl$
    given by
    $$f(x,y,z)=x^3-y^4-z$$
    at level $c=0$.
\end{mybox}
 
 
\item Precisely specify a function $f$ whose image
is the line in $\rl^3$ defined by the system
of equations
$$\begin{dcases*}
    y=2;\\
    x-3z=5.
\end{dcases*}
$$

\begin{mybox}

    In the given line, the second coordinate
    is always 2 and the first coordinate can be
    written as a function of the third coordinate
    as $x=5+3z$. Then, the function $f:\rl\to\rl^3$
    defined as
    $$f(x)=(5+3x,2,x)$$ has the given line as
    its image.
\end{mybox}

\item Do Exercise 3.3: check that the total derivative
$T$ of a function $f:\rl^n\to\rl^m$ at a point $p$ of
$\rl^n$ (if it exists) is unique.
        
\begin{mybox}

    Let $T$ and $T^\prime$ both be the derivative
    of a function $f:\rl^n\to\rl^m$ at the point $p$.
    Then for
    the total derivative $T$,
    $$\lim_{q\to p}{\frac{f(q)-f(p)-T(q-p)}{\|q-p\|}}=0.$$
    Let $h=q-p$, so we have,
    $$\lim_{h\to 0}{\frac{f(p+h)-f(p)-T(h)}{\|h\|}}=0.$$
    Now, for $T$ and $T^\prime$,
    \begin{align*}
        0\leq \lim_{h\to 0}{\frac{\|T(h)-T^\prime(h)\|}{\|h\|}}
        =&\lim_{h\to 0}{\frac{\|f(p+h)-f(p)-T^\prime(h)-
        (f(p+h)-f(p)-T(h))\|}{\|h\|}}\\
        \leq
        &\lim_{h\to 0}{\frac{\|f(p+h)-f(p)-T^\prime(h)\|+
        \|(f(p+h)-f(p)-T(h))\|}{\|h\|}}\\
        &=0
    \end{align*}
    Then, since $tx\to 0$ as $t\to 0$, we can say that,
    for $x\neq 0$ and $h=tx$ we have, (by linearity of
    $T$ and $T^\prime$)
    $$0=\lim_{h\to 0}{\frac{\|T(h)-T^\prime(h)\|}{\|h\|}}
    =\lim_{t\to 0}{\frac{\|T(tx)-T^\prime(tx)\|}{\|tx\|}}
    =\lim_{t\to 0}{\frac{|t|\|T(x)-T^\prime(x)\|}{\|tx\|}}
    =\frac{\|T(x)-T^\prime(x)\|}{\|x\|}.$$
    Hence $\|T(x)-T^\prime(x)\|\implies T=T^\prime$.
    So, the total derivative $T$ is unique.
\end{mybox}

\item Do Exercise 3.4: establish the given formula for
the Jacobian of the “matrix multiplication” map
$$\mu:\rl^{km}\times\rl^{mn}\to\rl^{kn}.$$

\begin{mybox}

    Let $T$ be the Jacobian of the map
    $\mu:\rl^{km}\times\rl^{mn}\to\rl^{kn}$ at
    $(a,b)$. Then $T$
    is given by
    $$\lim_{(A,B)\to 0}
    {\frac{\mu(a+A,b+B)-\mu(a,b)-T(A,B)}{\|(A,B)\|}}=0.$$
    Here $\mu$ is the matrix multiplication map that
    takes $k\times m$ matrix and $m\times n$ matrix.

    Furthermore,
    \begin{align*}
        &\lim_{(A,B)\to 0}
        {\frac{\mu(a+A,b+B)-\mu(a,b)-T(A,B)}
        {\|(A,B)\|}}=0
        \\
        \implies &\lim_{(A,B)\to 0}
        {\frac{\|\mu(a+A,b+B)-\mu(a,b)-T(A,B)\|}
        {\|(A,B)\|}}=0.
    \end{align*}
    Since $tA\to 0$ as $t\to 0$ and $tB\to 0$ as $t\to 0$,
    when $(A,B)\neq 0$, we can write the
    above limit as

    \begin{align*}
        \lim_{t\to 0}
        {\frac{\|\mu(a+tA,b+tB)-\mu(a,b)-T(tA,tB)\|}
        {\|(tA,tB)\|}}
        =&\lim_{t\to 0}
        {\frac{\|ab+tAb+taB+t^2AB-ab-tT(A,B)\|}
        {\|(tA,tB)\|}}\\
        =&\lim_{t\to 0}
        {\frac{|t|\|Ab+aB+tAB-T(A,B)\|}
        {|t|\|(A,B)\|}}\\
        =&\lim_{t\to 0}
        {\frac{\|Ab+aB+tAB-T(A,B)\|}
        {\|(A,B)\|}}\\
        =&\frac{\|Ab+aB-T(A,B)\|}
        {\|(A,B)\|}
    \end{align*}
    Since above limit equals 0, we see that
    the Jacobian
    $T(A,B)$ evaluated at $(a,b)$ equals $Ab+aB$.

\end{mybox}
\end{enumerate}
\end{document}