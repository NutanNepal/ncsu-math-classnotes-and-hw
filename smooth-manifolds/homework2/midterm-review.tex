\documentclass[12pt]{article}
\usepackage[]{blindtext}
\usepackage[letterpaper, total{216mm, 279mm}]{}
\usepackage{amssymb,amsmath,amsfonts,verbatim}
\usepackage[breakable, skins]{tcolorbox}
\usepackage[parfill]{parskip}
\usepackage[english]{babel}
\usepackage{mathtools, amsthm}
\usepackage{amsfonts, pifont}
\usepackage{amssymb}
\usepackage{mathrsfs}
\usepackage{verbatim}
\usepackage{tikz, graphicx}
\usepackage{verbatim}
\usepackage{tikz-cd}
\usepackage[]{geometry, titling, titlesec}

\newcommand{\rl}{\mathbb{R}}
\newcommand{\mc}{\mathbb{C}}
\newcommand{\mz}{\mathbb{Z}}
\newcommand{\id}{\text{id}}
\newtcolorbox{mybox}[1][]{
    arc=0mm, enhanced, frame hidden, breakable
}
\tikzcdset{
    bigcd/.style={
    arrows = {line width=0.8pt},
    cells = {sep=45pt,
    inner xsep=1ex, inner ysep=1ex},
    nodes = {font=\large},
    labels = {font=\normalsize}
    }
}
\newcommand{\qedbox}{$\hfill\blacksquare$}
\newcommand\nindent{.5pt}
\newcommand\noverline[1]{
  \kern\nindent\overline
  {\kern-\nindent#1\kern-\nindent}\kern\nindent
}
\geometry{top=21mm, left=30mm, right=21mm, bottom=21mm}
\pretitle{\begingroup\centering\huge}
\posttitle{\par\endgroup}
\title{Algebraic Topology\\
\large Midterm Review Sheets
}
\author{Nutan Nepal}

\begin{document}
\maketitle
\makebox[\linewidth]{\rule{190mm}{.5pt}}
\vspace{0mm}
\titleformat{\section}
{\large\centering}
{\thesection.}{0.5em}{\underline}

\section{Homotopy and Homotopy Type}
\begin{itemize}
\item Definition and basic intuition for homotopy of maps between topological spaces
    

\vspace*{30mm}
\item Homotopy of maps is an equivalence relation: statement and proof (can assume
    “restriction to closed subsets” lemma without proof)
    

\vspace*{30mm}
\item Homotopy of maps relative to a subset of their domain: definition and basic intuition
    

\vspace*{30mm}
\item Special case of homotopy rel endpoints for paths: homotopy relative to $\{0, 1\}\subset[0, 1]$
    

\vspace*{30mm}
\item Definition of homotopy equivalence between topological spaces
    

\vspace*{30mm}
\item Intuition for homotopy equivalence vs. homeomorphism, e.g. classifying letters of
    the alphabet
    

\vspace*{30mm}
\item Homotopy of maps is preserved when pre- or post-composing with some other map:
    statement and proof
    

\vspace*{30mm}
\item Homotopy equivalence of topological spaces is an equivalence relation: statement and
    proof
    

\vspace*{30mm}
\item Definitions of contractible topological space and nullhomotopic map
    

\vspace*{30mm}
\item Proof that $X$ is contractible iff every map into $X$ is nullhomotopic iff every map out
    of $X$ is nullhomotopic (HW 1)
    

\vspace*{30mm}
\item House with two rooms: basic idea (is it contractible? how would you describe it to a
    friend?)
    

\vspace*{30mm}
\item Definition of retraction of a topological space $X$ onto a subset $A$ (idempotent map $r$
    from $X$ to itself with image $A$)
    

\vspace*{30mm}
\item Definition of deformation retraction of a topological space X onto a subset A (homo-
    topy rel A from the identity on $X$ to a retraction onto $A$)
    

\vspace*{30mm}
\item Pictoral and formulaic descriptions of deformation retractions in examples (HW 1)
    

\vspace*{30mm}
\item Example of a retraction from a space $X$ onto a subset $A$ that is not homotopic rel
    $A$ to the identity map on $X$, i.e. doesn't come as the ending map of a deformation
    retraction (take $X$ = two-point set and send both points to the same point of $X$)
    

\vspace*{30mm}
\item Definition of quotient topology on the set of equivalence classes $X/\sim$ where $X$ is a
    topological space and $\sim$ is any equivalence relation on $X$
    

\vspace*{30mm}
\item Proof that the quotient topology is a valid topology, via the interaction of preimages
    with unions / intersections
    

\vspace*{30mm}
\item Definition of mapping cylinder $M_f$ for a map $f : X\to Y$ of topological spaces
    

\vspace*{30mm}
\item Proof that $M_f$ deformation retracts onto $Y$ (HW 1): you won't be asked to reproduce
    this on the exam, but you should remember that the statement is true.
    

\vspace*{30mm}
\item Statement (no proof) that if $f$ is a homotopy equivalence then $M_f$ deformation
    retracts onto $X$
    

\vspace*{30mm}
\item Corollary that two spaces $X$ and $Y$ are homotopy equivalent if and only if there exists
    a third space $Z$ deformation retracting onto both $X$ and $Y$ ; you should be able to
    prove this assuming the above statements.
    

\vspace*{30mm}
\item Example of a deformation retraction that's not “$M_f$ retracting onto $Y$ :” anything
    where distinct points of the larger space “collide” at an earlier time than $t = 1$, e.g.
    thickened letter $X$ deformation retracting onto a point by first deformation retracting
    onto an ordinary $X$ and then shrinking the legs


\vspace*{30mm}
\end{itemize}

\section{Cell Complexes}
\begin{itemize}
\item Definition of CW complex / cell complex, including definitions of $n$-skeleton and
attaching map $\varphi_\alpha$ for a cell index $\alpha$


\vspace*{30mm}
\item Definitions of finite-dimensionality and finiteness for CW complexes


\vspace*{30mm}
\item Definition of Euler characteristic of a finite CW complex; computing in examples by
counting vertices, edges, faces, higher cells if they exist


\vspace*{30mm}
\item Topological space associated to a finite-dimensional CW complex


\vspace*{30mm}
\item Topological space associated to an infinite-dimensional CW complex


\vspace*{30mm}
\item Constructing CW decompositions of $S^2$ with any allowable number of vertices, edges,
and faces (HW 2)


\vspace*{30mm}
\item Constructing standard CW structure on $\rl P^n$, number of cells in each dimension,
Euler characteristic, definition of $\rl P^\infty$


\vspace*{30mm}
\item Constructing standard CW structure on $\mc P^n$, number of cells in each dimension,
Euler characteristic, definition of $\mc P$


\vspace*{30mm}
\item CW decompositions of closed oriented surfaces of genus $g$; description of these surfaces
by identifying sides of $4g$-gon


\vspace*{30mm}
\item Proof that for a CW complex $X$, the natural maps $X^n\to X^{n+1}$ are injective (HW
2); you won't be asked to reproduce problem 3 on HW 2.


\vspace*{30mm}
\item Understand the statement (no proof) that on the $n$-skeleton $X^n$, the quotient
topology it began its life with agrees with the topology it acquires as a subspace of $X$ due
to the injectivity of the map $X^n\to X$


\vspace*{30mm}
\item Definition of characteristic map $\Phi_\alpha$ and cell $e^n_
\alpha$ for an $n$-cell index $\alpha$


\vspace*{30mm}
\item Statement (no proof) that $\Phi_\alpha$ gives a homeomorphism from int$(D^n)$ to
$e^n_\alpha\subset X$


\vspace*{30mm}
\item Identifying the cells $e^n_\alpha$ in the usual CW structure on the torus: which cells are closed
subsets of the torus? Only the vertex!


\vspace*{30mm}
\item If $X$ has a CW structure then every point of $X$ is in a unique cell: statement only
since that’s all we covered, although the proof isn't so bad.


\vspace*{30mm}
\item Definition of subcomplex of CW complex $X$ as subset $A$ of underlying topological
space that’s closed and consists of a union of cells


\vspace*{30mm}
\item Proposition enabling us to build a CW structure on a subcomplex $A$: technical, won't
be asked statement or proof, just know that if a subset $A$ is closed and consists of a
union of cells then “everything works” when viewing $A$ itself as a CW complex.


\vspace*{30mm}
\item Examples of subcomplexes: $\rl P^k\subset \rl P^n$ in the usual CW structure, same for complex
case; equators of spheres not subcomplexes in the simplest CW structure but you can
choose CW structure so that they're subcomplexes


\vspace*{30mm}
\item Example of a CW complex where the closure of some cell is not a subcomplex:
attaching a closed interval in a sphere to a proper subinterval of a larger closed
interval


\vspace*{30mm}
\end{itemize}
\section{Paths and Homotopy}
\begin{itemize}
\item Definition of straight line homotopy between any two paths with same endpoints in
a convex subset $X$ of $\rl^n$


\vspace*{30mm}
\item Definition of reparametrization used in this section, and homotopy between a path
and any reparametrization


\vspace*{30mm}
\item Definition of composite path $f* g$ (and condition required for $f* g$ to make sense)


\vspace*{30mm}
\item Definition of reverse path $\bar{f}$ (namely: $\bar{f} (s) := f (1- s)$)


\vspace*{30mm}
\item Composition respects homotopy rel endpoints: statement and proof.


\vspace*{30mm}
\item Definition of fundamental group $\pi_1(X, x_0)$ with group operation given by $*$ on
homotopy classes


\vspace*{30mm}
\item The group axioms hold for $\pi_1(X, x_0)$: statement and proof (but you'll be asked at
most a proper subset of the proof)


\vspace*{30mm}
\item Using straight-line homotopies to show $\pi_1(X, x_0)$ is trivial when $X$ is a convex subset
of $\rl^n$ (usual topology)


\vspace*{30mm}
\item Definition of change-of-basepoint isomorphism $\beta_h$ from $\pi_1(X, x_1)$ to $\pi_1(X, x_0)$ given
a homotopy class $[h]$ of paths from $x_0$ to $x_1$; proof that $\beta_h$ is an isomorphism


\vspace*{30mm}
\item Definition of simply connected


\vspace*{30mm}
\item $X$ is simply connected iff for any $x_0$, $x_1$ in $X$ there is a unique homotopy class of
paths joining them: statement and proof


\vspace*{30mm}
\item $\pi_1(X)$ is abelian iff all change-of-basepoint isomorphisms $\beta_h$ depend only on the
endpoints of $h$ (HW 3)


\vspace*{30mm}
\item You won't be asked about problems 2 or 3 on HW 3.


\vspace*{30mm}
\end{itemize}
\begin{itemize}
\section{The Fundamental Group of the Circle}
\item Definition of covering space


\vspace*{30mm}
\item Definition of homotopy lifting property with respect to a space $Y$


\vspace*{30mm}
\item Definition of fibration (homotopy lifting property holds with respect to every space
$Y$ )


\vspace*{30mm}
\item Statement that covering spaces are fibrations with unique lifts of homotopies (no
proof)


\vspace*{30mm}
\item Special case: path lifting principle with unique lifts of paths holds for covering spaces
(take $Y$ to be a single point)


\vspace*{30mm}
\item Corollary about lifting homotopies rel endpoints between paths


\vspace*{30mm}
\item Statement (no proof) that $p(s) := (\cos(2\pi s), \sin(2\pi s))$ gives a covering space $R\xlongrightarrow{p} S^1$


\vspace*{30mm}
\item $\pi_1(S^1)\simeq \mz$: statement and proof assuming all previous lemmas (just the final proof
itself)
\end{itemize}
\end{document}